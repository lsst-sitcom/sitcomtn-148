\section*{Executive Summary}
\textbf{Introduction} \\
The LSST Camera was constructed at the SLAC National Accelerator Laboratory in California, US. Its functionality and performance were evaluated through various integration phases, leading to the identification and mitigation of non-ideal features.

\textbf{Transportation and Installation} \\
In May 2024, LSST Camera was transported from San Francisco to Cerro Pachón, Chile, where the Vera C. Rubin Observatory is being constructed. The Camera was installed in the clean room (White Room) on Level 3 of the observatory. After connecting power and cooling lines and verifying the vacuum performance, the Camera underwent the seventh series of electro-optical (EO) testing (Run 7) from September 2024 to December 2024, collecting 56,066 exposures. This report describes the results from Run 7.

\textbf{Key Testing Points} \\
\begin{itemize}
    \item \textit{Testing Setup Differences}: The EO test setup in the White Room differed from previous setups, primarily using the CCOB Wide Beam projector with a new diffuser system.
    \item \textit{Performance Verification Post-Transportation}: The Camera's performance was reverified after transportation to ensure it matched the pre-transportation checks.
    \item \textit{Optimization of Features}: Previous EO testings identified issues such as persistence and bias instability, which were optimized during Run 7.
    \item \textit{Camera Performance Post-Optimization}: The Camera's performance was evaluated after implementing optimizations.
    \item \textit{Investigation of Other Features}: Additional sensor features were investigated.
    \item \textit{Summary of Run 7 Operations and Issues}: Overall operations and issues encountered during Run 7 were summarized.
\end{itemize}

\textbf{Electro-Optical Setup} \\
\begin{itemize}
    \item \textit{Run 7 Optical Modifications}: The EO test setup in the White Room included a cone attached to the L1 cover and a shroud to create a dark environment. A diffuser was installed to reduce weather patterns and uniformly illuminate the focal plane.
    \item \textit{Projector Spots}: A 4K projector was used for EO testing, illuminating all 3206 amplifiers. The projector's background illumination posed challenges, but adjustments were made to improve contrast.
    \item \textit{Dark Current and Light Leaks}: Initial measurements identified light leaks, which were mitigated by covering gaps and using a blackout fabric shroud.
\end{itemize}

\textbf{Reverification} \\
\begin{itemize}
    \item \textit{Background}: EO camera test data was processed to extract key metrics. The primary concern was maintaining performance characteristics between Run 6 and Run 7.
    \item \textit{Stability Flat Metrics}: Charge transfer inefficiency (CTI) measurements showed consistent performance between runs.
    \item \textit{Dark Metrics}: Dark current measurements indicated improved performance, likely due to better shrouding.
    \item \textit{Bright Defects}: Bright defects were evaluated, showing a small increase in Run 7.
    \item \textit{Flat Pair Metrics}: Linearity and PTC turnoff metrics were consistent, with minor differences in e2v sensors due to voltage changes.
\end{itemize}

\textbf{Camera Optimization} \\
\begin{itemize}
    \item \textit{Persistence Optimization}: Persistence issues were addressed by adjusting operating voltages, significantly reducing residual signals.
    \item \textit{Impact on Full-Well}: Reducing the parallel swing voltage decreased the full-well capacity by 22\%.
    \item \textit{Impact on Brighter-Fatter Effect}: The brighter-fatter effect increased slightly but remained within acceptable limits.
    \item \textit{Sequencer Optimization}: Several sequencer configurations were tested to improve performance, including changes to the clear method and toggling the RG bit during parallel transfer.
\end{itemize}

\textbf{Characterization \& Performance Stability} \\
\begin{itemize}
    \item \textit{Final Characterization}: Key metrics from initial and final Run 7 configurations were compared, showing high consistency.
    \item \textit{Stability Flat Metrics}: Serial and parallel CTI measurements remained stable.
    \item \textit{Dark Metrics}: Dark current and bright defect measurements were consistent, with improvements in some rafts.
    \item \textit{Flat Pair Metrics}: Linearity and PTC turnoff metrics showed minor changes, primarily in e2v sensors.
    \item \textit{PTC Gain}: PTC gain measurements were consistent, with a slight increase in e2v sensors.
    \item \textit{Read Noise}: Read noise remained stable across runs.
    \item \textit{PTC Noise}: PTC noise measurements showed no significant deviations.
    \item \textit{Brighter-Fatter Coefficients}: The brighter-fatter effect increased slightly in e2v sensors.
    \item \textit{Row-Means Variance}: Row-means variance showed a slight decrease in e2v sensors.
    \item \textit{Divisadero Tearing}: Divisadero tearing was significantly reduced in e2v sensors.
    \item \textit{Dark Defects}: Dark defect counts remained consistent.
    \item \textit{Persistence}: Persistence was minimized in e2v sensors, with sub-ADU levels across the LSST bandpass.
\end{itemize}

\textbf{Sensor Features} \\
\begin{itemize}
    \item \textit{Tree Rings}: Tree rings are concentric variations in silicon doping concentration observed in flat images. The centers of the tree rings were measured for all 189 LSST Camera science sensors, showing consistent positions around the average center for each direction.
    \item \textit{ITL Dips}: ITL dips were investigated using spot and rectangle projections. No evidence of ITL dips was found in the lab data, but further investigation is needed for on-sky data.
    \item \textit{Vampire Pixels}: Vampire pixels are characterized by a group of pixels with high photo-response surrounded by pixels with low photo-response. These features were identified on ITL sensors and correlated with phosphorescence.
    \item \textit{Phosphorescence}: Phosphorescence was observed in some ITL sensors, showing a transient signal after exposure. The effect was dependent on the HV Bias state and varied with wavelength and signal level.
\end{itemize}

\textbf{Operations and issues} \\
\begin{itemize}
    \item \textit{Camera Control Network Performance}: Network issues were addressed by simplifying the network configuration and upgrading the control system.
    \item \textit{REB PS Power Trip}: Power trips were mitigated by grounding the Utility Trunk door and implementing ESD controls.
    \item \textit{FES Latch Sensor Failure}: A faulty cable was replaced, resolving the issue.
    \item \textit{PCS Degradation}: Performance degradation was managed through various mitigation strategies, with further analysis required.
    \item \textit{R24/Reb0 and UT Leak Fault}: Low temperature issues were resolved by adjusting the Dynalene temperature and adding a load resistor to the DC-DC converter.
    \item \textit{Data Corruption}: Data corruption issues were resolved by restarting Data store RCEs.
    \item \textit{Guider High Gain Issue}: The guider issue was resolved by power cycling and resetting the RCE.
\end{itemize}

Run 7 successfully demonstrated the LSST Camera's readiness for installation on the Telescope Mount Assembly (TMA). The optimizations and mitigations are implemented balancing both the required specification and the real scientific needs, with robust performance across various metrics. The tests highlighted a few potential issues that need to be addressed. 

\clearpage