\section{Introduction}
The naming of the EO runs was established during initial LSSTCam
integration and testing. The final SLAC IR2 run from November 2023 was
named ``Run 6", while the data acquisitions from Cerro Pachon from September through December 2024 are
considered ``Run 7". Additionally, individual EO acquisitions are tagged
with a run identifier. This is commonly referred to as a Run ID. For all
SLAC runs, the run identifier was a five digit numeric code, while the
Cerro Pachon runs were ``E-numbers" that started with a capital E
followed by a numeric code.

\section{Electro-optical setup}\label{electro-optical-setup}

\subsection{Run 7 Optical modifications}\label{run-7-optical-modifications}

For Run 7 in the white room on Level 3 our electro-optical test setup had a few differences from the Run 6 setup in IR2 at SLAC. One difference was that we were not able to use the CCOB Narrow/Thin beam because we did not have the resources or expertise to configure it. As
such, the majority of the testing was done with the CCOB Wide Beam
projector. We did obtain an additional projector, the 4K projector, partway through Run 7 that will be discussed later. With the CCOB Wide Beam,
we used a cone attached to the L1 cover as well as a shroud to create a
dark environment (Fig.~\ref{fig:LSSTCam_config}).

\begin{figure}[htbp]
\centering
    \includegraphics[width=0.6\textwidth]{figures/Camera_Shroud.jpg} 
    \includegraphics[width=0.35\textwidth]{figures/CCOB_Wide_Shroud.jpg} \\
\caption{(left) Final shroud configuration of LSSTCam in Level 3 to reduce light leaks. (right) CCOB Wide Beam attached to the cone and shrouded.}
\label{fig:LSSTCam_config}
\end{figure}

This allowed us to operate on Level 3 with a dark current of
\textless0.1 ADU/sec with the shutter open. The initial setup of the
CCOB Wide Beam projector was the same as for Run 6, with a minimal ND filter (10\%)
attached to a C-mount lens. One difference was that the f/stop of the lens
was changed from 2.6 to 1.6 (fully open). This was done to try to
reduce the effect of the `weather' and
the `CMB pattern' two effects that we
found in Run 6 and were found to be due to our projection setup (see
\hyperref[Banovetz2024]{{[}Banovetz2024{]}}). While changing  the f/stop  did
reduce the weather pattern, it also caused a much steeper illumination roll-off
across the focal plane. We evaluated the weather pattern and illumination roll-off relative to Run 6.


To both reduce the effects of the
`weather' and
`CMB' but retain uniform illumination
across the focal plane, we installed a diffuser in the cone attached to
L1. Figure~\ref{fig:diffuser} shows the placement of the diffuser within the cone.

\begin{figure}
\centering
\includegraphics[width=0.6\textwidth]{figures/Diffuser.jpg}
\caption{Diffuser installed into the light cone.}
\label{fig:diffuser}
\end{figure}

We found that the diffuser greatly reduced the `weather' (Fig.~\ref{fig:weather}) and eliminated the CMB pattern and more uniformly illuminated the focal plane (Fig.~\ref{fig:roll-off}), with a penalty of decreasing the overall illumination by roughly 35\% even though we fully opened the f-stop.


\begin{figure}[htbp]
\centering
\begin{minipage}{0.45\textwidth}
    \centering
    \includegraphics[width=\linewidth]{figures/Run6_Weather.png}
%    \caption{Full focal plane image showing the fractional difference in Run 6.}
\end{minipage}
%\hfill
\begin{minipage}{0.5\textwidth}
    \centering
    \includegraphics[width=\linewidth]{figures/Run7_WeatherDiffuser.png}
\end{minipage}    
    \caption{Full focal plane fractional difference images for Run 6 (left) and Run 7 (right).}

\label{fig:weather}
\end{figure}

\begin{figure}[htbp]
\centering
\includegraphics[width=0.48\textwidth]{figures/Run7_Illumination.png}
\includegraphics[width=0.48\textwidth]{figures/Run7_DiffuserIllumination.png}
\caption{(left) Illumination across the focal plane from Run 7 without the diffuser (E968) as compared to Run 6. (right) Illumination across the focal plane from Run 7 with the diffuser (E1047) as compared to Run 6.}
\label{fig:roll-off}
\end{figure}



The diffuser was installed for all B protocol and PTC runs (see Section \ref{reverification}) moving
forward, being taken out only for pinhole projection runs and when using the
4K projector.

\subsection{Projector spots}\label{projector-spots}
The addition to the projectors used for EO testing was a 4K
projector, similar to those used in conference rooms. This projector was
first tested at SLAC and arrived at the observatory about halfway through Run 7.
It was used primarily as a spot projector, as the pinhole filter
was not available at that time because of the Filter Exchange System was temporarily inoperable. The projector has an advantage, instead, as it could
illuminate all 3206 amplifiers instead of just the 21 illuminated by the
pinhole projector. Figure \ref{fig:SpotProjector_L3_FP} shows both the setup of the projector on Level 3 and an example of a spot image and the spots across the focal plane. Since the projector does not have fast illumination control, we primarily used the LSST camera main shutter instead of any flashing of the light source (e.g., as we did with the LEDs of the CCOB Wide Beam). One
downside that was found was that the projector illuminated the entire focal plane at some background level, not just the spot regions. The background illumination also had
structure that changed with time and could not be easily subtracted. Figure \ref{fig:SpotProjector_Spots} shows an example of a spot image of just one detector as well as a zoomed in image of a single spot which highlights the background structure. The resulting contrast between the spot and the background was only about a factor of 6. Changing the spot shape to large rectangles for crosstalk
measurements increased the contrast ratio to 30. Examples of the rectangles can be seen in Figure \ref{fig:SpotProjector_Rect}. Though the contrast was much improved, there was still a background structure as can be seen in the saturated image of the figure.

\begin{figure}[htbp]
\centering
\includegraphics[width=0.48\textwidth]{figures/SpotProjector_Level3.jpg}
\includegraphics[width=0.48\textwidth]{figures/SpotProjector_FP.png}
\caption{(left) The spot projector set up on Level 3. (right) An example of an image taken with the spot projector with all the amplifiers containing a spot.}
\label{fig:SpotProjector_L3_FP}
\end{figure}

\begin{figure}[htbp]
\centering
\includegraphics[width=0.48\textwidth]{figures/Spot_Detector_Ex.png}
\includegraphics[width=0.48\textwidth]{figures/Spot_Spot_Ex.png}
\caption{(left) Example of a spot image zooming into a single detector. (right) Example of a spot image zooming further into a single spot. In both the images, there is a clear background structure caused by the projector.}
\label{fig:SpotProjector_Spots}
\end{figure}

\begin{figure}[htbp]
\centering
\includegraphics[width=0.32\textwidth]{figures/Rectange_Detector_Ex.png}
\includegraphics[width=0.32\textwidth]{figures/Rectange_Spot_Ex.png}
\includegraphics[width=0.32\textwidth]{figures/Rectange_Spot_Sat.png}
\caption{(left) Example of a spot image that utilized the rectangle shape, zoomed into a single detector (left), zoomed into the spot (middle) and zoomed into the spot with a saturated image to highlight the background pattern caused by the projector (right).}
\label{fig:SpotProjector_Rect}
\end{figure}

This section describes the spots and rectangle patterns used for tests with the 4K
projector.

\begin{itemize}
\tightlist
\item
  Projector background
\item
  Spots on many amps
\item
  Spots on one amp
\item
  Optical setup
\end{itemize}

\subsection{Dark current and light
leaks}\label{dark-current-and-light-leaks}

This section describes dark current and light leaks in Run 7 testing.

\subsubsection{Light leak mitigation with shrouding the camera
body}\label{light-leak-mitigation-with-shrouding-the-camera-body}

One of the first tests we attempted with LSSTCam was measuring dark
current and sources of light leaks in the camera body. Before beginning we covered gaps between the L1 cover and the gaskets with tape, in accessible locations . Below shows the gaps that we could see between L1 and its cover.

Once these were sealed, we took some initial measurements and then
started to cover the LSSTCam body with a blackout fabric shroud. Figure~\ref{fig:LSSTCam_config} shows the final configuration of the shroud covering the
camera.
We also found light leaks
where the light cone attached to L1 was housed, and from the Utility
Trunk. 


Table~\ref{tab:leak_chasing} includes the observations, the corresponding measured dark
currents, and comments on what changed during the leak chasing.

\begin{longtable}{|l|c|l|l|l|}
\caption{Summary of the 15\,s dark exposures, the different conditions, and the resulting dark current.
Exposure ID is preceded by ``MC\_C202409".  The shroud was in place for each of these measurements.  (``Initial Covering" was just the CCOB cone and around the L1 cover.) \label{tab:leak_chasing}} \\
\hline
\textbf{Exposure} & \textbf{Dark Current} & \textbf{Room Lights} &\textbf{Shutter} & \textbf{Comments} \\

\hline
\endfirsthead
\hline
%\textbf{Exposure ID} & \textbf{Dark Current (e$^-$/s)} & \textbf{Room Lights} & \textbf{Shutter} & \textbf{Comments} \\
\hline
\endhead
\hline
\endfoot
\hline
09\_000012 & 0.16 & Off & Closed & \\
09\_000018 & 0.16 & On & Closed & \\
09\_000038 & 2.94 & On & Open & Initial Covering  \\
09\_000054 & 1.34 & On & Open &  + Blanket over the FCS \\
09\_000072 & 0.41 & On & Open &  + Blanket over AND under the FCS \\
09\_000078 & 0.18 & Off & Open & + Blanket over AND under the FCS \\
10\_000031 & 0.03 & On & Open &  + Blanket over AND under the FCS + UT \\

\end{longtable}


\subsubsection{Filter Exchange System Autochanger light leak
masking}\label{successful-autochanger-light-leaks-masking}

A dedicated light leak study of the Filter Exchange System (FES) Autochanger (AC) was performed during Run 6 at SLAC
in summer 2023 and a localized faint light source of up to
\textasciitilde{}0.04 e$^-$/s/pix was found to be associated with the 24\,V Clean of
the AC.

In the AC this voltage is used to power some probes and all
controllers. In February 2024, as AC-1 was extracted from LSSTCam for
global maintenance, a direct investigation to localize the light
source was performed unsuccessfully. A light source in the AC
was not expected, as in the AC all controllers' LEDs have
been removed, and most electronics are in ``black boxes". Still, two small
probes, which had LEDs that could not be removed, were initially masked
by a black epoxy. As we had doubts about the quality of this masking at
IR wavelengths, we applied extra masking (aluminum black tape) on them during
the Feb 2024 maintenance (on AC 1 and 2).

At the start of Run 7 a new study of the light leak based on 900\,s
dark exposures with the shutter open and the empty frame filter in
place, showed that the AC light leaks were still present (see left hand image of Fig.~\ref{fig:ac-light-leak}). Following this finding, a full review of all the AC hardware powered
by the 24 V dirty was performed, and a candidate was found: the encoders
of the five main motors of the AC had only partial documentation from the
vendor that did not mention the presence of LEDs. After interaction with the
vendor, the encoders were understood to contain \textasciitilde700\,nm LEDs. The hypothesis of \textasciitilde700\,nm LED sources has been
found compatible with the observation as no AC light leaks were detected
using various filters (g, r, and y) in LSSTCam at the start of Run 7 (g, r, and y
filters). A dedicated test in Paris using an AC spare encoder and a
precision photometric set-up allowed identification of the leak in the masking of
those LEDs in the vendor packaging. A complementary masking method based
on a 3D printed part + tape + cable tie was qualified in Paris.  It was
found to mask the light leak and to be safe (all parts correctly secured).

In November 2024, we masked all the lights in the back of the Level 3 white
room (not the part containing LSSTCam) to set up a high-quality dark room
allowing a direct observation with a CMOS camera of the light leak on
the AC2 motor encoders. The level of darkness reached allowed us to
validate the quality of the light masking of the AC encoders. Notice that the
FES-prototype in Paris does not have encoders on the Online
Clamps, so we had to tune/qualify the masking of those encoders directly on the AC 2 at the summit.

For both AC 1 and 2, the encoders of the five motors with the vendor issue on
their LED masking have been successfully enveloped in a light-tight
mask.

We note that the AC was turned off starting on 27 September 2024 at 21:15 UTC in the
first part of Run 7. For the second part of Run 7 (i.e., after
mid-November) the AC was back on: as the AC 1 was back in LSSTCam with
the new light masks in place on the motor encorders, we were able to take a new series of
900\,s darks with the AC turned on and off, confirming that the light leak
associated with the FES was eliminated (see right hand image of Fig.~\ref{fig:ac-light-leak}).

\begin{figure}
\begin{centering}
\includegraphics[width=0.8\textwidth]{figures/AC_LightLeak_study.png}
\caption{ (left) The original impact of the AC light leak on a 900\,s dark difference image (AC on minus AC off). (right) The result after masking the LEDsc of the motor encoders in the AC.  No light associated with the FES is present in 900\,s dark difference image.  \label{fig:ac-light-leak}}
\end{centering}
\end{figure}

\paragraph{Shutter condition impact on
darks}\label{shutter-condition-impact-on-darks}

\paragraph{Filter condition impact on
darks}\label{filter-condition-impact-on-darks}

To investigate how the filter affects the dark current, we took 900 second darks with the available filters in the filter wheel: E1114 (empty filter), E1115 ($g$), E1116 ($y$), and E1117 ($r$). The heat maps of the dark currents from EO pipe can be found in Figure \ref{fig:filter-darkcurrent}. The major effect of including the filters was reducing the glow the AC (see Figure \ref{fig:ac-light-leak}). The global average of the median amplifier dark currents drop from 0.026 e-/sec with the empty filter to 0.0035 e-/sec for $r$, 0.0011 e-/sec for $y$, and 0.00063 e-/sec for $g$. The discrepancy between the filters could be the AC light shines more brightly in the redder wavelengths and even the IR. Unfortunately, we were not able to obtain data with the other 3 filters to confirm this.

\begin{figure}
\begin{centering}
\includegraphics[width=0.48\textwidth]{figures/E1114_Empty_DarkCurrent.png}
\includegraphics[width=0.48\textwidth]{figures/E1115_g_DarkCurrent.png} \\
\includegraphics[width=0.48\textwidth]{figures/E1116_y_DarkCurrent.png}
\includegraphics[width=0.48\textwidth]{figures/E1117_r_DarkCurrent.png}
\caption{ The heat map of the dark current with the empty filter installed (E1114; top left), the $g$ filter installed (E1115; top right), the $y$ filter installed (E1116; bottom left), and the $r$ filter installed (E1117; bottom right)  \label{fig:filter-darkcurrent}}
\end{centering}
\end{figure}

\subsubsection{Final measurements of dark
current}\label{final-measurements-of-dark-current}

\section{Reverification}\label{reverification}

All electro-optical (EO) camera test data is processed through the
\underline{\href{https://github.com/lsst/cp_pipe/tree/main}{calibration products}} and
\underline{\href{https://github.com/lsst-camera-dh/eo_pipe/tree/main}{electro-optical}}
pipelines to extract key metrics from the data run. The key camera
metrics from Run 7, and their comparison to previous runs are discussed
below.




Among the motivations for these measurements, the primary concern is whether LSSTCam has
maintained its performance characteristics between Run 6 and Run 7, since LSSTCam was transported from SLAC to Cerro Pachon.
The testing condition is supposed to be identical; however as described in Section \ref{brighter-fatter-a00-coefficient}, two Rafts have slightly different voltages between two runs.

\subsection{Background}\label{background}

Initial characterization studies performed on LSSTCam during Run 7 primarily used two
image acquisition sequences.

\begin{itemize}
\tightlist
\item
  B protocols: this acquisition sequence consists of the minimal set of
  camera acquisitions for electro-optical testing, including

  \begin{itemize}
  \tightlist
  \item
    Bias images
  \item
    Dark images
  \item
    Flat pairs - flat illumination images (flats) taken at varying flux levels
  \item
    Stability flats - flats taken at constant flux levels
  \item
    Wavelength flats - flats taken with different LEDs
  \item
    A persistence dataset - a saturated flat, followed by several darks
  \end{itemize}
\item
  PTCs (photon transfer curves): this acquisition sequence consists of a
  sequence of flat pairs taken at different flux levels. The flat
  acquisition sequence samples different flux levels at a higher density
  than the B protocol flat sequence, enabling more precise estimates of
  flat pair metrics including pixel covariances (see Fig. \ref{fig:PTC_BProtocol_Comparison}).
\end{itemize}

\begin{figure}[H]
\begin{centering}
\includegraphics[width=0.7\textwidth]{figures/baselineCharacterization/PTC_BProtocol_Comparison.jpg}
	\caption{Flat-pair comparison between PTC and B protocol
\label{fig:PTC_BProtocol_Comparison}}
\end{centering}
\end{figure}

For comparisons between Cerro Pachon EO runs and the final SLAC IR2 equivalents, the following runs are used (see Table~\ref{runTable-b-ptc}).

\begin{table}[h]
\centering
\caption{Reference runs for Run 6 and Run 7 comparisons} \label{runTable-b-ptc}
\begin{tabular}{lll}
\toprule
Run Type & Run 6 & Run 7 \\
\midrule
B Protocol & 13550 & E1071 \\
PTC        & 13591 & E749 \\
\bottomrule
\end{tabular}
\end{table}

\subsection{Stability flat metrics}\label{stability-flat-metrics}

\subsubsection{Charge transfer
inefficiency}\label{charge-transfer-inefficiency}

CTI, or charge transfer inefficiency, measures the fraction of charge that fails to transfer from row to row during readout, and appears as trailing charge in the image area. Consequences of high CTI include loss of charge, distorted signals in the direction of parallel transfer, and reduced sensitivity in low light imaging. CTI measurements are made using the EPER method \citep{2021JATIS...7d8002S}, for which the ratio of the residual charge in the overscan pixels to the total signal charge in the imaging region is evaluated. In the context of LSSTCam, we measure CTI along both the serial and parallel directions.

\paragraph{Serial CTI}\label{serial-cti}

\begin{figure}[H]
\begin{centering}
\includegraphics[width=0.7\textwidth]{figures/baselineCharacterization/13550_E1071_SCTI_EF_43_inset.png}
	\caption{Serial CTI amplifier measurements separated by raft for Run 7 (E1071) and Run 6 (13550)\label{fig:serial-cti}}
\end{centering}
\end{figure}

The CTI along the serial registers of the amplifier segments of the LSSTCam CCDs is consistent between Run 6 and
Run 7 (Fig.~\ref{fig:serial-cti}). Both sensor types show low CTI,
span a range  of \textasciitilde$2 \times 10^{-5}$ \% for e2v sensors, and
by \textasciitilde$4 \times 10^{-6}$  \% for ITL sensors (Fig.~\ref{fig:serial-cti-dist}).

\begin{figure}[H]
\begin{centering}
\includegraphics[width=0.7\textwidth]{figures/baselineCharacterization/SCTI_13550_E1071_diff.png}
\caption{Distributions of differences in serial charge transfer inefficiencies between Run 7 (E1071) and Run 6 (13550), grouped by CCD type.}
\label{fig:serial-cti-dist}
\end{centering}
\end{figure}

\paragraph{Parallel CTI}\label{parallel-cti}

The CTI along the parallel direction is consistent between Run 6 and
Run 7 (Fig.~\ref{fig:parallel-cti}). Both sensor types are found to have extremely low CTI on the order of $10^{-5}$ \%,
and span a range of \textasciitilde$1 \times 10^{-5}$ \% for e2v sensors, and
by \textasciitilde$7 \times 10^{-4}$ \% for ITL sensors (Fig.~\ref{fig:parallel-cti-dist}).

\begin{figure}[H]
\begin{centering}
\includegraphics[width=0.7\textwidth]{figures/baselineCharacterization/13550_E1071_PCTI_EF_43_inset.png}
\caption{Parallel CTI comparison by raft for Run 7 (E1017) and Run 6 (13550).}
\label{fig:parallel-cti}
\end{centering}
\end{figure}

R00 observations

\begin{figure}[H]
\begin{centering}
\includegraphics[width=0.7\textwidth]{figures/baselineCharacterization/PCTI_13550_E1071_diff.png}
\caption{Distributions of differences in parallel charge transfer inefficiencies between Run 7 (E1071) and Run 6 (13550), grouped by CCD type.}
\label{fig:parallel-cti-dist}
\end{centering}
\end{figure}

\subsection{Dark metrics}\label{dark-metrics}

\subsubsection{Dark current}\label{dark-current}

Dark current is the small amount of electrical charge generated in the
absence of light due to thermal activity within the semiconductor
material of a CCD. This effect occurs when electron/hole pairs are thermally released
into the conduction band in the CCD, mimicking the signal that light would
produce. Dark current increases with temperature, so cooling the CCD is
a common method to reduce it in sensitive imaging applications. Dark
current introduces noise into an image, particularly in low-sky background conditions in long exposures.
The measurement of dark includes the dark current and stray light, making them impossible to distinguish each other since they both linearly evolve with time.
In the context
of LSSTCam, we measure dark current from the combined dark images across
all amplifiers as the upper limit.

\begin{figure}[H]
\begin{centering}
\includegraphics[width=0.7\textwidth]{figures/baselineCharacterization/13550_E1071_DARK_CURRENT_MEDIAN_inset.png}
\caption{Dark current comparison by raft for Run 7 (E1071) and Run 6 (13550).}
\label{fig:dark}
\end{centering}
\end{figure}

Unexpectedly, the dark current was significantly less in Run 7 than
Run 6 (Fig.~\ref{fig:dark}). We do not attach particular significance to the finding because this could be the result of improved shrouding on the camera in the Level 3 white room relative to the IR2 clean room SLAC.

\subsubsection{Bright defects}\label{bright-defects}

Bright defects are localized regions or individual pixels that produce abnormally high signal levels, even in the absence of light. These defects are typically caused by imperfections in the semiconductor material or manufacturing process of the CCD. Bright defects can manifest as ``hot pixels" with consistently high dark current, small clusters of pixels with elevated dark current, or as ``hot columns" (pixels along the same column that have high dark current). 

In the context of LSSTCam, we identify and exclude bright pixels from the dark current measurement, with the threshold for a bright defect set at 5 e$^-$/pix/s, above which the pixel/cluster/column is registered as a bright defect. In addition to the bright pixel metric, eo-pipe also computes a bright column metric, which is any region of bright pixels that is contiguous over 50 pixels or more.

\begin{figure}[H]
\begin{centering}
\includegraphics[width=0.7\textwidth]{figures/baselineCharacterization/13550_E1071_BRIGHT_PIXELS_inset.png}
\caption{Bright pixel comparison by raft for Run 7 (E1071) and Run 6 (13550)}
\label{fig:bright}
\end{centering}
\end{figure}

Evaluating the change in defect counts on each amplifier segment between Run 6 and Run 7, and aggregating the amplifiers by the detector manufacturer shows a small increase of bright defects in Run 7 (Fig.~\ref{fig:bright}). Figure~\ref{fig:dark-dist}) displays differences of the measurements. The median values agree well, while there are signs of the positive tail. For ITL sensors, we find that 12\% of the amplifiers have more bright pixels than in Run 6. For e2v sensors, we find 4\% of the amplifiers that have more bright pixels. Despite this, the number of bright defects between runs does not increase for most sensors.

The reason is not totally clear, but the difference in the illumination pattern as described in Section \ref{run-7-optical-modifications} might play a role, which implies that a small number of defects could be involved by optical path.

\begin{figure}[H]
\begin{centering}
\includegraphics[width=0.7\textwidth]{figures/baselineCharacterization/BRIGHT_PIXELS_13550_E1071_diff.png}
\caption{Distributions of differences in bright pixel count per amplifier between Run 7 (E1071) and Run 6 (13550), grouped by CCD type.}
\label{fig:dark-dist}
\end{centering}
\end{figure}

\subsection{Flat pair metrics}\label{flat-pair-metrics}

\begin{figure}[H]
\begin{centering}
\includegraphics[width=0.95\textwidth]{figures/baselineCharacterization/run7PTCsToDate.jpg}
\caption{A comparison of Run 6 and Run 7 PTCs for a central amplifier.}
\end{centering}
\end{figure}

\subsubsection{Linearity and PTC turnoff}\label{linearity-and-ptc-turnoff}

Linearity turnoff and PTC turnoff are two closely related metrics used
to characterize the upper limit of the usable signal range for accurate shape measurements and photometry. Linearity turnoff is the signal level above which the PTC curve deviates from
linearity and is measured for each amplifier segment of each CCD. We have defined the deviation threshold as 2\%.
PTC turnoff refers to the high-signal region of the PTC above which the PTC
variance decreases with increasing signal. This is due to saturation within the pixel wells of the CCDs. While slightly different, both metrics
provide important information about the upper limits of the dynamic
range in our sensors. Linearity turnoff is measured in units of e$^-$,
while PTC turnoff is measured in ADU.

\begin{figure}[H]
\begin{centering}
\includegraphics[width=0.7\textwidth]{figures/baselineCharacterization/13591_E749_LINEARITY_TURNOFF.png}
\caption{A comparison of Run 7 amplifier measurements of linearity turnoff, separated by sensor type. For both sensor types, measurements agree across both runs.}
\end{centering}
\end{figure}

In our linearity turnoff measurements, we find close agreement between
our Run 7 and Run 6 measurements for both ITL and e2v sensors.

\begin{figure}[H]
\begin{centering}
\includegraphics[width=0.7\textwidth]{figures/baselineCharacterization/LINEARITY_TURNOFF_E749_sensorType.png}
\caption{A comparison of Run 7 amplifier measurements of linearity turnoff, separated by sensor type. For both sensor types, linearity turnoff is above the 90k e- specification for a majority of amplifiers. A subset of ITL amplifiers are below the 90k e- threshold, while two e2v amplifiers are below that specification.}
\end{centering}
\end{figure}

Run 7 PTC turnoff measurements agree closely between Run 6 and Run 7, differing by $\leq$ 200 $e^-$ for both ITL and e2v sensors. Notably, they are lower on average for both detector types.

\begin{figure}[H]
\begin{centering}
\includegraphics[width=0.7\textwidth]{figures/baselineCharacterization/PTC_TURNOFF_13591_E749_diff.png}
\caption{A comparison of Run 6 and Run 7 amplifier differences in PTC turnoff, separated by sensor type. For both sensor types, PTC turnoff is very consistent.}
\end{centering}
\end{figure}

\subsubsection{PTC Gain}\label{ptc-gain}

PTC gain is the conversion factor between digital output signal and the the number of electrons generated in the pixels of the CCD. It is one of the key parameters derived from the Photon Transfer
Curve, as it is the slope above the flux range at which the variance is dominated by shot noise, and below the PTC turnoff. Gain is expressed in e$^-$/ADU, and scales the digitized analog signals from the ASPICs to units of e$^{-1}$.

\begin{figure}[H]
\begin{centering}
\includegraphics[width=0.7\textwidth]{figures/baselineCharacterization/13591_E749_PTC_GAIN.png}
\caption{A comparison of Run 6 and Run 7 amplifier measurements in gain, separated by sensor type. For both sensor types, gain is very consistent.}
\end{centering}
\end{figure}

PTC gain measurements agree extremely closely across all sensors in the
focal plane.

\subsubsection{Brighter fatter coefficients}\label{brighter-fatter-a00-coefficient}


The brighter-fatter effect in CCDs refers to the phenomenon where brighter sources appear larger (or "fatter" than dimmer ones). This occurs due to electrostatic interactions within the pixel wells of the CCDs, when a pixel accumulates a high charge from incoming photons and creates an electric field that slightly repels incoming charge carriers into neighboring pixels. The brighter fatter effect can be modeled as the most dominant source of pixel-pixel correlations. Following the PTC model from \hyperref[Astier]{{[}Astier{]}} 
\citet{2019A&A...629A..36A}, $a_{00}$ describes the change of a pixel area due to its own charge content, or the relative strength of the brighter-fatter effect. Since same-charge carriers repel each other, the pixel area decreases as charge accumulates inside the pixel well, which implies $a_{00}$ \textless{} 0. Similarly $a_{10}$ describes the area change cause by a pixel to its nearest serial neighbor, and $a_{01}$ to the parallel nearest neighbor. Fig. \ref{fig:ratio_bf_coeff_6_7} compares the measurement of these coefficients carried out at SLAC and at the summit. We see that the variations are modest (and could be explained by noise) except for two rafts: R10 and R11. The Run 6 data used for this comparison was acquired with a high voltage of 45V applied to these two rafts, rather than the usual 50V. The sensitivity of our measurements of the brighter-fatter coefficients is sufficient to detect the change of electrostatic conditions due to this change of drift field in the sensors. In {\tt eo\_pipe}, an absolute value is taken of the $a_{00}$ parameter, so the tabulated quantities are positive.

\begin{figure}[H]
  \begin{centering}
   \minipage{0.32\textwidth}
  \includegraphics[width=\linewidth]{figures/baselineCharacterization/a00_ratios.png}
  \endminipage\hfill
   \minipage{0.32\textwidth}
  \includegraphics[width=\linewidth]{figures/baselineCharacterization/a01_ratios.png}
  \endminipage\hfill
   \minipage{0.32\textwidth}
  \includegraphics[width=\linewidth]{figures/baselineCharacterization/a10_ratios.png}
  \endminipage\hfill
  

\caption{Ratio of amplifier measurements of $a_{00}$, $a_{01}$ and $a_{10}$ coefficients at Run 6 and Run 7. They are very consistent, except for two rafts (R10 and R11) where the high voltage was changed between the two runs. the sense \label{fig:ratio_bf_coeff_6_7}}
  \end{centering}
  \end{figure}

The distribution of difference of $a_{00}$ measurements is displayed in Figure \ref{fig:ptc_a00_diff_hist}, and shows a tight agreement for both sensor types.

\begin{figure}[H]
\begin{centering}
\includegraphics[width=0.7\textwidth]{figures/baselineCharacterization/PTC_A00_13591_E749_diff.png}
\caption{A comparison of Run 6 and Run 7 amplifier differences in the $a_{00}$ coefficient, separated by sensor type. For both sensor types, the $a_{00}$ coefficient is very consistent. The two peaks on the left represent the two outlier rafts visible on Figure \ref{fig:ratio_bf_coeff_6_7}. The $a_{00}$ values are of the order of 2 to 3 $10^{-6}\ e^{-1}$. \label{fig:ptc_a00_diff_hist}}
\end{centering}
\end{figure}

However, the differences in the brighter-fatter $a_{00}$ coefficient between Run 6 and Run 7 show that the magnitude of $a_{00}$ decreased for most of the outliers, which implies an improvement in imaging for those pixels.

\subsubsection{Row-means variance}\label{row-means-var}

Row-means variance is a metric that measures the mean row-to-row variance of differences between a pair of flats. By computing variance of means of differenced rows at each flux level, we can measure any changes in gain row-by-row and also changes in correlated noise along with row.

\begin{figure}[H]
\begin{centering}
\includegraphics[width=0.7\textwidth]{figures/baselineCharacterization/13550_E1071_ROW_MEAN_VAR_SLOPE.png}
\caption{A comparison of Run 6 and Run 7 amplifier differences in row-mean-variance slope. For both sensor types, row-means-variance slope is weaker in Run 7. This is more pronounced for e2v sensors.}
\end{centering}
\end{figure}

Differences in row-means variance between runs are evident, and are distinctly different for different detector types. The difference between runs is more significant for ITL sensors, \textasciitilde9\% smaller on average in Run 7. For e2v sensors, the effect is \textasciitilde3\% smaller in Run 7. This indicates that either row-by-row correlated noise or row-by-row gain change is less in Run 7b. Since we did not change the sequencer file, the most natural explanation is the row-by-row correlated noise. But further investigation is needed.

\begin{figure}[H]
\begin{centering}
\includegraphics[width=0.7\textwidth]{figures/baselineCharacterization/ROW_MEAN_VAR_SLOPE_13550_E1071_diff.png}
\caption{A comparison of Run 6 and Run 7 amplifier differences in row-mean-variance slope, separated by sensor type. For both sensor types, row-means-variance slope is weaker in Run 7. This is more pronounced for e2v sensors.}
\end{centering}
\end{figure}

\subsubsection{Divisadero Tearing}

Divisadero tearing (or Rabbit ears) is manifested as signal variations near amplifier boundaries, connected features that are often jagged \cite{2020arXiv200209439J,2024SPIE13103E..0WU}. These variations are on the order of \textasciitilde1\% relative to the flat field signal. To quantify divisadero tearing in a given column, we measure the column signal, and compare it to the mean column signal from flat fields.

\begin{figure}[H]
\begin{centering}
\includegraphics[width=0.7\textwidth]{figures/baselineCharacterization/13550_E1071_DIVISADERO_TEARING_inset.png}
\caption{A comparison of Run 6 and Run 7 amplifier differences in Divisadero tearing, separated by sensor type. For both sensor types, Divisadero tearing is weaker in Run 7. The difference is more pronounced for e2v sensors, which have larger Divisadero tearing in general.}
\end{centering}
\end{figure}

Divisadero tearing is broadly consistent between Run 6 and Run 7, with both sensor types demonstrating lower Divisadero tearing in Run 7. Taking amplifier differences, e2v sensors show a weaker Divisadero signal in Run 7 by 0.1\%, while ITL sensors demonstrate a weaker Divisadero signal in Run 7 by 0.05\% (see Fig.~\ref{fig:divisidero_diff_baseline}).

\begin{figure}[H]
    \centering
    \includegraphics[width=0.7\linewidth]{figures/baselineCharacterization/DIVISIDERO_TEARING_13550_E1071_diff.png}
    \caption{A comparison of Run 6 and Run 7 amplifier differences in Divisadero tearing, separated by sensor type. For both sensor types, Divisadero tearing is weaker in Run 7.}
    \label{fig:divisidero_diff_baseline}
\end{figure}

\subsubsection{Dark defects}\label{dark-defects}

Dark defects are localized regions or individual pixels that produce abnormally low signal levels, even in the presence of light. Similar to bright pixels, dark pixels are also quantified in dark columns over 50 pixel contiguous regions. These
defects are caused by imperfections in the semiconductor
material, imperfections during the manufacturing process of a CCD. For our evaluation, we extract dark pixels from combined flats, with the threshold for a dark defect defined as a $-$20\% deficit from the average flat field flux measured in the image segment.

\begin{figure}[H]
\begin{centering}
\includegraphics[width=0.7\textwidth]{figures/baselineCharacterization/detector_85.jpg}
\caption{Illustration of masked border pixels (yellow) for detector 85 (R21\_S11). The average defect mask size is 4 pixels along the serial (x-pixel) direction, and 5 pixels along the parallel direction. Additional dark defects exist in the sensor, but are difficult to quantify due to the overwhelming contribution from the picture frame response.}
\label{fig:fig-edge-mask}
\end{centering}
\end{figure}

%%
The eo-pipe configuration for evaluating dark defects considers a border pixel region that is masked differently from the dark pixels. The default size for this edge is zero pixels. With a zero pixel border mask, the average dark defect count is 1800 per amplifier, with \geq 95\% of the contribution coming from the picture frame. The `picture-frame response' (also called `edge roll-off') near the edges of the sensors is due to a decrease in the pixel active area. It is difficult to extract useful information about the dark defects in the focal plane without excluding the picture frame. The effects of the picture frame signal on dark defect masking is shown in figure~\ref{fig:fig-edge-mask}.


\begin{figure}[H]
\begin{centering}
\includegraphics[width=0.7\textwidth]{figures/baselineCharacterization/13550_E1071_DARK_PIXELS_inset2.png}
\caption{Comparison of dark pixel counts in Run 7 (E1071) and Run 6 (13550), with separate plots for each raft.  Within each plot the color coding for all amplifier segments in a given CCD is the same.}
\label{fig:dark-pixels}
\end{centering}
\end{figure}

The default eo-pipe configuration has no border masking. The largest region permitted for the picture frame region is 9 pixels, determined by LCA-19363. Using a 9 pixel mask, the picture frame signal is removed, leaving true dark defects to be measured without contamination.

\begin{figure}
    \centering
    \includegraphics[width=0.5\linewidth]{figures/baselineCharacterization/darkDefects_comparison_initial.jpg}
    \caption{Comparison of dark pixel counts in Run 7 (E1071) and Run 6 (13550). Top: A histogram of amplifier measurements, separated by run number and sensor type. Bottom: A histogram of the amplifier dark pixel count differences, the difference is taken as the measurement from Run 6 and the measurement from Run 7.}
    \label{fig:initChar:DarkPixels:hists}
\end{figure}

In both instances, the contamination of dark pixels across the focal plane is \leq 10 pixels per amplifier on average. There is a measurable improvement in the dark pixel counts, decreasing by one pixel per amplifier between Run 6 and Run 7.

\subsection{Persistence}\label{initPersistenceChar}

Persistence is a feature of CCDs and how they are operated involving charge trapped in the
surface layer after high-flux exposures \citep{dmtn-276,2024SPIE13103E..0WU}.
Persistence is described in detail in Section~\ref{sec:persistence-optimization}.
Here we consider the measurements taken as
part of a persistence measurement task in the typical B protocol. For
measuring persistence, a high-flux acquisition is taken, followed by a
sequence of dark images. The persistence signal has been observed to
decrease in subsequent dark images as the trapped charge is released (see Figure~\ref{fig:persistence-decay} for an example). As a metric for persistence,
we evaluate the difference between the residual ADU in the first dark
image and the average of the residual ADU in the final dark images. This residual signal is found to be \textasciitilde10 ADU.

\begin{figure}[H]
\begin{centering}
\includegraphics[width=0.7\textwidth]{figures/baselineCharacterization/persistence_plot_LSSTCam_R22_S11_u_lsstccs_eo_persistence_E1110_w_2024_35_20240926T235141Z.png}
\caption{Persistence signal observed in R22\_S11 in Run 7 (E1110) as a function of time after the high-flux flat image.  The color coding indicates the individual amplifier segments.  The persistence metric is defined as the residual signal in the first dark image after the flat acquisition (red box).  Note that over time the signal does not decay entirely to zero.}
\label{fig:persistence-decay}
\end{centering}
\end{figure}

In the initial Run 7 measurements, we had not changed any operating
parameters of LSSTCam, so we would expect persistence to still be
present images at the same level as in Run 6.

\begin{figure}[H]
\begin{centering}
\includegraphics[width=0.7\textwidth]{figures/baselineCharacterization/13550_E1071_persist_inset.png}
\caption{Comparison of persistence metric between Run 7 (E1071) and Run 6 (13350), organized by raft.  The color coding indicates individual CCDs.  Several e2v CCDs have markedly greater persistence in Run 7.}
\label{fig:persistence-comp}
\end{centering}
\end{figure}

The persistence signal is generally consistent in e2v sensors between Run 6 and Run 7. Several e2v CCDs have greater persistence metric value in Run 7 (Fig.~\ref{fig:persistence-comp}). The outliers in
these persistence measurements are due to higher initial residual ADU, resulting in an excess of \textasciitilde5 ADU when comparing Run 6 with Run 7 (see Fig.~\ref{fig:persistence-decay-comp}).


\begin{figure}
\centering
\begin{subfigure}{0.5\textwidth}
  \centering
  \includegraphics[width=1.0\textwidth]{figures/baselineCharacterization/persistence_plot_LSSTCam_R11_S12_u_lsstccs_eo_persistence_13550_w_2023_41_20231117T001459Z.png}
\end{subfigure}%
\begin{subfigure}{0.5\textwidth}
  \centering
  \includegraphics[width=1\textwidth]{figures/baselineCharacterization/persistence_plot_LSSTCam_R11_S12_u_lsstccs_eo_persistence_E1071_w_2024_35_20240925T180602Z.png}
\end{subfigure}
\caption{Comparison of persistence profiles for R12\_S21 between (left) Run 6 (13550) and (right) Run 7 (E1071).  The decay time constants are similar but the initial persistence level is greater in Run 7.  The asymptotic levels are also slightly different.}
\label{fig:persistence-decay-comp}
\end{figure}

\subsection{Differences between Run 6 and Run 7}\label{differences-from-previous-runs}

All camera performance metrics from the summit show close agreement with SLAC IR2 tests. PTC/full-well metrics were consistent, and no significant bright cosmetic defects developed. Dark cosmetic defects are difficult to quantify due to the edge sensor effects, though the consistency in CTI measurements would indicate that dark defects did not change from previous runs. Dark current and Divisadero tearing show improved performance compared to Run 6, while the Persistence feature is still prominent in e2v sensors. 

\begin{table}[H]
\centering
\resizebox{\textwidth}{!}{ % Resize the table to the page width
\begin{tabular}{|l|l|ll|ll|}
\hline
\multirow{2}{*}{Parameter [unit]} & \multirow{2}{*}{Specification} & \multicolumn{2}{l|}{e2v}                   & \multicolumn{2}{l|}{ITL}                    \\ \cline{3-6} 
                                   &                                & \multicolumn{1}{l|}{Run 6}     & Run 7     & \multicolumn{1}{l|}{Run 6}     & Run 7      \\ \hline
Serial CTI {[}\%{]}                & Val               & \multicolumn{1}{l|}{3.7068E-5} & 1.1357E-5 & \multicolumn{1}{l|}{1.1488E-4} & 1.6478E-4  \\ \hline
Parallel CTI {[}\%{]}              & Val               & \multicolumn{1}{l|}{1.2162E-5} & 1.1534E-5 & \multicolumn{1}{l|}{3.4067E-7} & -4.7849E-6 \\ \hline
Dark current {[}e-/pix/s{]}        & Val               & \multicolumn{1}{l|}{5.5439E-2} & 2.4783E-2 & \multicolumn{1}{l|}{4.6424E-2} & 2.1217E-2  \\ \hline
Bright defects {[}count{]}         & Val               & \multicolumn{1}{l|}{0}          & 0         & \multicolumn{1}{l|}{0}          & 0          \\ \hline
Linearity turnoff {[}e-{]}         & Val               & \multicolumn{1}{l|}{156,339} & 167,797 & \multicolumn{1}{l|}{172,580} & 178,154  \\ \hline
PTC turnoff {[}e-{]}               & Val               & \multicolumn{1}{l|}{126,002}  & 132,963  & \multicolumn{1}{l|}{117,019}  & 128,595   \\ \hline
PTC Gain {[}e- / ADU{]}            & Val               & \multicolumn{1}{l|}{1.4785}    & 1.4811    & \multicolumn{1}{l|}{1.6717}    & 1.6760     \\ \hline
PTC $a_{00}$ [$\frac{1}{pix^2}$]   & Val               & \multicolumn{1}{l|}{3.0854E-6} & 3.0863E-6 & \multicolumn{1}{l|}{1.7119E-6} & 1.7031E-6  \\ \hline
BF x-correlation                   & Val               & \multicolumn{1}{l|}{0.5236}    & 0.5169    & \multicolumn{1}{l|}{0.7155}    & 0.7521     \\ \hline
BF y-correlation                   & Val               & \multicolumn{1}{l|}{0.1785}    & 0.1707    & \multicolumn{1}{l|}{0.2859}    & 0.2869     \\ \hline
Row-means variance                 & Val               & \multicolumn{1}{l|}{0.9927}    & 0.8836    & \multicolumn{1}{l|}{0.9924}    & 0.9466     \\ \hline
Dark defects {[}count{]}           & Val               & \multicolumn{1}{l|}{4} & 3 & \multicolumn{1}{l|}{9} & 8  \\ \hline
Divisadero tearing maximum {[}\%{]}& None               & \multicolumn{1}{l|}{0.32709}   & 0.27348   & \multicolumn{1}{l|}{0.75191}   & 0.62622    \\ \hline
Persistence {[}ADU{]}              & None               & \multicolumn{1}{l|}{5.6673}    & 5.6435    & \multicolumn{1}{l|}{0.48018}   & 0.42051    \\ \hline
\end{tabular}
}
\caption{Comparison of the median values of different parameters between Run 6 and Run 7, separated by detector type. For this comparison, only science detectors are considered.}
\end{table}


\section{Camera Optimization}\label{sec:camera-optimization}

\subsection{Persistence optimization}\label{sec:persistence-optimization}


Leftover signal (``persistence") in the first dark image acquired after intense illumination has
been observed \ref{persistence}.  Persistence has been observed
in an early prototype e2v sensor as early as 2014
\citep{2014SPIE.9154E..18D}. It was confirmed that the amplitude of
the persistence decreased as the parallel swing voltage was decreased.
This is consistent with the persistence being a Residual Surface Image (RSI) effect as described in
\citep{2001sccd.book.....J}, i.e., the excess charges are being trapped
at the surface layer. The level of persistence is about 10--20 ADU,
and the decay time constant is about 30\,s
\citep{dmtn-276}.

During the EO testing in 2021 (Run 13177, for example) Run 5, we also found the persistence made a
streak toward the readout direction from the place where a bright spot illumination occurred 
in a previous image. We call this ``trailing persistence".

As noted in Section (ref. tearing section above), depending on operating conditions e2v sensors have another major non-ideality, so-called ``tearing", which is
considered a consequence of the non-uniform distribution of holes. Over the past few years, our
primary focus in the optimization of the operating parameters was mitigation of the tearing, and we successfully eliminated the tearing by changing the
e2v voltages from unipolar (both parallel rails high and low
are positive) to bipolar (the parallel high is positive, and
the low is negative) following the Bipolar voltage formula
\footnote{https://github.com/lsst-camera-dh/mkconfigs/blob/master/newformula.py}.
However, the persistence issue
remained unchanged.

For the persistence issue, if this is a residual surface image, two
approaches could be taken as discussed in \citep{2024SPIE13103E..0WU}:  
either 1) establishing the pinning condition where the holes make a thin
layer at the front surface so that the excess charges recombine with
the holes, or 2) narrowing the parallel swing so that the accumulated
charges in the silicon do not get close to the surface state.

The pinning condition could be established by decreasing the parallel low
voltage to as low as -7\,V or lower. The transition voltage needs to be
empirically determined. However, Teledyne e2v advised that the measured
current flow increases as the parallel low voltage is decreased, which
increases the risk of damaging the sensor by inducing a
breakdown\footnote{We note that ITL operates at a parallel low voltage
  of $-$8.0\,V. We have observed the increased current flow. But we have
  software protection so that the current does not increase too much.}.
Also, the excess charges could be recombined by the thin layer of
the holes, which could affect linearity at high flux levels when
charges start to interact with the holes.

The parallel swing determines the full-well. Depending on whether the
accumulated charges spread over the columns or interact with the surface
layer, there are blooming full-well regimes and the surface full-well
regime. A full-well level between these two regimes is considered to be
optimal \citep{2001sccd.book.....J}, with no persistence and dynamic range as great as
possible. Because we observe the persistence effect, we likely operate the sensor in the
surface full-well condition and we need to decrease the parallel swing to
get the blooming full-well or the optimal full-well. The obvious downside
decreasing the full-well capacity.

The sensor control voltages are defined relative to each other. Changing, e.g., the parallel
swing also requires changes to all other voltages to
operate the sensor properly, e.g., to properly reset the amplifier.
The initial voltages were given in the original Bipolar formula
 but to decrease the parallel swing we had
to switch to the new persistence mitigation formula in order to satisfy the constraints\footnote{Persitence mitigation voltage: \url{https://github.com/lsst-camera-dh/e2v_voltages/blob/main/setup_e2v_v4.py}}

\citet{2024SPIE13103E..21S}, set up a single sensor test-stand at UC
Davis. They attempted multiple different approaches mentioned above and
reported the results\footnote{Davis report: \url{https://docs.google.com/document/d/1V4o9tzKBLnI1nlOlMFImPko8pDkD6qE7jzzk-duE-Qo/edit?tab=t.0\#heading=h.frkqtvvyydkr}}. The
summary is as follows:

\begin{itemize}
\tightlist
\item
  The new voltages following the persistence mitigation voltage rule produces reasonable bias, dark, flat images visually.
\item
  Narrowing the parallel swing eliminates the persistence.
\item
  Lowering the parallel low voltage did not work
  as we expected; going to a more negative voltage is probably needed.
\end{itemize}

Note that the e2v sensor in the UCD setup did not exhibit persistence.
This might be due to the characteristics of the sensor, or perhaps
the differences in the electronics (e.g., the long cable between CCD and REB). They need to move the both parallel high and low up to reproduce persistence as the similar as the main Camera.

\subsubsection{Persistence optimization}\label{persistence-optimization-1}
Based on this test result, we decided to test the new voltages with
the narrower parallel swing on the LSSTCam focal plane. Keeping the
parallel low voltage at -6\,V in order to operate the sensor safely (very
conservative limit), we changed the parallel swing voltage from 9.3\,V to
8.0\,V as well as all the other voltages using the new formula. We
overexposed the CCDs and took 20 darks afterward. Figure~\ref{fig:peristence-swing} compares the
mean and median of pixel-by-pixel differences between the first and the
last dark exposures, as a function of the parallel swing (We note that this is not the persistence metric defined in Sec. 2.5. but almost identical).
As the parallel swing is decreased, the residual signal decreases, reaching
roughly 10$\times$ less than the original level at 9.3\,V. Although we sampled
at 8.0 (E1363), 8.4 (E1430), 8.65 (E1411), 8.8 (E1424)and 9.3\,V (E1110), 8.0\,V appears to work the best and could
be lower with the penalty of decreasing the full-well capacity.

\begin{figure}
\begin{centering}
\includegraphics[width=0.9\textwidth]{figures/e2v_transient_dark_vs_dp.png}

\caption{The remaining charges measured in every amplifier but
aggregated by mean and median as a function of the parallel clock swing
are shown.}
\label{fig:peristence-swing}
\end{centering}
\end{figure}

Figure~\ref{fig:persistence-reduction} displays how the persistence is reduced by the
parallel swing decrease. The images were processed with the standard instrumental
signature removal and assembled in the full focal-plane view. The
dark exposure was taken right after a 400\,ke-equivalent flat exposure.
The figure shows the distinct pattern of elevated signal associated with
the e2v sensors, which fill the inner part of the focal plane.

The right-hand figure shows the same dark exposure but taken with the narrow
parallel swing voltage of 8.0\,V. The distinct pattern goes away. This
demonstrates the persistence in e2v sensors becomes the (low) level of the 
ITL sensors.


\begin{figure}
\centering
\begin{minipage}[b]{0.45\textwidth}
\centering
\includegraphics[width=\textwidth]{figures/E1110dp93.png}
\end{minipage}
\hfill
\begin{minipage}[b]{0.45\textwidth}
\centering
\includegraphics[width=\textwidth]{figures/E1880dp80.png}
\end{minipage}
\caption{Comparison of dark exposures under different parallel swings. (left) The first dark exposure after a 400\,ke$^-$ flat image under the parallel swing of 9.3\,V (Run E1110); (right) The first dark exposure after a 400\,ke$^-$ flat image under the parallel swing of 8.0\,V (Run E1880). The figure shows no distinct patterns from persistence in e2v sensors. Note that the guide sensors were not displayed here because they were being operated in guider mode. Also some of the residuals in ITL caused by defects disappeared here because of the employment of the new sequencer file (v30).}
\label{fig:persistence-reduction}
\end{figure}



\subsubsection{Impact on full-well}\label{impact-on-full-well}

Reduction of the full well is expected from narrowing the parallel swing
voltage. This subsection explores how much reduction in the PTC turnoff
is observed in the dense PTC runs. Two runs were acquired with identical
setting except for the CCD operating voltage (E1113 for 9.3\,V and E1335
for 8.0\,V). As the PTC turnoff is defined in ADU, it needs to be
multiplied by PTC\_GAIN to compare the turnoff values in electrons.
Figure~\ref{fig:ptc-turnoff} compares the PTC turnoffs in electrons and also shows their
fractional difference. The median of each peak are 133065e-, 102728e-, and the median reduction was 22\%.

\begin{figure}
\begin{centering}
\includegraphics[width=0.9\textwidth]{figures/PtcTurnoffRatio.png}
\end{centering}
\caption{Histograms of the PTC turnoff values scaled to electron units (left) and the ratios of
differences (right) between E1113 (9.3\,V) vs E1335 (8.0\,V). The median of
the reduction is 22\%.}
\label{fig:ptc-turnoff}
\end{figure}



\subsubsection{Impact on brighter-fatter effect}\label{impact-on-brighter-fatter-effect}

Reducing the parallel swing is expected to enhance the brighter-fatter
effect (BFE), possibly in an anisotropic way. The BFE can be
characterized via the evolution of the variance and covariances of
flat field exposures as a function of flux, i.e., via a PTC analysis. To evaluate the
impact of reducing the parallel voltage swing on e2v sensors, we
acquired two series of flat field exposures with the respective voltage
setups and extracted the ``area" coefficients $a_{ij}$
(Equation (1) in \citet{2023A&A...670A.118A}).
The area coefficients describe by how much a unit charge stored in
a pixel will alter the area of some other pixel (or itself). We find that
reducing the parallel swing from 9.3\,V to 8.0\,V typically increases the
area coefficients by 10\% (between 5 and 19\% depending on distance indexed by $i,j$),
and the increase is almost isotropic (i.e., very similar along serial and parallel
directions; see Fig.~\ref{fig:area-coeffs}). From these measurements, we anticipate that the increase of
star sizes with flux in LSST data will not become more anisotropic at 8.0\,V than it was at
9.3\,V, and hence this reduction of parallel swing does not 
risk increasing systematic uncertainty of the PSF ellipticity.

\begin{figure}
\begin{centering}
\includegraphics[width=0.8\textwidth]{figures/aScatterPlots8vs9-3.png}
\end{centering}
\caption{Scatter plots of area coefficients $a_{ij}$ (one entry per amplifier)
measured at 8.0\,V and 9.3\,V. The sub-figures correspond to separations in rows ($j$) and columns ($i$)
between the source of the area distortion and its victim, with the self
interaction coefficient $a_{00}$ at the bottom left. The first neighbors increase
respectively by 19\% in the parallel direction by 14\% in the serial
direction. So the BFE is slightly larger at 8.0\,V but not dramatically
more anisotropic: the ratio of parallel to serial nearest neighbor correlations increases only from 3.43 to 3.54 with the reduction of the parallel swing.}
\label{fig:area-coeffs}
\end{figure}


\subsection{Sequencer Optimization}\label{sequencer-optimization}
Several efforts were undertaken to optimize the sequencer configurations during Run 7. The following points summarize the key investigations:

\begin{itemize}
\item {\bf Clear}: Addressing the leftover charges at the image/serial register. The discussion is provided in Section \ref{sec:improved-clear}:
%  \begin{itemize}
%  \item
%    \textbf{No Pocket}
%  We introduced the v29\_Nop (No Pocket)
%  sequencer, which is an improved clear method using a serial register
%  configuration that reduces the formation of pockets at the Image/Serial
%  register interface. This clear method showed an approximately 2$\times$
%  improvement in the saturated image clear for e2v devices and completely
%  resolved the issue for ITL devices, except for R01\_S11,
%  where the No Pocket method performs approximately 2$\times$ worse than the default clear. 
%  For an unknown reason, this ITL CCD retains a significant amount of uncleared charges 
%  (hundreds of lines) after a saturated flat. This issue prevents the use of the No
%  Pocket configuration with ITL devices.
%
%  \item
%    \textbf{No Pocket with Serial Flush}
%  We introduced V29NopSf (No Pocket with Serial Flush), an enhanced version of the No Pocket Clear
%  sequencer, which includes a variable configuration of the serial register
%  during the clear process (mimicking a serial flush), to further prevent the
%  formation of pockets. This solution has been shown to completely prevent the presence of leftover charges after clearing a saturated image for e2v devices.
%  \end{itemize}
%  
\item {\bf Whether toggling the RG output during the parallel transfer for the e2v sensors is needed or not.}: Given the fact that there was some impact on making the bias structure in ITL better. The same question was raised for e2v sensors. The detail is described in Section \ref{noRGe2v}
\item {\bf Whether keeping the IDLE\_FLUSH running or not}: Addressing the worsening of the Divisadero tearing. The detail is described in Section \ref{section:disablingIDLEFLUSH}
\item {\bf Phase overlap during parallel transfer for e2v}: e2v sensors feature four parallel phases. To improve the uniformity of the full well across a sensor, overlapping two phases during each time slice of the parallel transfer was introduced.
\begin{itemize}
    \item Sequencer files that are based on the regular v29 but have changes in the parallel transfer by having a half overlap of what it was in the original (\_halfoverlapping.seq), a small amount of overlap compared to what it was in the original (\_overlap113.seq), ero overlapping at all (\_nonoverlapping.seq) are created.
    \item This overlap is known to cause trailing persistence, as reported in the Davis Report. We conducted several runs using both half overlapping (E1245) and non-overlapping (E1396) sequencers but we have not studied these because the trailing persistence is no longer a concern by optimizing the operating voltages to avoid charge trapping. 
\end{itemize}


\end{itemize}

\subsection{Improved Clear}\label{sec:improved-clear}

\subsubsection{Overview}\label{overview}

In this section, we describe the work done during Run 7 to improve
the image clear prior to collecting a new exposure.

The problem we wanted to address is the presence of residual charges in
the first rows read for an image taken just after the clear of a saturated
image. These ``hard to clear" charges are associated with highly
saturated flats or columns (or stars as observed in AuxTel or ComCam),
which leave signal in the first row of the subsequent exposure. The effect has a sensor-specific signature:

\begin{itemize}
\item
  In all ITL CCDs (except in R01\_S10 for which
  the effect is much more significant and which will be addressed later
  in this section): After a very bright exposure that saturates the overscan, the first row of the subseqeunt image has residual charges which are close to saturation. In most cases a small leftover signal in the second row is also present.
\item
  In e2v CCDs: the first row read after an exposure that follows an exposure with saturated
  overscan, has residual charges which are close to saturation, and a significant signal is visible
  in the subsequent 20--50 rows (see left-hand plot in Figure~\ref{fig:clear_e2v}).
  The effect is slightly amplifier dependent.
\end{itemize}

These leftover electrons are not associated with what we usually call residual image or persistence. They are suspected to be associated with pockets, induced by the electric field configuration in the sensor and the field associated with saturated pixels.
Investigation has revealed that only the first exposure taken after an image with saturated overscan is impacted. Our standard clear is not able to flush away those charges, while a standard readout of $\gtrsim$ 2000 rows does remove them.
There is a chance that a change of the electric field (e.g., a change in the clocking scheme defined in sequencer files) can remove the pockets, and free the charges, allowing them to be cleared.

The location of these uncleared electrons in the first row of the
CCDs indicates that the interface between the image area and the serial register
is the location of the pockets. For this reason we investigated
changes in the electric field configuration of the serial register during the
clear, to avoid generating pockets at the image-serial register interface.


To address this clear issue, we focused on updating the serial
register field as the rows are moved into it. The constraint is that
the changes introduced should not significantly increase the clear
execution time. It should be noted that in 2021 we tried a sequencer
called ``Deep Clear" \hyperref[sequencerV23_DC]{{[}sequencerV23\_DC{]}} as a first attempt to address the clear issue; it added one full row 
flush on top of the existing one at the end of the clear. This sequencer
did improve the clear, but did  not fully fix the clear issue (see
Table~\ref{tab:clears}).

{\tiny
\begin{longtable}{|p{0.2\linewidth}|p{0.12\linewidth}|p{0.2\linewidth}|p{0.2\linewidth}|p{0.2\linewidth}|}
\caption{Clear methods used so far. \label{tab:clears}} \\
\hline
\textbf{Clear Type} & \textbf{Duration (ms)} & \textbf{e2v after Saturated Flat} & \textbf{ITL after Saturated Flat} & \textbf{R01\_S10 ITL ``unique"} \\
\hline
\endfirsthead

%\hline
%\textbf{Clear Type} & \textbf{Clear duration} & \textbf{``E2V" after saturated Flat} & \textbf{``ITL`` after saturated Flat} & \textbf{R01 ITL ``unique"} \\
%\hline
\endhead

\hline
\endfoot

\hline
\endlastfoot

\textbf{Default Clear} 1~clear (seq. V29) & 65.5 & First row saturated signal up to row 50 & 1st row saturated signal up to 2nd row & First 500 rows saturated for 4 amp, 13 amp with signals \\
\textbf{Multi Clear} 3~clears (seq. V29) & 196.5 & No residual electrons & No residual electrons & First 150 rows saturated for 2 amp, 5 amp with signals \\
\textbf{Multi Clear} 5~clears (seq. V29) & 327.4 & No residual electrons & No residual electrons & First 100 rows saturated for 2 amp, 2 amp impacted \\
\textbf{Deep Clear} 1~clear (Seq. V23 DC) & 64.69 & 1st row saturated signal up to row <20 & Tiny signal left in the first row & not measured \\
\textbf{No Pocket (Nop)} 1~clear (seq. V29) & 65.8 & signal up to row 20 & No residual electrons & First 1000 rows saturated for 16 amp, 16 amp with signals \\
\textbf{No Pocket Serial Flush (NopSf)} 1~clear (seq. V29, V30) & 67.0 & No residual electrons & No residual electrons & first 750 rows saturated for 16 amp, 16 amp with signals \\
\end{longtable}
}

\subsubsection{New sequencers}\label{new-sequencers}
In Run 7, we considered two new configurations on top of the default clear. The changes are in the ParallelFlush function, which
moves the charges from the image area to the serial register:

\begin{itemize}
\tightlist
\item
  The default clear (V29): In the default clear, all serial clock voltages are
  kept up as the parallel clocks move charges from the image area to the
  serial register (\hyperref[sequencerV29]{{[}sequencerV29{]}}). The
  charges once on the serial register are expected to flow to the ground;
  the serial register clocks being all up, without pixel boundaries, and
  with its amplifier in clear state. At the end of the clear, a full
  flush of the serial register is done (\textasciitilde{} the serial
  clocks changes to read a single row).
\item
  The No-pocket Clear (Nop): a clear where the serial register has the
  same configuration (S1 \& S2 up, S3 low) when the parallel clock P1
  moves the charges to the serial register than in a standard image read. Still we kept all phases up for the rest of the time for a fast clear
  of the charges along the serial register
  (\hyperref[sequencerV29_Nop]{{[}sequencerV29\_Nop{]}}). The idea is
  that the S3 phase is not designed to be up when charges are transferred
  to the serial register, and is probably playing a major role in the creation of pockets.
\item
  The No-Pocket with Serial Flush Clear (NopSf): this sequencer is close
  to the Nop solution, except that during the transfer of one row to
  the serial register, the serial phases are also manipulated to transfer two
  pixels along the serial register. The changes in electric field at the
  image-serial register interface are then even more representative of
  what a standard read produces, and should further prevent the
  creation of pockets.
  (\hyperref[sequencerV29_NopSf]{{[}sequencerV29\_NopSf{]}}).
\end{itemize}
\begin{description}
\item[\label{sequencerV23_DC}{sequencerV23\_DC}]
\url{https://github.com/lsst-camera-dh/sequencer-files/blob/master/run5/FP_E2V_2s_ir2_v23_DC.seq}
\item[\label{sequencerV29}{sequencerV29}]
\url{https://github.com/lsst-camera-dh/sequencer-files/blob/master/run7/FP_E2V_2s_l3cp_v29.seq}
\item[\label{sequencerV29_Nop}{sequencerV29\_Nop}]
\url{https://github.com/lsst-camera-dh/sequencer-files/blob/master/run7/FP_E2V_2s_l3cp_v29_Nop.seq}
\item[\label{sequencerV29_NopSf}{sequencerV29\_NopSf}]
\url{https://github.com/lsst-camera-dh/sequencer-files/blob/master/run7/FP_E2V_2s_l3cp_v29_NopSf.seq}
\end{description}

\subsubsection{Results on standard e2v and ITL CCDs}\label{results-on-standard-e2v-and-itl-ccd}

\begin{figure}
\begin{centering}
\includegraphics[width=0.9\textwidth]{figures/plots_R12_S20_C15_E1880_bias_2024103000303.png}
\end{centering}
\caption{Impact of the three types of clear on a bias
taken after a saturated flat for an e2v sensor (R12\_S20).
The three panels on top show the interface region between the imaging section and the serial register. The aspect ratio is not 1 for presentation purpose; the bottom three plots are the averaged column profiles.}
\label{fig:clear_e2v}
\end{figure}

%\emph{Figure showing the impact of the various types of clear on a bias
%taken after a saturated flat for an e2v sensor.}

\begin{figure}
\begin{centering}
\includegraphics[width=0.9\textwidth]{figures/plots_R03_S11_C14_E1812_bias_2024102800352.png}
\end{centering}
\caption{Same as Figure \ref{fig:clear_e2v} but for an ITL sensor (R03\_S11).}
\label{fig:clear_ITL}
\end{figure}

%\emph{Figure showing the impact of the various types of clear on a bias
%taken after a saturated flat for an ITL sensor.}

In Figures~\ref{fig:clear_ITL} and \ref{fig:clear_e2v}, we present for three types of sequencer (from left to
right: V29, Nop, and NopSf), a zoom on the first rows of an ITL or e2v
amplifier (for ITL R03\_S11\_C14 and for e2v
R12\_S20\_C10 shown as a 2D row-columns
image (top plots) or as the mean signal per rows for the first row
read of an amplifier (bottom plots).

As seen in the left-hand panel of Figure~\ref{fig:clear_e2v}
for an e2v CCD, a bias taken just after a saturated flat will show a
residual signal in the first lines read when using the default clear
(left images, clear= V29): the first row has an almost saturated signal
($\sim$100 kADU here), and a significant signal is seen up
to row \textasciitilde50. In practice, depending on the 
amplifier, signal can be seen up to row 20--50. When using the Nop clear
(central plots), we can already see a strong reduction of the uncleared
charges in the first acquired bias after a saturated flat.  Still a small
residual signal is visible in the first $\sim$ 20 rows. The
NopSf clear (right plots) fully clears the saturated flat, and no
uncleared charges are observed in the following bias.

As seen in the left-hand panel of Figure~\ref{fig:clear_ITL}
for an ITL CCD, a bias taken just after a saturated flat will show a
residual signal in the first rows read when using the default clear
(left images, clear=v29): the first row has an almost saturated signal
($\sim$ 100 kADU here), and a significant signal is seen in
the following row. Both Nop clear (central plots) and NopSf clear
(right plots) fully clear the saturated flat, and no uncleared charges
are observed in the following bias.

\subsubsection{An exceptional case: ITL R01\_S10}\label{results-on-itl-r01s10}

\begin{figure}
\begin{centering}
\includegraphics[width=0.9\textwidth]{figures/Clear_R01_S10.png}
\end{centering}
\caption{Impact of the various types of clear on ITL
R01\_S10 after a saturated flat (bias after a saturated flat), from left
to right: 1 standard clear, 3 standard clears, 5 standard clears, 1 Nop
clear, 1 NopSf clear.}
\label{fig:clears_R01_S10}
\end{figure}

%\emph{Figure showing the impact of the various types of clear on ITL
%R01\_S10 after a saturated flat (bias after a saturated flat), from left
%to right: 1 standard clear, 3 standard clears, 5 standard clears, 1 Nop
%clear, 1 NopSf clear}

One ITL sensor, R01\_S10,
presents a specific behavior that is not understood:

\begin{itemize}
\tightlist
\item
  It has a quite low full well (2/3 of nominal).
\item
  The 3 CCDs of this REB (REB1) have a gain 20\% lower than all other ITL CCDs.
\item
  The images taken after a large saturation, as seen in Figure~\ref{fig:clears_R01_S10},
  show a large amount of uncleared charged (with the standard clear: 4
  amplifiers retain \textasciitilde500 rows of saturated signal!).
\end{itemize}

It appears that putting S3 low during the clear as done in Nop and NopSf,
is even worse than a standard clear. This is strange, as a full frame
read, which does this too, manages to clear a saturated image. We notice
that NopSf is \textasciitilde50\% better than Nop, but still worse than
the standard clear, in particular for the 12 amplifiers that are almost correct
with the standard clear.

At this time we do not have a correct way to clear this
sensor once the CCD heavily saturates by uniform illumination, but it is not
clear yet if a saturated star in this sensor, leaving signal in the
parallel overscan, will present the same clear issue.

\subsubsection{Conclusion on clears}\label{conclusion}
For e2v sensors, Run 7 finds the NopSf clear fully clear the leftover electrons at the image and the serial register interface.
The NopSf clear grants that the first 50 rows of e2v CCDs that had leftover electrons from the previous exposure are now free of such contamination. NopSf will be the default clear method.

For the ITL sensors, the improvement is still needed even if Nop or NopSf overcome the clear issue because there is the exception of R01\_S10 prevented the usage of those sequencers for ITL devices for Run 7. Note that
aside from R01\_S10 the numbers of lines potentially
``not cleared" in ITL devices after saturated images are small (2 first rows), and they
correspond to a CCD area that is difficult to use anyway (sensor edges with low
efficiency). So at this stage the default clear is still our default
for ITL, and further studies to overcome the problem with
R01\_S10 are forseen (e.g., investigate using a continuous
serial flush during exposure at low rate, 10$^6$ pixel flushes in 15\,s).
The original clear (serial phase 3 always), slightly extended in time to match the NopSf e2v clear execution time, will stay the default method.

\subsection{Toggling the RG Bit During Parallel Transfer for e2v sensors}\label{noRGe2v}
This investigation comes from an analogy drawn with the ITL sequencer file. Although the vendor recommended toggling the RG bit at the end of the parallel transfer, it was unclear whether this step was truly necessary. Given the improvements observed in ITL devices, applying this approach to e2v devices also became an area of interest.

\subsection{Disable IDLE FLUSH}\label{section:disablingIDLEFLUSH}

IDLE\_FLUSH is one of the main settings in the sequencer file that enables the sequencer output to run while in the IDLE state (the period between one exposure and the next). The specific implementation of IDLE\_FLUSH can be selected from various functions in the sequencer file. In Run 5, we chose the \texttt{ReadPixel} function, which reads out a pixel. This choice was initially made to mitigate the so-called yellow corner issue, a 2D structure of elevated signal near an amplifier corner observed in bias and dark exposures for certain amplifiers on e2v CCDs (see details in \citet{2024SPIE13103E..0WU}).

However, it was reported that running IDLE\_FLUSH exacerbates the Divisidero tearing issue. Divisidero tearing appears as a signal deficiency at amplifier boundaries in e2v sensors, accompanied by increased signal in adjacent columns. Additionally, using \texttt{ReadPixel} as the IDLE\_FLUSH function has the highest thermal impact because it continuously operates the Analog-to-Digital Converter at its maximum rate. This results in a significant difference in power consumption, more than 50\,W over all rafts, between the exposure state and the IDLE state. Consequently, the focal plane experiences a temperature variation of approximately 2 deg C between periods of image acquisition and idle periods (Figure~\ref{fig:IdleFlushEffect}).

\begin{figure}
\begin{centering}
\includegraphics[width=0.9\textwidth]{figures/REB_power_temp6_sept24_to_Oct23.png}
\end{centering}
\caption{Impact of enabling and disabling IDLE\_FLUSH on focal-plane temperature and power consumption.}\label{fig:IdleFlushEffect}
\end{figure}

This temperature variation in the focal plane can lead to changes in the REB temperature, potentially causing gain variations or instability in the bias. Based on these considerations, we decided to disable IDLE\_FLUSH. The impact of this change on bias stability is discussed in Sections~\ref{sec:bias-stability-2} and~\ref{sec:gain-stability-2}.

\begin{figure}
\begin{centering}
\includegraphics[width=0.9\textwidth]{figures/divisadero.png}
\end{centering}
\caption{Impact of disabling IDLE\_FLUSH on Divisadero tearing}\label{IdleFlushEffect:divisadero}
\end{figure}
Figure \ref{IdleFlushEffect:divisadero} shows the impact on the Divisadero tearing. The runs shown here are selected B protocol runs with different settings in the time order. There were few changes: (1) switching to narrower parallel swing voltage, (2) changing the number of clears before the exposure, (3) disabling IDLE\_FLUSH.  Some minor changes in each changes are also included such as changing the number of clears, or changing the sequencer file (the change from v29 to v30  is primarily incorporation of the change in the clear). The figure includes both ITL and e2v results. The two distinct distributions in earlier runs correspond to the differences between the two types of CCD (the higher one is e2v and the lower one is ITL). The greatest change can be seen when we switched to not running IDLE\_FLUSH at E1429, which brought the overall distribution down. The two distributions became indistinguishable, which indicates the majority of the Divisadero tearing for e2v is mitigated.

E3380 was the run taken after the recovery from the shutdown due to poor performance of the Pumped Coolant System. This fact confirms that the metric is consistent over power cycling of LSSTCam.

\subsection{Summary}\label{summary}

e2v sensors had persistence. We confirmed that narrowing the parallel swing voltage of the e2v CCD operation greatly reduced persistence. As penalties, we observed a full well reduction of 22\% and a \textasciitilde10\% increase of the
brighter-fatter effect, essentially in an isotropic way.

Sequencer files have undergone evolution for both ITL and e2v versions.
The final sequencer file from Run 6 was the
v26noRG version for ITL and the regular v26
for e2v. The suffix noRG indicates that the
RG bit is not toggled during parallel transfer. This modification
appears to enhance the stability of the bias structure for most ITL
amplifiers.

During Run 7, several changes were implemented, as described below:

\begin{itemize}
\tightlist
\item
  v27 incorporated guider functionalities, including ParallelFlushG and
  ReadGFrame. However, the noRG change was inadvertently included.
  Consequently, we abandoned this version and switched to v28.
\item
  v28 sequencer files merged v26noRG and
  v27. \url{https://rubinobs.atlassian.net/browse/LSSTCAM-5}
\item
  v29 introduced changes to speed up the guider.
  \url{https://rubinobs.atlassian.net/browse/LSSTCAM-34}
\item
  v30 primarily focused on e2v. We introduced a new approach to NopSf
  for e2v CCDs
  \url{https://github.com/lsst-camera-dh/sequencer-files/pull/17}. To
  align timing with the ITL version, a change was made.
  \url{https://github.com/lsst-camera-dh/sequencer-files/pull/18}
\end{itemize}

We also disabled IDLE\_FLUSH to improve the thermal situation and the Divisadero tearing.




\section{Characterization \& Camera
stability}\label{characterization-camera-stability}

\subsection{Illumination corrected flat}
% eventually we move these two section after the final characterization

\subsection{Glow search}
% eventually we move these two section after the final characterization


\subsection{Final characterization}
\subsubsection{Background}\label{final_background}

For a description of each quantity within this section and its acquisition process, refer to Section~\ref{reverification}. To compare initial and final camera metrics on Cerro Pachón, we used standard B protocols and dense red PTCs.

\begin{table}[H]
\centering
\caption{Reference runs for initial and final Run 7 comparisons}
\label{runTable}
\begin{tabular}{|l|l|l|}
\hline
\textbf{Run Type} & \textbf{Cerro Pachón Initial Run} & \textbf{Cerro Pachón Final Run} \\ \hline
B Protocol & E1071 & E1880 \\ \hline
PTC        & E749  & E1881 \\ \hline
\end{tabular}
\end{table}


For the final operating parameters of LSSTCam for Run 7, see Section \ref{run-7-final-operating-parameters}.

\subsubsection{Stability flat metrics}\label{stability-flat-metrics}

\paragraph{Serial CTI}\label{sec:finalChar-serial-cti}

Serial CTI is extracted from the B protocols, and show high consistency between initial and final operating parameters. 

\begin{figure}[H]
    \centering
    \includegraphics[width=0.7\linewidth]{figures/finalCharacterization/E1071_E1880_SCTI_EF_43_inset.png}
    \caption{Comparison of serial CTI measurements for initial and final Run 7 configurations.}
    \label{fig:finalChar-SCTI-5x5}
\end{figure}

The serial CTI for both sensor types is a noisy measurement, but serial CTI is not impacted by the changes to the camera operating configuration.

\begin{figure}[H]
    \centering
    \includegraphics[width=0.7\linewidth]{figures/finalCharacterization/SCTIComp.jpg}
    \caption{Histogram of serial CTI measurements for initial and final Run 7 configurations, separated by detector type.}
    \label{fig:finalChar-SCTI-hist}
\end{figure}


\paragraph{Parallel CTI}\label{sec:finalChar-parallel-cti}

Parallel CTI is extracted from the B protocols, and show high consistency between initial and final operating parameters. 

\begin{figure}[H]
    \centering
    \includegraphics[width=0.7\linewidth]{figures/finalCharacterization/E1071_E1880_PCTI_EF_43_inset.png}
    \caption{Comparison of parallel CTI measurements for initial and final Run 7 configurations.}
    \label{fig:finalChar-PCTI-5x5}
\end{figure}

Similar to serial CTI, the parallel CTI for both sensor types is a noisy measurement, but is not impacted by the changes to the camera operating configuration. 

\begin{figure}[H]
    \centering
    \includegraphics[width=0.7\linewidth]{figures/finalCharacterization/PCTIComp.jpg}
    \caption{Histogram of parallel CTI measurements for initial and final Run 7 configurations, separated by detector type.}
    \label{fig:finalChar-PCTI-hist}
\end{figure}

\subsubsection{Dark metrics}\label{final-dark-metrics}

\paragraph{Dark current}\label{dark-current}

 Dark current measurements were extracted from the B-protocol runs. Across the focal plane, dark current measurements are consistent with initial and final Run 7 runs. In a subset of rafts, a notable decrease in dark current is observed. These rafts are local to the autochanger light leak, which was mitigated as part of optimization efforts (see Sec.~\ref{successful-autochanger-light-leaks-masking}).

\begin{figure}[H]
    \centering
    \includegraphics[width=0.7\linewidth]{figures/finalCharacterization/E1071_E1880_DARK_CURRENT_MEDIAN.png}
    \caption{Comparison of dark current measurements for initial and final Run 7 measurements. A marked decrease in dark current is present in rafts local to the autochanger light leak (see Sec.~\ref{successful-autochanger-light-leaks-masking})}
    \label{fig:finalChar-DarkCurrent-5x5}
\end{figure}

The reduction in dark current in the subset of rafts is indicative of successful light leak mitigation, and lowers the dark current on local rafts to levels similar to the rest of the focal plane.

\paragraph{Bright defects}\label{final-bright-defects}

Bright defects are extracted using the B protocol runs, and show an extremely close agreement between runs. No significant bright defects developed as a result of the different voltage, sequencer, and idle flush conditions.

\begin{figure}[H]
    \centering
    \includegraphics[width=0.7\linewidth]{figures/finalCharacterization/E1071_E1880_BRIGHT_PIXELS.png}
    \caption{Comparison of bright pixel measurements for initial and final Run 7 measurements.)}
    \label{fig:finalChar-DarkCurrent-5x5}
\end{figure}

For additional discussion about defect stability, see Section~\ref{defect-stability}.

\subsubsection{Flat pair metrics}

\begin{figure}
    \centering
    \includegraphics[width=0.7\linewidth]{figures/finalCharacterization/run7PTCsToDate(2).jpg}
    \caption{Comparison of PTCs from initial and final Run 7 conditions, evaluated on a central detector and amplifier.}
    \label{fig:finalChar-PTCComparison}
\end{figure}

\paragraph{Linearity and PTC turnoff}\label{final-linearity-and-ptc-turnoff}

Both linearity and PTC turnoff were extracted from the PTC runs. Due to the lower parallel swing for e2v sensors, we anticipate a lower full-well capacity (see \cite{2001sccd.book.....J}). As described in Section~\ref{persistence-optimization-1}, we observe a decrease in full well capacity for e2v sensors. ITL sensors exhibit stable full-well measures, despite the changes to the v30 sequencer and disabling idle flush.

\begin{figure}
    \centering
    \includegraphics[width=0.7\linewidth]{figures/finalCharacterization/E749_E1881_LINEARITY_TURNOFF.png}
    \caption{Comparison of linearity turnoff measurements from the initial and final Run 7 measurements. The e2v sensors show a notable decrease in linearity turnoff, while ITL sensors stay the same. The values reported here are in ADU, and not gain corrected.}
    \label{fig:finalChar-Linearity-5x5}
\end{figure}

For e2v sensors we find the reduction in full-well to be significant, \textasciitilde20 - 25\% depending on the full-well metric used (see Fig.~\ref{fig:finalChar-fullWell}).

\begin{figure}
    \centering
    \includegraphics[width=0.7\linewidth]{figures/finalCharacterization/E749_E1881_PTC_TURNOFF.png}
    \caption{Comparison of PTC turnoff measurements from the initial and final Run 7 measurements. The e2v sensors show a notable decrease in PTC turnoff, while ITL sensors stay the same. The values reported here are in ADU, and not gain corrected.}
    \label{fig:finalChar-PTCTurnoff-5x5}
\end{figure}


Early in LSST's design, a 90,000 e- full-well capacity was established as a requirement for an 8 magnitude dynamic range across all bands. If linearity turnoff is the metric used to quantify full-well, all e2v amplifiers pass this system requirement. If PTC turnoff is the metric used to quantify full-well, 6.41\% e2v amplifiers do not pass this system requirement (120/1872 total amplifiers). 

\begin{figure}
    \centering
    \includegraphics[width=0.7\linewidth]{figures/finalCharacterization/fullWellComparisons.jpg}
    \caption{Comparisons of PTC turnoff and linearity turnoff for e2v science sensors. The runs analyzed here are the PTC runs noted in table \ref{runTable}. Note that both metrics have been gain corrected.}
    \label{fig:finalChar-fullWell}
\end{figure}


\paragraph{PTC Gain}\label{final-ptc-gain}

PTC Gain was extracted from the PTC runs. PTC gain is extremely comparable between initial and final Run 7 conditions, with a minor increase in gain observed in e2v sensors.

\begin{figure}
    \centering
    \includegraphics[width=0.7\linewidth]{figures/finalCharacterization/E749_E1881_PTC_GAIN.png}
    \caption{Comparison of PTC gain amplifier measurements, showing high consistency from initial and final Run 7 conditions.}
    \label{fig:finalChar-PTCGain-5x5}
\end{figure}

The magnitude of the gain increase for e2v sensors in final Run 7 conditions is \textasciitilde0.03 e-/ADU on average.

\begin{figure}
    \centering
    \includegraphics[width=0.7\linewidth]{figures/finalCharacterization/PTCGainComp(1).jpg}
    \caption{Comparison of PTC gains in e2v science sensors, with a moderate increase in the final run condition.}
    \label{fig:finalChar-PTCGain_HistComp}
\end{figure}

\paragraph{Brighter fatter $a_{00}$ coefficient}\label{final-brighter-fatter-a00-coefficient}

The relative strength of the brighter-fatter effect, quantified by $a_{00}$ following the model from \cite{2019A&A...629A..36A}, is modified in the final Run 7 operating conditions by the lower parallel swing for e2v sensors. We observe an extremely high consistency for ITL sensors.

\begin{figure}[H]
    \centering
    \includegraphics[width=0.7\linewidth]{figures/finalCharacterization/E749_E1881_PTC_A00.png}
    \caption{Comparison of amplifier measurements of the $a_{00}$ parameter for initial and final Run 7 conditions.}
    \label{fig:finalChar-PTC_A00_5x5}
\end{figure}

The change in the $a_{00}$ value for e2v sensors is illustrated in Figure~\ref{fig:finalChar-PTC_A00_E2VComp}, showing a \textasciitilde12\% increase in the strength of the brighter fatter effect for e2v sensors due to the lower parallel swing. For additional discussion on the brighter fatter coefficient, see Section \ref{impact-on-brighter-fatter-effect}.

\begin{figure}[H]
    \centering
    \includegraphics[width=0.7\linewidth]{figures/finalCharacterization/PTCA00Comp.jpg}
    \caption{Comparison of the $a_{00}$ values in e2v sensors, showing a notable increase from the final operating conditions.}
    \label{fig:finalChar-PTC_A00_E2VComp}
\end{figure}

\paragraph{Brighter-Fatter Correlation}\label{final-brighter-fatter-correlation}

The strength of the brighter fatter correlation was extracted from the PTC runs. In both instances, the correlation is extremely consistent across initial and final Run 7 operating conditions.

\begin{figure}[H]
    \centering
    \includegraphics[width=0.7\linewidth]{figures/finalCharacterization/E749_E1881_BF_YCORR.png}
    \caption{Comparison of amplifier measurements of the brighter-fatter y-correlation for initial and final Run 7 conditions.}
    \label{fig:finalChar-BFYCorr-5x5}
\end{figure}

Both correlations vary by \lesssim 2.2\% on average, decreasing in both instances (see table \ref{table:FinalChar-paramTable}). The measurement is noisy, with all rafts showing unbiased scatter around the correlation measurement on the raft level, evident in figures \ref{fig:finalChar-BFXCorr-5x5} and \ref{fig:finalChar-BFYCorr-5x5}.

\begin{figure}[H]
    \centering
    \includegraphics[width=0.7\linewidth]{figures/finalCharacterization/E749_E1881_BF_XCORR.png}
    \caption{Comparison of amplifier measurements of the brighter-fatter x-correlation for initial and final Run 7 conditions.}
    \label{fig:finalChar-BFXCorr-5x5}
\end{figure}


\paragraph{Row-means variance}\label{final-row-means-var}

Row means variance is extracted from the PTC runs, and shows an extremely tight correlation when comparing the initial and final operating conditions of Run 7. ITL sensors show an extremely tight agreement, while e2v sensors show a lower row-means variance by \textasciitilde1.8\% in the final operating conditions.

\begin{figure}[H]
    \centering
    \includegraphics[width=0.7\linewidth]{figures/finalCharacterization/E749_E1881_ROW_MEAN_VAR_SLOPE.png}
    \caption{Comparison of amplifier measurements of the row-means variance slope  for initial and final Run 7 conditions.}
    \label{fig:finalChar-RowMeanVarSlope-5x5}
\end{figure}

\paragraph{Divisadero Tearing}\label{final-divisadero-tearing}

Divisadero tearing measurements were extracted from the B protocols, and are significantly different for e2v sensors in the final operating condition. The change in Divisadero strength is driven by idle flush, which is described in detail in Section~\ref{section:disablingIDLEFLUSH}. The e2v sensors show a 60.7\% decrease in the original Divisadero signal under the final operating conditions. ITL sensors show a 0.2\% increase in the original Divisadero signal under the final operating conditions.

\begin{figure}[H]
    \centering
    \includegraphics[width=0.7\linewidth]{figures/finalCharacterization/E1071_E1880_DIVISADERO_TEARING.png}
    \caption{Comparison of amplifier measurements of Divisadero tearing for initial and final Run 7 conditions.}
    \label{fig:finalChar-Divisadero-5x5}
\end{figure}

In Figure \ref{fig:finalChar-Divisadero-5x5}, several e2v sensors do not follow the global trend of decreased Divisadero signal.  

\paragraph{Dark defects}\label{final-dark-defects}

Dark defects in LSSTCam were extracted using the B protocols, and are contaminated by the picture frame effect regardless of operating conditions (see Sec.~\ref{dark-defects} for additional discussion). When applying a 9 pixel mask to the edges of each sensor, the picture frame signal is removed, leaving true dark defects acquired by the analysis pipeline.

\begin{figure}[H]
    \centering
    \includegraphics[width=0.7\linewidth]{figures/finalCharacterization/E1880_E1071_DARK_PIXELS_inset.png}
    \caption{Comparison of amplifier measurements of the dark pixel for initial and final Run 7 conditions.}
    \label{fig:finalChar-DarkPix-5x5}
\end{figure}

Dark defects are consistent between initial and final Run 7 data. Dark defects are a minimal contribution to the focal-plane, with an average contribution of 3 pixels per e2v amp and 8 pixels per ITL amp. There is no global change in dark defect counts per amp, with measurements of the difference of 
dark pixel counts per detector centered on zero for both detector types (see Fig.~\ref{fig:finalChar:darkDefectsComparison}). 

\begin{figure}[H]
    \centering
    \includegraphics[width=0.9\linewidth]{figures/finalCharacterization/darkDefects_comparison_final.jpg}
    \caption{The amplifier measurements of dark pixel defects, with a 9 pixel mask applied to each sensor. Top: A histogram of the dark pixel measurements, with each count representing one amplifier. Histogram groups are separated by sensor type, and also by initial (E1071) and final (E1880) runs. Bottom: The difference in amplifier dark pixel measurements, separated by detector type. For both detector types, there is no significant evolution in the defect counts.}
    \label{fig:finalChar:darkDefectsComparison}
\end{figure}

\subsubsection{Persistence}\label{final-persistence}

The primary optimization target of Run 7 was to mitigate persistence, described in Section~\ref{persistence-optimization-1}. The major change in the final camera operating conditions to combat persistence is decreased parallel swing. This change is applied to the e2v sensors only, as they are the subset of sensors that exhibit \geq 1 ADU persistence when using the Run 7 initial operating parameters.

\begin{figure}[H]
    \centering
    \includegraphics[width=0.7\linewidth]{figures/finalCharacterization/E1071_E1880_persist_inset.png}
    \caption{Comparison of amplifier measurements of persistence for initial and final Run 7 conditions.}
    \label{fig:finalChar-Persist-5x5}
\end{figure}

The amplifier measurements of persistence using the metric described in Section~\ref{initPersistenceChar} show a significant decrease in persistence signal in e2v sensors due to the lower parallel swing, from 5.66 ADU \rightarrow 0.40 ADU on average when measured using the red LED (corresponding to the LSST r band filter).

\begin{figure}
    \centering
    \includegraphics[width=0.7\linewidth]{figures/PersistenceE2V.jpg}
    \caption{Demonstration of persistence mitigation in e2v sensors.}
    \label{fig:finalChar-persistenceAllLEDs}
\end{figure}

The B protocol uses a persistence dataset that uses the red LED, and flashes at 400k e- only (description of B protocol persistence dataset in Section~\ref{background}). Additional persistence datasets were acquired using other LEDs and other exposure levels with the CCOB wide beam projector. This was to verify that persistence was mitigated for the complete LSST photometric range. The runs used for this analysis are listed in table \ref{table:runs_persistence}. 

We find that $\geq$95\% of e2v sensors exhibit a persistence signal $\leq$ 0.55 ADU at all flux levels below full-well capacity. The CCOB requested flux varied by $\sim$ 10\% across the focal plane, and the maximum PTC turnoff for e2v amplifiers under the final operating conditions was 123,243 e- (see Fig.~\ref{fig:finalChar-fullWell}), within the flux levels probed in Fig.~\ref{fig:finalChar-persistenceAllLEDs}.

\subsubsection{Differences between Run 7 initial and Run 7 final measurements}\label{final-differences-from-previous-runs}

Comparing the initial and final Run 7 measurements, there are four metrics that are impacted by the optimization efforts described in Section~\ref{sec:camera-optimization}.
\begin{itemize}
  \item \textbf{Persistence:} We minimized persistence in e2v sensors, the main optimization target of Run 7, decreasing it from 5.66 ADU to 0.40 ADU on average with the red LED (LSST-r band), and maintaining sub-ADU levels across the entire LSST bandpass. Due to no change in ITL voltages and lack of an initial persistence feature, ITL sensors do not show a significant change in persistence, and remain at a sub-ADU level (0.48 ADU \rightarrow 0.32 ADU).
  \item \textbf{Full well capacity:} As a direct consequence of lower parallel swing in e2v sensors, the full-well capacity of e2v sensors decreased significantly with the final operating parameters. For linearity turnoff, e2v sensors decrease from 167,796 e- \rightarrow 136,302 e-. PTC turnoff measurements decrease from 132,963 e- \rightarrow 102,713 e-. ITL sensors do not show a significant change, and remain consistent between initial and final runs.
  \item \textbf{Brighter-fatter strength (PTC $a_{00}$):} The strength of the brighter fatter effect is also significantly impacted by the change in parallel swing for e2v sensors. The $a_{00}$ parameter increases from $3.08\times10^{-6} \rightarrow 3.49\times10^{-6}$ for e2v sensors, a 13\% increase. ITL sensors are not significantly impacted.% Consequence of changed dP
  \item \textbf{Divisadero:} The strength of Divisadero tearing is impacted by idle flush. For e2v sensors, we measure a reduction in maximum Divisadero signal from 0.62\% \rightarrow 0.25\%, a 60\% reduction in signal. ITL sensors did not exhibit a strong divisadero signal under the initial conditions, and therefore did not measure a reduction in maximum Divisadero signal (0.273\% \rightarrow 0.274\%). The initial strength of Divisadero tearing in ITL sensors is taken as a reference size, and is therefore not minimized by the change in idle flush.
\end{itemize}

\begin{table}[H]
\centering
\resizebox{\textwidth}{!}{ % Resize the table to the page width
\begin{tabular}{|l|l|ll|ll|}
\hline
\multirow{2}{*}{Parameter [unit]} & \multirow{2}{*}{Specification} & \multicolumn{2}{l|}{e2v}                   & \multicolumn{2}{l|}{ITL}                    \\ \cline{3-6} 
                                  &                                & \multicolumn{1}{l|}{R7 initial} & R7 final & \multicolumn{1}{l|}{R7 initial} & R7 final \\ \hline
Serial CTI {[}\%{]}               &                                & \multicolumn{1}{l|}{1.1357E-5}              &      7.3015E-6       & \multicolumn{1}{l|}{1.6478E-4}              &        1.5221E-4     \\ \hline
Parallel CTI {[}\%{]}             &                                & \multicolumn{1}{l|}{1.0555E-5}              &       1.1111E-5      & \multicolumn{1}{l|}{-4.7850E-6}              &       1.2103E-6      \\ \hline
Dark current {[}e-/pix/s{]}       &                                & \multicolumn{1}{l|}{0.024783}              &       0.023188      & \multicolumn{1}{l|}{0.021217}              &       0.020734      \\ \hline
Bright defects {[}count{]}        &                                & \multicolumn{1}{l|}{0}              &   0          & \multicolumn{1}{l|}{0}              &          0   \\ \hline
Linearity turnoff {[}e-{]}        &                                & \multicolumn{1}{l|}{167,796}              &        136,302     & \multicolumn{1}{l|}{178,153}              &      178,177       \\ \hline
PTC turnoff {[}e-{]}              &                                & \multicolumn{1}{l|}{132,963}              &      102,713       & \multicolumn{1}{l|}{128,594}              &        128,487     \\ \hline
PTC Gain {[}e- / ADU{]}           &                                & \multicolumn{1}{l|}{1.4812}              &      1.5107       & \multicolumn{1}{l|}{1.6761}              &       1.6752      \\ \hline
PTC $a_{00}$ [$\frac{1}{pix^2}$]  &                                & \multicolumn{1}{l|}{3.0863E-6}              &       3.4899E-6      & \multicolumn{1}{l|}{1.7031E-6}              &        1.7009E-6     \\ \hline
BF x-correlation                  &                                & \multicolumn{1}{l|}{0.51693}              &     0.51022        & \multicolumn{1}{l|}{0.75212}              &       0.73648      \\ \hline
BF y-correlation                  &                                & \multicolumn{1}{l|}{0.17077}              &       0.16740      & \multicolumn{1}{l|}{0.28695}              &        0.28439     \\ \hline
Row-means variance                &                                & \multicolumn{1}{l|}{0.88367}              &      0.86809       & \multicolumn{1}{l|}{0.94664}              &        0.94633     \\ \hline
Dark defects {[}count{]}          &                                & \multicolumn{1}{l|}{3}              &       3      & \multicolumn{1}{l|}{7}              &       7      \\ \hline
Divisadero tearing maximum {[}\%{]} &     None                     & \multicolumn{1}{l|}{0.62622}              &       0.24599      & \multicolumn{1}{l|}{0.27348}              &       0.27414      \\ \hline
Persistence {[}ADU{]}             &       None                         & \multicolumn{1}{l|}{5.6673}              &       0.40181      & \multicolumn{1}{l|}{0.48018}              &       0.32639      \\ \hline
\end{tabular}
}
\caption{Comparison of median parameter values on each amplifier between Run 7 initial and final measurements, separated by detector type.}
\label{table:FinalChar-paramTable}
\end{table}

All other metrics were not significantly impacted by the final operating conditions. For a complete list of the final operating conditions of LSSTCam as a result of Run 7 testing, see Section~\ref{run-7-final-operating-parameters}.

\subsection{List of Non-Functional Amplifiers}\label{deadamplifiers}

We classify amplifier sections as non-functional if they produce effectively no signal ({\it dead}) for incident light, or if the read noise level is above $18e^{-}$ ({\it hi-noise}).  Dead amplifiers are found with either read noise levels below $4e^{-}$ which indicates no signal is reaching the ADC, or from anomalous PTC gain values, outside the range 1.2--2.0 (or 0.8--1.8 for BOT data). 

A list of nonfunctional amplifiers on Science Rafts was produced from both single-raft testing as well as a selection of runs from the BOT data taking period. A summary of those amplifiers is shown in Table~\ref{tab:BOTbadamp}. As the table indicates, two amplifiers (R01\_S01\_C00 and R10\_S00\_C00) transitioned from dead to working during the course of the BOT testing, and another channel (R03\_S11\_C00) was dead in single-raft testing, then began working during BOT testing but was dead at the end of BOT testing. At the end of the BOT testing, only (R03\_S11\_C00 and R30\_S00\_C10) were classified as dead. Furthermore, of the six channels that were flagged as hi-noise during single raft or BOT testing, only one (R41\_S21\_C02) remained as hi-noise at the end of BOT testing.

\begin{table}[!ht]
    \tiny
    \centering
    \begin{tabular}{|l|l|l|l|l|l|l|l|l|l|l|l|l|}
    \hline
        Channel & Problem & \makecell{ Single Raft \\ testing } & \makecell{ Run 12433 \\ Oct 19} & \makecell{ Run 12610 \\ Oct 20} & \makecell{ Run 12795 \\ Nov 20} & \makecell{ Run 12845 \\ Jan 21} & \makecell{ Run 13016 \\ Nov 21} & \makecell{ Run 13101 \\ Nov 21} & \makecell{Run 13137 \\ Dec 21} \\ \hline
        R01\_S01\_C00  & Dead Amp & Dead & Dead & OK & OK & OK & OK & OK & OK \\ \hline
        R01\_S02\_C07  & Hi Noise & OK & 27e & 22e & 20e & 21e & 15e & 14e & 14e \\ \hline
        R01\_S11\_C00  & Hi Noise & OK & 24e & OK & OK & 12e & OK & OK & OK \\ \hline
        R03\_S11\_C00  & Dead Amp & OK & NA & OK & OK & Dead & Dead & Dead & Dead \\ \hline
        R10\_S00\_C00  & Dead Amp & Dead & NA & OK & OK & OK & OK & OK & OK \\ \hline
        R30\_S00\_C10  & Dead Amp & Dead & Dead & Dead & Dead & Dead & Dead & Dead & Dead \\ \hline
        R41\_S11\_C14  & Hi Noise & OK & NA & 36e & OK & OK & OK & OK & OK \\ \hline
        R41\_S21\_C02  & Hi Noise & OK & NA & OK & 108e & 96e & 85e & 110e & 115e \\ \hline
        R43\_S02\_C03  & Hi Noise & 18e & NA & 18e & 18e & 18e & 17e & 18e & 17e \\ \hline
        R43\_S20\_C14  & Hi Noise & OK & NA & OK & OK & 69e & 145e & OK & OK \\ \hline

    \end{tabular}
    \caption{Table of non-functioning Science Raft amplifiers. For hi-noise amplifiers the measured read noise is listed for levels above $12e^{-}$. \label{tab:BOTbadamp}}
\end{table}

Next, we list non-functional amplifiers detected in full Camera EO testing during Runs 6a, 6b and 7. We filter for potentially non-functional amplifiers with the same cuts as above a) read noise less than $4e^{-}$, b) read noise greater than $18e^{-}$, or c) PTC gain outside the range from $1.2 - 2.0 e^{-}/ADU$ in a number of B sequence runs (13391,13557,E1110,E1363,E1880,E2233,E3380) and PTC runs (13412,13591,E1113,E1364,E1881,E2237,E3577).  Note that one amplifier flagged in the BOT EO period (R10\_S00\_C00) is not flagged here, while there is one new amplifier (R03\_S01\_C05) which had never previously been flagged as non-functional.  To study these further, the PTC and linearity plots for these eight amplifiers are shown in PTC runs in Figure~\ref{fig:ptc-badamps} and Figure~\ref{fig:lin-badamps}. The eight amplifiers flagged by this selection are listed in Table~\ref{tab:badamps}, with comments.  Note that the amplifiers listed as {\it Dead} come in two flavors: no signal whatsoever (R30\_S00\_C10) or a tiny signal roughly linear with input but reduced by $~10^3$ (R01\_S01\_C00, R03\_S11\_C00).

\begin{figure}
    \centering
    \includegraphics[width=0.95\linewidth]{figures/ptc_badamps.png}
    \caption{PTC plots for amplifiers flagged as potentially non-functional, from Dense PTC runs}
    \label{fig:ptc-badamps}
\end{figure}

\begin{figure}
    \centering
    \includegraphics[width=0.95\linewidth]{figures/lin_badamps.png}
    \caption{Linearity plots for amplifiers flagged as potentially non-functional, from Dense PTC runs}
    \label{fig:lin-badamps}
\end{figure}

\begin{table}[!ht]
    \tiny
    \centering
    \begin{tabular}{|l|l|l|l|l|l|l|l|l|l|l|l|l|}
    \hline
        Channel & Summary & Comments  \\ \hline

R01\_S01\_C00  & SometimesDead & Usually OK, turns Dead in E3577, previously seen as Dead \\ \hline
R01\_S02\_C07  & OK & noise fluctuates sometimes over $18e^-$ consistent with previous behavior \\ \hline
R01\_S11\_C00  & OK & noise fluctuates sometimes over $18e^-$ consistent with previous behavior \\ \hline
R03\_S01\_C05  & SometimesHiNoise & previously OK,  High Noise in E3577 for first time, NEW bad amp \\ \hline
R03\_S11\_C00  & SometimesDead & Usually OK, turns Dead in E3577, previously seen as Dead \\ \hline
R30\_S00\_C10  & Dead & always Dead \\ \hline
R41\_S21\_C02  & HiNoise & always Hi Noise \\ \hline
R43\_S20\_C14  & HiNoise & always Hi Noise \\ \hline

    \end{tabular}
    \caption{Table of potentially non-functioning Science Raft amplifiers, from Runs 6a, 6b, and 7. Categories are OK, SometimesHiNoise, SometimesDead, HiNoise, Dead. \label{tab:gds_amps}}
\end{table}

Finally, we list non-functional Corner raft amplifiers, selected with the same filter.  There are three such amplifiers all in Guide sensors, that have been non-functional since single CCD testing.  These CCDs were selected for the Guiders due to the single non-functional amplifier, rather than use a fully working Science grade device. PTC and linearity curves for these channels, for three dense PTC runs, are shown in Figure~\ref{fig:ptclin-badamps-corner}, to classify these channels as either Dead or HiNoise. These three channels are listed in Table~\ref{tab:badamps-corner}

\begin{figure}
    \centering
    \includegraphics[width=0.45\linewidth]{figures/ptc_badamps_corner.png}
    \includegraphics[width=0.45\linewidth]{figures/lin_badamps_corner.png}
    \caption{PTC and Linearity plots for Corner Raft amplifiers flagged as potentially non-functional, from Dense PTC runs}
    \label{fig:ptclin-badamps-corner}
\end{figure}

\begin{table}[!ht]
    \tiny
    \centering
    \begin{tabular}{|l|l|l|l|l|l|l|l|l|l|l|l|l|}
    \hline
        Channel & Summary & Comments  \\ \hline

R04\_SG0\_C11  & Dead & always Dead \\ \hline
R40\_SG1\_C10  & HiNoise & always Hi Noise \\ \hline
R44\_SG0\_C02  & HiNoise & always Hi Noise\\ \hline

    \end{tabular}
    \caption{Table of non-functioning Corner Raft amplifiers, from Runs 6a, 6b, and 7. Categories are OK, SometimesHiNoise, SometimesDead, HiNoise, Dead. \label{tab:badamps-corner}}
\end{table}

\subsection{Full well measurements}\label{fullwell}

\subsection{Non-linearity studies}\label{nonlinearity}
\begin{figure}[h]
\begin{centering}
\includegraphics[width=0.9\textwidth]{figures/nonlin_plots.png}
\end{centering}
\caption{top: fitted nonlinearity spline (divided by the signal level) for the 16 channels of the central CCD. The main feature is due to the distortion introduced by the preamplifier. Left bottom: The fit residuals for channel 0 of the same CCD. Different colors refer to different LED current settings. Bottom right : r.m.s of high-flux fit residuals (the $c$ parameter of the fitted dispersion model) for all camera channels. Those are about $10^{-4}$ on average, and some are correlated within REBs, for an unknown reason. The quality of the obtained correction is well within goals.\label{fig:nonlin_model}}

\end{figure}

PTC runs are meant primarily to measure variance and co-variance curves. We collect pairs of flat images, obtained using the CCOB wide-beam described in \ref{electro-optical-setup}. To cover the entire dynamic range of the CCDs, we vary the length of the LED flash, the number of flashes, and the current of the LED. These data sets can be used to measure nonlinearity by comparing the CCD response to the integrated signal measured from a photodiode installed on a port of the integrating sphere that feeds a picoammeter. To avoid any shortcomings from picoammeter nonlinearity, we only compare photodiode signals of the same amplitude (illumination intensity) but different durations. We do not assume that integrated charges measured at different LED currents (and hence different photodiode currents) are on the same scale, although this turns out to be essentially true, as discussed later. 

For the nonlinearity study, we use the average signal measured on each CCD channel separately, using 2D overscan subtraction and masking outlier pixels. The photodiode signal is simply bias-subtracted and time-integrated. 

Technically, we model the nonlinearity using a spline function that we fit to the CCD/photodiode data pairs by minimizing:
\begin{equation}
Q = \sum_{ij} w_{ij}^2 \left( \frac{ S(\mu_{ij}) +\mu_{ij}  }{D_{ij} f_i} -1 \right)^2
\label{eq:nonlin0}
\end{equation}
where $Q_{ij}$ is the CCD signal measured in exposure $j$ at LED current $i$,
$D_{ij}$ is the corresponding photodiode signal, $f_i$ is the ``photodiode factor" for current $i$, $S$ is the spline nonlinearity correction, and $w_{ij}$ is some weight. We add two constraints: the average of the spline over the fitting range is zero $<S(\mu)=0>$, and $S(0) = 0$. If we choose equal fitting weights $w_{ij}=1$, the residuals exhibit a scatter that varies a lot with signal level, and hence forbid meaningful outlier detection. We model the fitting weights $w_{ij}$ using an expression determined empirically, $w_{ij} = 1/\sqrt{c^2+v^2/\mu_{ij}}$, and the two extra parameters, $c$ and $v$ are also fitted by turning the least-squares expression \ref{eq:nonlin0} into a maximum likelihood one:
\begin{equation}
Q = \sum_{ij} w_{ij}^2 \left( \frac{ S(\mu_{ij}) +\mu_{ij}  }{D_{ij} f_i} -1 \right)^2 - 2 \sum_{ij} \log w_{ij}
\label{eq:nonlin1}
\end{equation}
We fit the spline coefficients, the $f_i$ factors (there are typically 3 of them), and the weight parameters $c$ and $v$, for every image segment separately. We perform an iterative 5$\sigma$ outlier rejection which rejects on average $\sim $0.5 \% of the data points (this small rates validates the modeling of weights). Figure~\ref{fig:nonlin_model} displays some results of the fits. The quality of the measured non-linearity is sufficient for our needs.   

\subsection{Guider operation}\label{guider-operation}

This section describes guider operation.

\begin{itemize}
\tightlist
\item
  Initial guider operation
\item
  Power cycling the guiders to get to proper mode
\item
  Synchronization
\item
  Guider ROI characterization
\end{itemize}

%% jgt - here's the table with the configuratons and results (noise results omitted)
\begin{longtable}{|c|c|c|c|c|c|c|}
\caption{Summary of results for the different Guider configurations. \label{tab:gds_configs}} \\
\hline
\makecell{\textbf{ROI} \\ \textbf{Size}} & 
\makecell{\textbf{Integration} \\ \textbf{Time (ms)}} & 
\makecell{\textbf{Number} \\ \textbf{of} \\ \textbf{Sensors}} &
\makecell{\textbf{Number} \\ \textbf{of} \\ \textbf{Rafts}} &
\makecell{\textbf{ROI} \\ \textbf{Alignment}} &
\makecell{\textbf{Rate} \\ \textbf{(Hz)}} &
\makecell{\textbf{Noise} \\ \textbf{(ADU)}} \\
\hline
\hline
\endhead
\multicolumn{7}{|c|}{\textbf{Noise Study Configurations}} \\
\hline
50x50   & 50  & 1 & 1 & n/a       &  9.28 & 5.60 \\ % gds_noise_01.cfg
50x50   & 50  & 2 & 1 & aligned   &  9.27 & 4.57 \\ % gds_noise_02.cfg
50x50   & 50  & 2 & 1 & unaligned &  9.26 & 6.98 \\ % gds_noise_03.cfg
50x50   & 50  & 4 & 4 & aligned   &  9.26 & 4.70 \\ % gds_noise_04.cfg
50x50   & 50  & 4 & 4 & unaligned &  9.26 & 4.71 \\ % gds_noise_05.cfg
50x50   & 50  & 8 & 4 & aligned   &  9.23 & 4.61 \\ % gds_noise_06.cfg
50x50   & 50  & 8 & 4 & unaligned &  9.23 & 4.62 \\ % gds_noise_07.cfg
\hline
\multicolumn{7}{|c|}{\textbf{Nominal Configurations}} \\
\hline
50x50   & 50  & 8 & 4 & aligned   &  9.22 & 4.61 \\ % gds_nom_aligned.cfg
50x50   & 50  & 8 & 4 & unaligned &  9.23 & 6.50 \\ % gds_nom_unaligned.cfg
\hline
\multicolumn{7}{|c|}{\textbf{ROI Study Configurations}} \\
\hline
400x400 & 200 & 1 & 1 & n/a       &  1.67 & 4.03 \\ % gds_roi_01.cfg
400x400 &  50 & 1 & 1 & n/a       &  2.23 & 3.95 \\ % gds_roi_02.cfg
400x400 &   5 & 1 & 1 & n/a       &  2.48 & 3.91 \\ % gds_roi_03.cfg
10x10   &  50 & 1 & 1 & n/a       & 11.80 & 13.56 \\ % gds_roi_04.cfg
400x400 &  50 & 1 & 1 & SplitROI  &  2.23 & 105.30 \\ % gds_roi_05.cfg
\hline
\end{longtable}

\subsubsection{Noise Investigation (take 20 images for each, 15 seconds each)}
We measure the noise level of ROIs acquired under various configurations, shown in Table~\ref{tab:gds_configs}. We take 20 images in each configuration, where each image undergoes a 15-second exposure time. Due to different ROI sizes (and thus different read-out frequencies), the number of frames within each image varies.  The noise is calculated as the standard deviation of an entire ROI, and averaged over all frames from all of the 20 images. In cases where multiple sensors are running at the same time, the noise is also averaged over all sensors. The images were taken on 30 Nov. 2024 and 01 Dec. 2024. We note that all images taken on 30 Nov. 2024 suffer from an abnormal high-gain sensor state, where the counts level in each image is about one-tenth of expected values. This affects most of the rows in Table~\ref{tab:gds_configs} except the last two rows. The cause of such an abnormal state is under active investigation.

Guider configurations for Table~\ref{tab:gds_configs} are listed below:
\paragraph*{Start with a single GREB}
\begin{itemize}
    \item Acquire nominal ROI on a single sensor
    \begin{verbatim}
    gds_noise_01.cfg
    { "common": { "rows": 50,"cols": 50, "integrationTimeMillis": 50  }, 
      "R00SG0": { "segment": 3, "startRow":  975, "startCol": 254} }
    \end{verbatim}

    \item Acquire nominal aligned ROIs on two sensors on the same GREB
    \begin{verbatim}
    gds_noise_02.cfg
    { "common": { "rows": 50,"cols": 50, "integrationTimeMillis": 50 }, 
      "R00SG0": { "segment": 3, "startRow":  975, "startCol": 254},
      "R00SG1": { "segment": 3, "startRow":  975, "startCol": 254} }
    \end{verbatim}

    \item Acquire nominal misaligned ROIs on two sensors on the same GREB
    \begin{verbatim}
    gds_noise_03.cfg
    { "common": { "rows": 50,"cols": 50, "integrationTimeMillis": 50 }, 
      "R00SG0": { "segment": 3, "startRow":  975, "startCol": 254},
      "R00SG1": { "segment": 3, "startRow": 1075, "startCol": 254} }
    \end{verbatim}
\end{itemize}

\paragraph*{Four GREBs}
\begin{itemize}
    \item Acquire nominal aligned ROIs on single sensors on all GREBs
    \begin{verbatim}
    gds_noise_04.cfg
    { "common": { "rows": 50,"cols": 50, "integrationTimeMillis": 50 }, 
      "R00SG0": { "segment": 3, "startRow": 975, "startCol": 254},
      "R04SG0": { "segment": 3, "startRow": 975, "startCol": 254},
      "R40SG0": { "segment": 3, "startRow": 975, "startCol": 254},
      "R44SG0": { "segment": 3, "startRow": 975, "startCol": 254} }
    \end{verbatim}

    \item Acquire nominal misaligned ROIs on one sensor on all GREBs
    \begin{verbatim}
    gds_noise_05.cfg
    { "common": { "rows": 50,"cols": 50, "integrationTimeMillis": 50 }, 
      "R00SG0": { "segment": 3, "startRow":  775, "startCol": 254},
      "R04SG0": { "segment": 3, "startRow":  875, "startCol": 254},
      "R40SG0": { "segment": 3, "startRow":  975, "startCol": 254},
      "R44SG0": { "segment": 3, "startRow": 1075, "startCol": 254} }
    \end{verbatim}
\end{itemize}

% Add similar formatting for the rest of the sections

\subsubsection*{ROI reconstruction (take 20 images for each, 15 seconds each)}
\paragraph*{Unsplit ROI different exposures and sizes}
\begin{itemize}
    \item 200 ms
    \begin{verbatim}
    gds_roi_01.cfg
    { "common":{ "rows": 400,"cols": 400, "integrationTimeMillis": [200] }, 
      "R00SG0": { "segment": 2, "startRow":  800, "startCol": 54} }
    \end{verbatim}

    \item 50 ms
    \begin{verbatim}
    gds_roi_02.cfg
    { "common":{ "rows": 400,"cols": 400, "integrationTimeMillis": 50 }, 
      "R00SG0": { "segment": 2, "startRow":  800, "startCol": 54} }
    \end{verbatim}
\end{itemize}


\subsubsection*{Impact on science sensors}\label{sec:guiderimpactonscience}

\subsection{Defect stability}\label{defect-stability}

% Add explanation of how to measure defect stability

\subsubsection{Bright defects}

\subsubsection{Dark defects}

\subsection{Bias stability}\label{sec:bias-stability-2}

We have found bias instabilities, typically above the 1 ADU level, for a number of CCDs in the focal plane, both ITL and e2v. Two main kinds of instability are observed:

\begin{enumerate}
\tightlist
\item
  ITL bias jumps : large variations of the column-wise structure from
  exposure to exposure.
\item
  e2v yellow corners : a residual 2D shape of the bias even after
  2D-overscan correction. These residuals depend on the acquisition
  sequence and the exposure time, and the enhancement is greatest near the readout nodes (hence `yelllow corner').
\end{enumerate}

Both issues were observed and deeply studied in Run 6 EO data. The ITL
issue is believed to be phase shifts in clocks between Readout
Electronics Boards (REBs) because REBs rely on the frequency converted
from their natural frequency. We tried to mitigate the e2v issue by
optimizing the acquisition configuration in Run 7.

For the baseline acquisition configuration (see conclusion), three
relevant stability runs were recorded:

\begin{enumerate}
\tightlist
\item
  Run E2136: 15\,s darks with some very long delays throughout the run
\item
  Run E2236: 50 15\,s darks, 50 biases recorded with 30\,s delays between
  exposures
\item
  Run E2330: 15\,s and 30\,s darks with variable delays between exposures
\end{enumerate}

To analyze these runs for bias instability, the {\tt eo\_pipe} bias
stability task is used.  For the ISR part, a serial
%(\textquotesingle mean\phantomsection\label{per_row}{per\_row}\textquotesingle)
(`meanper\_row')
overscan correction and a bias subtraction (computed from the
corresponding B-protocol run) are applied. The final data product of the task is the
mean of the per-amplifier science image over the full set of exposures
of the run. Two typical examples from Run E2136 are shown in Figure~\ref{fig:bias-instability}. In the stable case, the variations are typically at the 0.1 ADU
level; in the unstable case, the variations range up to 4 ADUs.

\begin{figure}[h]
\centering
\begin{minipage}[b]{0.5\textwidth}
\centering
\includegraphics[width=\textwidth]{figures/E2136_R21_S21.png}
\end{minipage}
%\hfill
\begin{minipage}[b]{0.45\textwidth}
\centering
\includegraphics[width=\textwidth]{figures/E2136_R23_S22.png}
\end{minipage}
\caption{(left) Stable case for bias (R21\_S21); (right) Unstable case (R23\_S22)}
\label{fig:bias-instability}
\end{figure}


A comparison of the results for an unstable e2v CCD (R33\_S02) is shown in Figure~\ref{fig:r33_s02_bias} for the
three runs.

\begin{figure}[htbp]
\centering
\begin{minipage}[b]{0.45\textwidth}
    \centering
    \includegraphics[width=\textwidth]{figures/E2136_R33_S02.png}
    %\caption{Run E2136, R33\_S02}
\end{minipage}
\hspace{0.05\textwidth}
\begin{minipage}[b]{0.45\textwidth}
    \centering
    \includegraphics[width=\textwidth]{figures/E2236_R33_S02.png}
    %\caption{Run E2236, R33\_S02}
\end{minipage}

\vspace{0.05\textwidth}

\begin{minipage}[b]{0.45\textwidth}
    \centering
    \includegraphics[width=\textwidth]{figures/E2330_R33_S02.png}
    %\caption{Run E2330, R33\_S02}
\end{minipage}
\hspace{0.05\textwidth}
\hfill
\begin{minipage}[b]{0.45\textwidth}
    \centering
    % 空白セルとして保持
\end{minipage}
\caption{Bias level variations for R33\_S02, an unstable e2v CCD for three runs:  (upper left) E2136, (upper right) E2236, (lower left) E2330.  The segments CXX and CYY are most strongly variable in each run.  Note that the range of the time axes is different in each plot.}
\label{fig:r33_s02_bias}
\end{figure}


%\begin{figure}
%\begin{centering}
%\includegraphics[width=0.9\textwidth]{figures/E2136_R33_S02.png}
%\end{centering}
%\caption{Run E2136, R33 S02}
%\end{figure}
%
%\begin{figure}
%\begin{centering}
%\includegraphics[width=0.9\textwidth]{figures/E2236_R33_S02.png}
%\end{centering}
%\caption{Run E2236, R33 S02}
%\end{figure}
%
%\begin{figure}
%\begin{centering}
%\includegraphics[width=0.9\textwidth]{figures/E2330_R33_S02.png}
%\end{centering}
%\caption{Run E2330, R33 S02}
%\end{figure}

To highlight the 2D shape differences in e2v bias instability, a 2D-overscan correction
is applied. A few exposures illustrating the variations of the 2D shape
for the same unstable CCD R33\_S02 are shown in Figures~\ref{fig:r33_s02_1880}-\ref{fig:r33_s02_2136_delay}. The 2D shape of the image in
amplifier C01 is different in the 3 cases.

\begin{figure}[htbp]
\centering
\begin{minipage}{0.45\textwidth}
    \centering
    \includegraphics[width=\textwidth]{figures/E1880_bias_R33_S02.png}
    \caption{Bias exposure, run 1880, R33\_S02}
    \label{fig:r33_s02_1880}
\end{minipage}
\hfill
\begin{minipage}{0.45\textwidth}
    \centering
    \includegraphics[width=\textwidth]{figures/E2136_dark15_R33_S02.png}
    \caption{15-s dark exposure, run E2136 in 'stable' conditions, R33\_S02}
    \label{fig:r33_s02_2136}
\end{minipage}

\vspace{0.5cm}

\begin{minipage}{0.45\textwidth}
    \centering
    \includegraphics[width=\textwidth]{figures/E2136_dark15_delay_R33_S02.png}
    \caption{15\,s dark exposure, run E2136 after a 3\,min delay, R33\_S02}
    \label{fig:r33_s02_2136_delay}
\end{minipage}
\hfill
\begin{minipage}{0.45\textwidth}
    \centering
\end{minipage}
\end{figure}

%\begin{figure}
%\begin{centering}
%\includegraphics[width=0.9\textwidth]{figures/E1880_bias_R33_S02.png}
%\end{centering}
%\caption{Bias exposure, run 1880, R33 S02}
%\end{figure}
%
%\begin{figure}
%\begin{centering}
%\includegraphics[width=0.9\textwidth]{figures/E2136_dark15_R33_S02.png}
%\end{centering}
%\caption{15-s dark exposure, run E2136 in
%\textquotesingle stable\textquotesingle{} conditions, R33 S02}
%\end{figure}
%
%\begin{figure}
%\begin{centering}
%\includegraphics[width=0.9\textwidth]{figures/E2136_dark15_delay_R33_S02.png}
%\end{centering}
%\caption{15-s dark exposure, run E2136 after a 3-minute delay, R33 S02}
%\end{figure}

In order to quantify the number of unstable e2v amplifiers, a stability
metric \emph{d} is defined from the {\tt eo\_pipe}
stability task data products. More precisely, \emph{d} is defined, for a
given amplifier in a given run, as the difference between the 5th and
95th percentiles of the image mean over all the bias image acquisitions. The
distribution of \emph{d} for run E2136 is shown in Figure~\ref{fig:stability_dist}. Applying a
threshold at 0.3\,ADU, 51 amplifiers are identified as unstable (see the
corresponding mosaic in Fig.~\ref{fig:stability_mosaic}). This corresponds to \textasciitilde3\% of the e2v
amplifiers.

\begin{figure}[htbp]
\centering
\begin{minipage}{0.45\textwidth}
    \centering
    \includegraphics[width=\textwidth]{figures/E2136_distribution_d.png}
    \caption{Distribution of the stability metric for the e2v amplifiers in run E2136}
    \label{fig:stability_dist}
\end{minipage}
\hfill
\begin{minipage}{0.45\textwidth}
    \centering
    \includegraphics[width=\textwidth]{figures/E2136_mosaic_d.png}
    \caption{Mosaic of e2v amplifiers identified as unstable (white color) in run E2136}
    \label{fig:stability_mosaic}
\end{minipage}
\end{figure}

%\begin{figure}
%\begin{centering}
%\includegraphics[width=0.9\textwidth]{figures/E2136_distribution_d.png}
%\end{centering}
%\caption{Distribution of the stability metric for the e2v amplifiers in
%run E2136}
%\end{figure}
%
%\begin{figure}
%\begin{centering}
%\includegraphics[width=0.9\textwidth]{figures/E2136_mosaic_d.png}
%\end{centering}
%\caption{Mosaic of e2v amplifiers identified as instable (white color)
%in run E2136}
%\end{figure}

Further studies are required in order to converge on the best mitigation
strategy for the start of the LSST survey.

\subsection{Gain stability}\label{sec:gain-stability-2}
We use ``relative gain'' in this section to study gain stability over time. 
The relative gain is defined as the ratio of the signal observed in a CCD image segment divided by the integration of the photodiode current with respect to an arbitrary normalization.
With a fixed flat illumination, the variation of the relative gain over successive exposures can be utilized to investigate the gain stability. 
In the past run \citep{2024SPIE13103E..0WU}, we used the run that was obtained at the constant temperature, which reflects the real observing condition. We repeated this test during Run 7 and we acquired flat images at the two representative flux level with two distinct temperature conditions: either intentionally altered or maintained constant.
\begin{itemize}
    \item E1496 (dp80, 6 hours, constant temp, v29\_Nop, nm750, 10k e-)
    \item E1367 (dp80, 6 hours, temp swing, v29, nm750, 50k e-)
    \item E756 (dp80, 6 hours, gain stability @ 50k e-), unprocessed
    \item E1362 (dp80, 6 hours, 10k e-), unprocessed
\end{itemize}

Here we focus on E1496 and E1367 which have a difference in the temperature condition whether the temp was kept constant (E1496) or altered (E1367).

\begin{figure}[htbp]
\centering
\begin{minipage}{0.45\textwidth}
    \centering
    \includegraphics[width=\textwidth]{figures/gaintemp/E1496RelgainParametersTrending.png}
    \caption{relative gain changes with other parameters for one amplifier R01/S00/C11 in run E1496}
    \label{fig:relgainparamE1496}
\end{minipage}
\hfill
\begin{minipage}{0.45\textwidth}
    \centering
    \includegraphics[width=\textwidth]{figures/gaintemp/E1496RelgainDetail.png}
    \caption{Relative gain as a function of REB temp (TEMP6 and TEMP10), with color based on temperature swing run E1496}
    \label{fig:gaintempE1496}
\end{minipage}
\end{figure}
Figure \ref{fig:relgainparamE1496} shows the derived 
 relative gain change for one amplifier (R11/S00/C11) over time alogn with other representative parameters such as an aggregated REB temperature (TEMP6+TEMP10)/2, CCD temp, and led1temp (the temperatue measured at the LED board on the CCOB projector). The REB temp was determined as a good proxy for the relative gain change in the past run.
 As the intention of the acquisition condition, the REB temperature was almost maintained at the same level within 0.2 deg, with a slight decreasing slope probably due to the change in the thermal load and stabilization process of the entire thermal system.
 At the same time the gain slowly increase over time, while other CCD, LED temperatures are kept at the same.

 Figure \ref{fig:gaintempE1496} shows the relationship between the relative gain and the REB temperature, with color coding by the speed of temperature change, along with its fit. A reference line from the past result is overlaid with an arbitrary vertical offset. Clearly, the gain-temp relationship is steeper than the previous result. The distribution of the data points has a more complicated structure than the linear relationship, while there is no obvious change in either CCD nor LED temperatures. 



\begin{figure}[htbp]
\centering
\begin{minipage}{0.45\textwidth}
    \centering
    \includegraphics[width=\textwidth]{figures/gaintemp/E1367RelgainParametersTrending.png}
    \caption{relative gain changes with other parameters for one amplifier R01/S00/C11 in run E1367}
    \label{fig:relgainparamE1367}
\end{minipage}
\hfill
\begin{minipage}{0.45\textwidth}
    \centering
    \includegraphics[width=\textwidth]{figures/gaintemp/E1367RelgainDetail.png}
    \caption{Relative gain as a function of REB temp (TEMP6 and TEMP10), with color based on temperature swing run  in run E1367}
    \label{fig:gaintempE1367}
\end{minipage}
\end{figure}

Figures \ref{fig:relgainparamE1367} and \ref{fig:gaintempE1367} show the same set of figures but for the Run that has a temperature change in the cold plate by 2 degrees.
The temperature was kept the same in the beginning, but the set point was changed later, and then it was brought back to the original temperature. 
Clearly, Figure \ref{fig:gaintempE1367} shows not only temperature dependency but also hysteresis in the gain-temp relationship, which does not match the slope originally derived from the past run, although there are no obvious changes in the system other than the REB temperature.


The reason why the relationship becomes much more complicated is not clear. It is understandable that hysteresis was not observed in Run 6 because there was no intentional temperature change in the cold plate, which means the cold plate/REB temperature swing was minimal. However, looking at the result from E1496 where we took images at the same temperature, the relationship is much more complicated than what it looked like before. A number of possibilities can be considered to explain this: 1) there is a hidden variable that changes the gain other than the REB temp, 2) illumination from the LED is somehow changing overtime, which is not correlated with the LED temp, 3) air turbulence in the lens volume contributes to this, or 4) condensation on the lens might come into play. As the hysteresis is observed, the possibility 1 is definitely present but it cannot explain the gain temp change in the constant temperature. For the possibility 2, it is unlikely given the fact that Run 6 observed the complicated relationship. The option 3 could play some role since \citet{2024arXiv241113386B} discovered illumination changes due to the turbulence in the lens volume. However, it is not clear if any kind of long-term trending over 6 hours can be explained by this. For the possibility 4, we did visual check in a different period and we did not find anything obvious. 

The gain change issue can be split into 2 categories: global or local in an amplifier.  The global coherent change can be, in principle, correctable as it degenerates with the atmospheric transparency, which will be corrected by the calibration process. The local amp-by-amp change is a more serious issue in respect because the number of stars might not be sufficient for making the precise photometric calibration statistically. In order to study the local amp-by-amp gain change, Figures \ref{fig:relative-gain-E1496} for the constant temperature condition and \ref{fig:relative-gain-E1496} for the temperature swing condition show the differential gain changes with respect to the medianed relative gain for the entire focal plane. %: $\delta g_i-\delta g_0;~{\rm where}~\delta g_i(t_j)=\delta G_i (t_j)/\delta G_i(0)~{\rm and}~\delta G_i(t_j)=({\rm CCD~count}(t_j))/({\rm PD~measurement}(t_j))_i$~{\rm and}~{i~is~the~index~of~amplifier}. 

\begin{figure}
    \centering
    \includegraphics[width=1\linewidth]{figures/gaintemp/E1496gainoverall_global.png}
    \caption{Differential gain change with respect to the median of relative gain change for the whole focal plane, for E1496}
    \label{fig:relative-gain-E1496}
\end{figure}
The differential gain change with respect to global change for the constant temperature appears mostly stable within the level of $10^{-4}$. Some of the measurements deviated from zero because of the normalization of the first measurement. R11/S12, R12/S10, R12/S22, R24/S11, R34/S20 have one amplifier that have a higher relative gain than up to $5\times 10^{-4}$ but others behave stable. This could be contaminated by the yellow corner in e2v sensors but this could be mitigated by throwing away the first few exposures, which probably required for other aspect such as bias instability. Further investigation is needed. Another interesting behavior is seen in R11/S2x. There were a spike in three sensors at the same time. We have not figured out what happened at that time.

\begin{figure}
    \centering
    \includegraphics[width=1\linewidth]{figures/gaintemp/E1367gainoverall_global.png}
    \caption{Same as Figure \ref{fig:relative-gain-E1496} but for E1367}
    \label{fig:relative-gain-E1367}
\end{figure} 
The case for the temperature swing is complicated. Some of the amplifiers behave as well as the ones for the constant temperature case, but some of the amplifiers show correlation with the temp change in the Cold plate temp. This indicates that the relative gain change among amplifiers with respect to REB/Cold plate temperature exists. 
Note that E1367 has a 5 times higher flux than E1496, which reduces shot noises in the measurement. However, the conclusion still holds. 

To further study, we step back to the raw measurements. Figure \ref{fig:separateE1496} shows the constant temperature case. The change in the relative gain is a level of $2\times 10^{-4}$, which appears to be driven by the photodiode integration.
Figure \ref{fig:separateE1367} shows the temp swing case with a change of $5\times10^{-4}$, which appears to be dominated by a change in image counts. The changes in the PD integration are about the same in both plots. So from these facts, both of the gain change in the Camera due to the temperature change and some illumination difference of the CCOB projector play role here.
\begin{figure}[htbp]
\centering
\begin{minipage}{0.45\textwidth}
    \centering
    \includegraphics[width=\textwidth]{figures/gaintemp/E1496separate.png}
    \caption{Raw measurements of image count and photodiode integration, as well as the ratio of those -- the relative gain for E1496}
    \label{fig:separateE1496}
\end{minipage}
\hfill
\begin{minipage}{0.45\textwidth}
    \centering
    \includegraphics[width=\textwidth]{figures/gaintemp/E1367separate.png}
    \caption{Same as Figure \ref{fig:separateE1496} but for E1367}
    \label{fig:separateE1367}
\end{minipage}
\end{figure}

In summary, we find
\begin{itemize}
    \item The gain-REB temperature relationship is not as simple as Run 6.
    \item Global gain change could be due to the artifacts/setup, or potentially, the Camera could have a complicated behavior with respect to the REB temperature. No conclusive statement can be drawn.
    \item Local amp-by-amp gain change is minimal $10^{-4}$ over 6 hours if the REB or the focal plane temperature is maintained at the same.
    \item Further analysis is needed in understanding the gain change in the beginning of Run and some random spikes is needed.
\end{itemize}

\section{Sensor features}\label{sensor-features}

\subsection{Tree rings}\label{tree-rings}
Tree rings is circular variations in silicon doping concentration which can be observed in flat images. Both LSST The impact of the tree rings is assessed in \citep{2023PASP..135k5003E}. In this section we describe an attempt to measure tree rings for each sensor from the laboratory data taken in Run 7.
\subsubsection{Center of the Tree Ring}
From the past study, the center of tree rings is known to have 4 distinct positions with respect to each sensor. This is because four (4) CCD is cut from one wafer. 
So far we have been using the four average position for the center of the Tree ring, according to the pattern direction, because it was difficult to make measurement of the treering for all the sensors due to their low amplitude. However we have new data with 0\,V of back bias voltage, which increases the amplitude of the treering, allowing us to revisit the measurement of each individual center.

\begin{figure}
\begin{centering}
\includegraphics[width=0.9\textwidth]{figures/TR_centers.png}
\end{centering}
\caption{The center of the Tree Rings were measured for all 189 LSST sensors. Red point indicates the average center on each direction.}
\label{fig:tree_ring_center}
\end{figure}
Figure~\ref{fig:tree_ring_center} shows the positions of the Tree ring centers measured for the 189 sensors. All the measurements are concentrated around each averaged position, however, as now we have better individual measurements, we decided to use center of each sensor instead of the average value. 



\subsubsection{Radial study}
Radial study for Tree rings pattern has been done to see if the rings are perfectly circular in shape. 

%This is the image of transforming the 
Figure~\ref{fig:tree-ring-radial-transform} illustrates the transformation of a flat image into a radial profile plot as the y axis to be the distance from the center of the rings. 

\begin{figure}
\begin{centering}
\includegraphics[width=0.9\textwidth]{figures/TR_subtraction.png}
\end{centering}
\caption{Folding image on diagonal line from the center of the ring, and subtracting from each other.}
\label{fig:tree-ring-radial-transform}
\end{figure}

\begin{figure}
\begin{centering}
\includegraphics[width=0.9\textwidth]{figures/TR_radial.png}
\end{centering}
\caption{Radial study of the Tree Rings. Right: image subtracting left to right, right to left.}
\end{figure}

\subsubsection{Effect of diffuser}
We expect that with the diffuser installed, there will be less contribution from effects such as CMB and weather patterns discussed in \S~XX. Comparing R22\_S12 of Run 6 run 13379 (without diffuser) with Run 7 E937 (with diffuser), we verified the significant improvement from use of the diffuser.
\paragraph{Tree rings without diffuser}

\begin{figure}
\begin{centering}
\includegraphics[width=0.9\textwidth]{figures/TR_wo_diffuser.png}
\end{centering}
\caption{Tree ring without diffuser}
\end{figure}

\paragraph{Tree rings with diffuser}

\begin{figure}
\begin{centering}
\includegraphics[width=0.9\textwidth]{figures/TR_w_diffuser.png}
\end{centering}
\caption{Tree ring with diffuser}
\end{figure}


\subsubsection{Voltage dependency}
\begin{figure}[h]
\centering
\begin{minipage}[b]{0.45\textwidth}
\centering
\includegraphics[width=\textwidth]{figures/R01_S20_wBBV.png}
\end{minipage}
%\hfill
\begin{minipage}[b]{0.45\textwidth}
\centering
\includegraphics[width=\textwidth]{figures/R01_S20_woBBV.png}
\end{minipage}
\caption{Comparing Tree Rings pattern on the sensor R01 S20 with (left) and without (right) back bias voltage (50\,V), we can clearly see that back bias voltage reduces the impact of the tree ring effect.}
\end{figure}

\subsubsection{Wavelength dependency}
\begin{figure}[h]
\centering
\begin{minipage}[b]{0.45\textwidth}
\centering
\includegraphics[width=\textwidth]{figures/R22_S11_red.png}
\end{minipage}
%\hfill
\begin{minipage}[b]{0.45\textwidth}
\centering
\includegraphics[width=\textwidth]{figures/R22_S11_blue.png}
\end{minipage}
\caption{Comparing Tree Rings pattern on the sensor R01\_S20 for red (run E1050, left image) and blue (run E1052, right image) wavelength, without back bias voltage.}
\end{figure}

\begin{figure}
\centering
\includegraphics[width=0.7\textwidth]{figures/subtract_red_blue.png}
\caption{Subtracting blue image from red image}
\end{figure}

\subsection{ITL Dips}\label{itl-dips}

One of the phenomena that was studied in the later part of Run 7 was so-called 
`ITL dips'. These were discovered in LSST ComCam on-sky data as
bleed trails from bright stars that traversed the entire detector,
crossing the amplifier boundaries. These bleed trails are unique
though in that the core of the bleed trail is actually `dark'
compared to the wings of the trail, with a flux $\sim$2\% less relative to the rest of the
bleed trail.

We investigated whether ITL dips could also be observed in the CCDs of
LSSTCam. For this study, we used spots and rectangles projected onto the focal plane by the 4K
projector. The spots were approximately 30 pixels across
and were projected onto every amplifier segment of each detector. The rectangles were only
in the top right amplifier (C10). One consideration with this spot
projection was that the projector also provided background illumination. This led to the spots having a peak signal only 6 times greater
than the background and the rectangles having a peak signal 30 times greater
than the background.

We were unable to find any evidence of ITL
dips in the images. Below are the images themselves along with binned horizontal
cutouts of the the amplifier below the source. These show the background
pattern of the projector, but no 2\% dip.

While we were not able to find evidence of the ITL dip in Run 7 data, it
is still not clear whether the effect will be visible in LSSTCam on-sky data.
The photon rate of the in-lab data was roughly XXX per second for the 15\,s exposures. The stars that were seen in ComCam with the ITL dip
have a magnitude of XXX corresponding to a photon rate of XXX. This is
combined with a sky background of XXX as compared with the lab sensor
background of XXX.

\subsection{Vampire pixels}\label{vampire-pixels}

A category of sensor feature found on some ITL sensors, that has recently benefited from fresh attention, now has a new name. They are called {\it vampire pixels} because of their curious flat field response: a group of pixels with photo-response exceeding the flat-field mean, surrounded by a concentric distribution of pixels with photo-response below the same flat-field mean. The {\it vampire pixel} name sticks because the over-responsive pixels have apparently ``sucked'' signal from the rightful owners, a sort of {\it reverse brighter-fatter} effect excited simply with flat-field illumination. 

The sizes of these {\it vampire pixel} complexes can typically extend to tens of pixels in radius, which place strong constraints on their origin. Also, it turns out that all prominent {\it vampire pixel} complexes are also seen in their (ITL) phosphorescence response which is indicative of the backside surface layer ({\it cf.} \S\ref{phosphorescence}). This means that {\it vampire pixels} are likely to appear also in dark images, but only if trigger illumination is delivered a few tens of seconds prior to the dark acquisition ({\it cf.} vampire transients, Tab.~\ref{qualitative_assessment:itl_sensors}).

All known cases appear to round or with circular symmetry. There are plenty of cases with similar pixel complexes that lack the central group of pixels with {\it high-amplitude} photo-response excess, but that the photo-response excess is simply divided into a larger number of pixels. We suggest that the underlying origin of these is common with the easier-to-detect {\it bright pixels} ({\it cf.}\S\ref{first-observations}) but appear with different response properties simply because of mundane geometric details. Different detection algorithms may therefore be required finding those {\it vampire pixel} complexes that do {\it not} show central bright pixels as opposed those that {\it do}. Moving forward, we choose not to invent a new name for the former type, but call them both {\it vampire pixel} complexes.

\subsubsection{First observations}\label{first-observations}

Initial identification of these on ComCam may have been in a study that called them {\it bright pixels} by A. Roodman \href{https://confluence.slac.stanford.edu/download/attachments/209355949/Bad%20Pixels%20and%20Bright%20Spots.pdf?version=1&modificationDate=1724769154615&api=v2}{(20240827)} and quantified in more depth by A. Fert\'e in a ComCam defects study \href{https://rubin-obs.slack.com/files/U07MZAE6V3P/F080JU4CH8A/isr_science_unit_meeting__11_12_2024_-_vampire_pixels.pdf}{(20241112)}. First electrostatic simulations performed to help understand them were made by C.~Lage \href{https://confluence.slac.stanford.edu/download/attachments/209355949/Vampire_Pixel_Simulations_18Nov24.pdf?version=1&modificationDate=1731964502136&api=v2}{(20241119)} who inferred that a circumferential surface charge variations\footnote{We suggest that any such variations would necessarily require that the backside electrode ceases to act as an electrostatic equipotential as it does elsewhere on the sensor, with total surface charge density governed only by the normal component of the electric field strength within the silicon, responding to the HV Bias potential, and so on.} 
on the backside electrode could reproduce the sort of charge redistribution observed (while conserving photo-conversion charge) -- and so these may be effectively described as pixel boundary distortions throughout, mediated simply by lateral (non-axial) contributions to the drift field. Any such lateral fields would mean a localized loss in pixel fidelity, not limited to the sensor's thickness scale (10 pix) as are apparently in effect in other known pixel distortion mechanisms (brighter-fatter effect, tree rings, edge rolloff, tearing, pixel boundary distortions due to midline implant \& hot columns).

Since ComCam on-sky data has been available, more attention has been paid to these features and how they may impact source detection and photometric determination of field sources next to them. Luckily, {\it vampire pixels} are less common on average in the 88 ITL sensors of the Main Camera than they are in the 9 sensors of ComCam. 

\subsubsection{LSSTCam vampire pixel
features}\label{lsstcam-vampire-pixel-features}

One prominent example of such a feature is located in the Main Camera on R01\_S00\_C13-4. This feature is often overshadowed by the bright, dark current ``scratch'' in close proximity to it (when HV Bias is on). In Figure~\ref{fig:vamp:ffresp} and Table~\ref{tab:vamp:samples} we include this example along with two other prominent {\it vampire pixel} complexes (and others) located on ITL sensors in the Main Camera's focal plane to describe their individual properties. Inspection reveals a broad parameter space that describe these pixel complexes that can cause distortion in one way or another as soon as they are used to record cosmic sources: astrometric and shape transfer errors inferred from flat response, and background estimation or source confusion errors from the pixel complexes' phosphorescence properties!

%Because of the variation in how they appear, no doubt a flexible finding algorithm will need to be written.

%Vampire pixels were first observed in ComCam observations {[}need more
%info to properly give context{]} - Andy's study on Oct.
%8 - Agnes masking effort

%The vampire pixels have distinct features, both on the individual defect
%level, and across the focal plane

\begin{figure}[!htbp]
\centering
 \begin{subfigure}{0.3\textwidth}
     \includegraphics[width=\textwidth]{figures/vamp_desc/vamp_desc_ffresp_R01S00.png}
     \caption{R01\_S00\_C13-4.}
     \label{subfig:vamp_desc_R01_S00}
 \end{subfigure}
 \begin{subfigure}{0.3\textwidth}
 \vskip 0pt
     \includegraphics[width=\textwidth]{figures/vamp_desc/vamp_desc_ffresp_R03S10.png} 
     \caption{R03\_S10\_C15}
     \label{subfig:vamp_desc_R03_S10}
 \end{subfigure}  
 \begin{subfigure}{0.3\textwidth}
 \vskip 0pt
     \includegraphics[width=\textwidth]{figures/vamp_desc/vamp_desc_ffresp_R20S20.png} 
     \caption{R20\_S20\_C13}
     \label{subfig:vamp_desc_R20_S20}
 \end{subfigure}  
 \newline
\caption{Three prototypical {\it vampire pixel} complexes occurring in the Main Camera. Each of these have counterparts that appear in phosphorescence (transient dark). These are each described in Table XX. In R01\_S00, Two ``baby'' vampires appear in the 8:00 and 8:30 positions that each posess 300-400\% nominal flat response. The recorded phosphorescent counterparts for R03\_S10 and R20\_S20 are dependent on HV Bias (cf. Figs.~\ref{subfig:hvb_off_R03_S10} through \ref{subfig:hvb_on_R20_S20}) and should provide some constraints as to the pixel astrometric shifts at play near these {\it vampire pixel} complexes.}
\label{fig:vamp:ffresp}
\end{figure}



\begin{table}[h!]
\caption{Sample vampire pixel complex parameters. The last column indicates whether a concentric phosphorescence center is present.}
\label{tab:vamp:samples}
\centering
\begin{tabular}{llllll}
\toprule
{\it vampire pixel}& radius at 99\% & minimum& maximum& peak at & P?\\
complex & flat response & under-response & over-response & center?\\
\midrule
R01\_S00\_C13-4 & 200 pix & 1.4\%&1660\% & distributed & yes\\
                &         &      &       &~(10 pix offset)&\\
~baby1 (8:00)   & 6 pix   & 83\% & 375\% & yes & yes \\
~baby2 (8:30)   & 4 pix   & 84\% & 290\% & yes & yes \\
R03\_S10\_C15   & 36 pix  & 21\% & 1570\%& yes & yes \\
~baby3 (8:30)   & 4 pix?  & 97\% & 152\% & yes & no \\
~baby4 (10:30)  & 4 pix?  & 98\% & 120\% & yes & no \\
~baby5 (4:00)   & 4 pix?  & 89\% & NA    & no peak & yes \\
R20\_S20\_C13   & 52 pix  & 40\% & 829\% & no (ring-like) & yes \\
~baby6 (10:00)  & NA      & NA   & 108\% & yes & no \\
~baby7 (3:00)   & NA      & NA   & 119\% & yes & no \\
~baby8 (7:00)   & NA      & NA   & 207\% & yes & no \\
\bottomrule
\end{tabular}
\end{table}


The pixel complexes evaluated in Table~\ref{tab:vamp:samples} were chosen based on their proximity to the prominent {\it vampire pixel} appearing at the center of the corresponding image given in Figure~\ref{fig:vamp:ffresp}, and include (on average) 3 other, nearby complexes that may be more representative of these artifacts found on ITL sensors in the Main Camera focal-plane. In the Table they are indicated by their name (``babyX'') followed by the clocking angle where they can be identified relative to the prominent pixel complex located at the center. From this list, it appears that the {\it vampire pixel} complexes may be reliably identified by applying a OR combination of thresholds: under-response less than 90\%, or an over-response exceeding 120\% and some consideration of the presence of phosphorescence. The phosphorescence, more than anything, may help to distinguish the dark pixel complexes {\it without} central bright pixels from dust spots (which presumably would not preserve flux). 

There are a handful of dust spots seen in these images that were not included in~\ref{tab:vamp:samples}. They would presumably be detected as dark pixels provided the lower threshold is raised to levels that would be sensitive to their detection.


%\paragraph{Individual vampire
%features}\label{individual-vampire-features}
%
%\begin{itemize}
%\tightlist
%\item
%  General size
%\item
%  Radial kernel
%\item
%  uniformity
%\end{itemize}
%

An imperfect listing of ITL sensors in the Main Camera focal-plane showing such {\it vampire pixel} complexes, is given in a table in a later section (\S\ref{phosphorescence}, Tab.~\ref{qualitative_assessment:itl_sensors}). This list shows that 83 of the 88 Main Camera ITL sensors contain finite numbers of {\it vampire pixel} complexes, with typical numbers less than 30, revealed by spot-like phosphorescence counts. There would be significant overlaps with pixel complex lists generated using only stacks of flat-field response, which would also be sensitive to dust spots (in {\it dark pixels}), which are likely to have greater numbers (and may be partially salvageable depending on host surface and dust opacity). Meanwhile, {\it bright pixels} will also pinpoint the larger particulates that reflect light while also casting shadows (we've seen such cases, but their discussion is outside scope of this section). 

Based on this listing, there are a total of 17 (of 88) ITL sensors in the Main Camera that appear to host more than 30 {\it vampire pixel} complexes. These identifications may be used to study in depth more fully the science impact of their transverse electric fields as well as their phosphorescent properties (when recording is preceded directly by illumination by a star in the previous image).

As a \href{https://rubin-obs.slack.com/archives/C07QJMQAP6E/p1731348605966989?thread_ts=1730921120.364949&cid=C07QJMQAP6E}{proof of concept}, a task was added to {\tt eo\_pipe} to search for bright defect pixels in combined flats. Figure~\ref{fig:eopipe_brightdefects_task_result} displays the resulting distribution, which efficiently identifies or picks up the ITL set of sensors. Without looking more closely at the specific regions flagged, it would not be possible to separate out the {\it vampire pixel} complexes from {\it reflecting particulates}. Conversely, repeating this task to identify dark pixels with a threshold of 0.90 (90\% of flat level), we expect to see a combination of (flux conserving) {\it vampire pixel} complexes and garden variety (flux attenuating) dust to appear. (Currently, we do not know whether dust particulates prefer to stick to e2v sensors or to their ITL counterparts.)

\begin{figure}
    \centering
    \includegraphics[width=0.75\linewidth]{figures/vamp_desc/vampire_defects_fp_plot_LSSTCam_u_jchiang_eo_vampire_defects_E1880_w_2024_35_20241111T173034Z.png}
    \caption{Results of the {\tt eo\_pipe} task written to search for bright defects in combined flats. Using a threshold of 1.2 (120\% of flat level), the result highlights the 8 RTMs that operate ITL sensors (as well as those CRTMs with sensors operating in science mode).}
    \label{fig:eopipe_brightdefects_task_result}
\end{figure}

%\paragraph{Vampire features across the focal
%plane}\label{vampire-features-across-the-focal-plane}
%
%\begin{itemize}
%\tightlist
%\item
%  sensor type
%\item
%  static or dynamic
%\item
%  higher concentrations? Particularly bad sensors?
%\end{itemize}
%
%\subsubsection{Current masking
%conditions}\label{current-masking-conditions}
%
%\begin{itemize}
%\tightlist
%\item
%  Bright pixels
%\item
%  Dark pixels
%\item
%  Jim's task
%\end{itemize}

%\subsubsection{Analysis of flats}\label{analysis-of-flats}
%
%\begin{itemize}
%\tightlist
%\item
%  LED effect
%\item
%  Intensity effect
%\end{itemize}
%
%\subsubsection{Analysis of darks}\label{analysis-of-darks}
%
%\begin{itemize}
%\tightlist
%\item
%  Previous LED effect
%\item
%  Intensity of LED effect
%\item
%  dark cadence and exposure times
%\end{itemize}

%\subsubsection{Current models of
%vampires}\label{current-models-of-vampires}
%
%\begin{itemize}
%\tightlist
%\item
%  Tony \& Craig model
%\item
%  Others?
%\end{itemize}



\subsection{Phosphorescence}\label{phosphorescence}

The Run 7 persistence optimization process (cf. \S\ref{persistence-optimization-1}) used a short EO image acquisition sequence and analysis script, which rapidly provided persistence performance metrics as feedback for each configuration tested. Thus, as soon as the e2v sensors were shown to be nearly free of {\it their} undesirable effects by reducing their clock swing voltages from 9.3\,V down to 8.0\,V, a similar persistence (or memory effect) was immediately noticed, affecting a subset of the ITL sensors. This discovery gained immediate interest for at least two reasons: (1) that it had not been detected in prior EO campaigns, and (2) that the new memory effect on certain ITL sensors was morphologically distinct from what had just been cured on the e2vs. 

The ITL sensors with the largest memory effect were evaluated, and the following observations were made:
\begin{itemize}
    \item[1.] The morphology of the expressed memory effect in the first dark image acquired after the trigger (the saturation flat) was reminiscent of the {\it ``coffee stains''} seen on the same sensors in flat field response, but with the opposite polarity. The ``coffee stains'' are commonly assumed to be associated with minor, localized variations in the sensors' antireflective coatings or perhaps a very thin, dead layer associated with the backside surface: they tend to be larger in amplitude when shorter wavelengths are used to expose the sensors with flat field illumination.
    \item[2.] The attenuation timescale of the memory effect is curiously comparable to the timescales that were seen in the persistence suffered by the e2v sensors (which are believed due to exposure of surface states by the collected conversions, on the semiconductor-insulator interface on the front side): exponential time constants of between 20 and 40\,s, which unfortunately are in turn very close to the nominal exposure cadence for the LSST survey.
    \item[3.] The similarity in memory effect time constants (de-trapping charges from surface states near the channel on the front side -- the e2v case -- vs. either de-trapping of charges located near the backside window surface or relaxation by photon emission by some excited states there -- the ITL case) can be thought to favor the electron de-trapping mechanism, just from the other surface. Otherwise, the nearly matched time constants would have to be seen as an improbable coincidence.
    \item[4.] A list of 12 ITL sensor serial numbers corresponding to those showing the memory effect was communicated to Mike Lesser at ITL. The list of parts shared certain properties according to his notes, and led him to develop a placeholder theory that would partially explain the mechanism. If true, it could explain what might be responsible for both the coffee stains and the memory effect with similar spatial distribution. He wrote that he tried, but was unsuccessful in diagnosing, using optical characterization tools (e.g., ellipsometer), any changes in optical constants on the affected regions of the ``stained'' sensors. The origin of the ``stains'', according to this theory, is as a consequence of there being ``raised spots'' on the sensors' backside surfaces that survive the final silicon acid etch. The raised silicon areas could potentially be trapping the resist used during the cleaning process that directly follows the etching step. Lesser wrote that the resist is wax-based and {\it does} fluoresce. If the theory is correct, he suggests that the medium would definitely be located {\it under} the AR coating and related neither to the coating nor the oxidation processes.
    \item[5.] Discussions among the Rubin team led to the following distinction of terminology that served to name the ITL memory effect in question. The main difference between ``fluorescence'' and ``phosphorescence'' is in that the former is considered prompt re-emission and the later could be re-emission following a finite characteristic time constant. Characteristic time constants are in the nanosecond scale for fluorescence, while for phosphorescence it would be in the milliseconds to seconds range. For the purpose of this discussion, we adopt the word ``phosphorescence'' to refer to the memory effect present in some ITL sensors.
    \item[6.] Lesser mentioned that the wax-based resist fluoresces (that would be the prompt mechanism with very short relaxation time). If there is any such residual material between the coating and the passivated silicon, it would be natural to expect a halo that would accompany any sharp (PSF-scale features) illumination that passes through these ``stains'' on the sensor surface: a scatter term with low integrated amplitude, whose scale should depend upon the re-emission wavelength. This has not yet been seen in lab data but may appear once the Camera goes on-sky.
    
% I highly doubt that they are the resist itself but more likely etch patterns in the silicon itself which could have an effect on scattering and perhaps surface charge. But they could be initially caused by resist issues in during etching process.
% It’s rather obvious of course, but I checked the notes and device 231 was noted as “stained” during processing (after etching).
% In the past we did SEM work as well and could never get an answer on composition, thickness, etc.
\end{itemize}

\subsubsection{Measurement techniques for detecting and quantifying phosphorescence}\label{phos-measurement}
We mentioned above that certain phosphorescent morphologies strongly resemble the ``coffee stains'' seen on the same (ITL) sensors. It should be noted that measurement of the {\it shadow} caused by excess absorption (usually a couple percent) is a great deal simpler than collecting any deferred charge with adequate sensitivity and confidence. This section describes the methods used to identify the transient term we consider phosphorescence in the ITL sensors, and list the regions where it was detected. Following that, we describe in some detail the kinematics of its expression (cherry-picking specific easy-to-measure cases), together with the wavelength- and its excitation flux-level dependence.

We parasitically used a series of B-protocol and BOT-persistence EO testing runs that were executed for the purpose of tuning the operation of e2v sensors. The reason for this was that the ITL operating parameters were left unchanged from run to run, and thereby provided multiple instances of the same EO measurement conditions, although the acquisitions were captured over a span of a few weeks. The relevant EO runs acquired a series of dark images (with the nominal 15\,s integration time, or `EXPTIME') that followed a deliberate overexposure and readout of a FLAT (CCOB LED `red', target signal 400 ke$^-$/pix). The dark images acquired in succession following the FLAT image recorded the re-emitted or deferred signal collected within each 15\,s period, and there were 20 such dark images acquired within each EO run. In all, we identified and analyzed a total of 22 runs containing this data, where the excitation flat had the properties described above. The first and the twentieth dark images were stacked and medianed following a nominal instrumental signal removal (ISR) step. The twentieth median dark images were then subtracted from the first median darks. This further suppressed any remaining ISR residuals from the pixel data, which nominally now contain the {\it transient term} of the ITL phosphorescence, because as far as we could tell, the 15\,s expression of the deferred signal 300\,s after overexposure had almost completely attenuated.

\subsubsection{Results of phosphorescence detection in ITL sensors}\label{phos-results}

Table~\ref{tab:phosphorescence:datasets} provides the EO run IDs analyzed according to the process outlined above. Figures~\ref{fig:phos:R00} through \ref{fig:phos:R44} display the transient term in 8$\times$8 blocked images of the 12 rafts containing ITL sensors. These serve primarily to help identify which ITL sensors exhibit regions where we suspect presence of the phosphorescence effect. It should be noted that we retained the full 
1$\times$1 
pixel resolution images for follow-up inspection, because there is no guarantee that high spatial frequencies in the phosphorescence expression will not be washed out by the rebinning routinely performed for display purposes.

A subset of the 88 sensors, specifically those that either show high-signal diffuse, or morphologically unique structure in the transient term of the phosphorescence detected, are singled out to compare side-by-side with {\it blue} CCOB LED flat illumination, in Figures~\ref{fig:phos:stains:R01S00} through \ref{fig:phos:stains:R43S20} in the Appendix. It is apparent from viewing these side-by-side comparisons that generally, expression of phosphorescence has a complex relationship with the {\it much-easier-to-detect} coffee stains (or other diffuse variations in quantum efficiency) seen on the same sensors: Presence of a coffee stain seen in flat field response may be suggestive of phosphorescence on the sensor, but predicting where it might be (or its transient amplitude) is another matter entirely. In some cases (as in Fig.~\ref{fig:phos:stains} noted above), the phosphorescence appears to be correlated with the darker absorbed features of the coffee stain. In others (e.g., Fig.~\ref{fig:phos:stains:R02S02}), the opposite correlation is seen. In still other cases (e.g., Fig.~\ref{fig:phos:stains:R02S12}), there are regions of strong detail in the phosphorescence without very much coffee stain action at all. Our conclusions are that presence of coffee stains do not provide a useful proxy for the phosphorescent properties of the sensor.

\begin{center}
\begin{longtable}{lll}
\caption{Zephyr Scale E-numbers and corresponding SeqIDs analyzed to estimate phosphorescence in the 88 ITL sensors.} \label{phosphorescence:datasets} \\
%\hline 
\toprule\noalign{}
%\multicolumn{1}{c}{\textbf{Run Type}} & 
\multicolumn{3}{c}{\textbf{Run numbers and SeqIDs of first dark following trigger}} \\
%\multicolumn{1}{c}{\textbf{Third column}} \\ 
%\hline 
\midrule\noalign{}
\endfirsthead

\multicolumn{3}{c}%
{{\tablename\ \thetable{} -- continued from previous page}} \\
%\hline 
\toprule\noalign{}
%\multicolumn{1}{c}{\textbf{Run Type}} & 
\multicolumn{3}{c}{\textbf{Run numbers and SeqIDs of first dark following trigger}} \\
%\hline 
\midrule\noalign{}
\endhead

%\hline 
\midrule\noalign{}
\multicolumn{3}{r}{{Continued on next page}} \\ 
%\hline
\bottomrule\noalign{}
\endfoot

\hline \hline
\endlastfoot
\midrule\noalign{}
\multicolumn{3}{c}{B-protocol runs, HVBias {\it off}, HVBias {\it on} for Corners}\\
\midrule\noalign{}
E1003:20240920\_000056&E1009:20240921\_000222&E1003:20240920\_000056\\
\midrule\noalign{}
\multicolumn{3}{c}{B-protocol runs, HVBias {\it on}}\\
\midrule\noalign{}
E1071:20240924\_000300&E1110:20240926\_000242&E1144:20240927\_000369\\
E1146:20240928\_001525&E1195:20241002\_000235&E1245:20241003\_000245\\
E1290:20241008\_000286&E1329:20241011\_001555&E1363:20241012\_000546\\
E1392:20241014\_000444&E1396:20241014\_000701&E1411:20241015\_000322\\
E1419:20241016\_000397&E1429:20241016\_000742&E1449:20241017\_000548\\
E1497:20241020\_000225&E1812:20241028\_000481&E1880:20241030\_000432\\
E2233:20241108\_001468&E3380:20241130\_000355\\
\end{longtable}
\end{center}

\begin{figure}
\centering
\begin{minipage}{1.0\textwidth}    
  \centering
  \includegraphics[width=.95\linewidth]{sections/figures/phosphorescence-survey/stains_phos.png}    
\end{minipage}
\begin{minipage}{1.0\textwidth}
  \centering
  \includegraphics[width=.95\linewidth]{sections/figures/phosphorescence-survey/stains_abs.png}
\end{minipage}
\caption{R00\_SW1 image showing phosphorescence (top) with morphology similar to the ``coffee stains'' (bottom) observed with {\it blue} CCOB LED illumination. The phosphorescence acquired in dark exposures within the first 15\,s following trigger (top) uses a logarithmic stretch with limits 5--25 e$^-$/pixel. The {\it blue} flat field (bottom) is displayed normalized, with 4\% stretch limits (0.97 to 1.01), for a target signal level of $10^4$ e$^-$/pixel. Note that the phosphorescence pattern resembles the dark wisps in the flat (with opposite polarity) but that there are apparently no significant phosphorescence features corresponding to the bright wisps.}
\label{fig:phos:stains}
\end{figure}


While characterizing the phosphorescence expressed by ITL sensors using the data products described above, we have also identified correlations that concerns the localized, phosphorescence centers that tend to appear as circular disks. While we typically see a dozen or so (on average) per sensor, those with larger amplitude are strongly associated with {\it vampire pixels} (which are easily identified by their localized flat field response). The correlation is not perfect, meaning that not all localized (circular) phosphorescence centers can be associated with {\it vampire pixels} but that nearly all {\it vampire pixels} express localized phosphorescence with some amplitude. 

When data products of the 88 ITL sensors are inspected for transient phosphorescent response, very few, perhaps only a single sensor, show insignificant phosphorescence. Although $\sim$24\% of the ITL sensors show diffuse phosphorescence, a majority of sensors ($\sim$83\%) show spot-like phosphorescence centers. Presence of diffuse phosphorescence probably can frustrate spot-like phosphorescence detection by eye, and the estimated frequency of the latter may consequently serve as a lower limit to the true frequency. The identification of the sensor groups is given in Table~\ref{qualitative_assessment:itl_sensors}.

\begin{center}
\begin{longtable}{lll}
\caption[Qualitative grouping of ITL sensors]{
    Qualitative grouping of the 88 ITL sensors based on inspection of full resolution representations 
    of Figures~\ref{fig:phos:R00} through \ref{fig:phos:R44}. In cases of spot-like phosphorescence, 
    the numbers of features counted are given within parenthesis. Transient features appearing similar to 
    \textit{hot columns} or as other connected pixel groups are additionally signified with a double-plus (++).
} \label{qualitative_assessment:itl_sensors} \\
\toprule
\multicolumn{3}{c}{\textbf{Sensor Grouping}} \\
\midrule
\endfirsthead

\multicolumn{3}{c}{{\tablename\ \thetable{} -- continued from previous page}} \\
\toprule
\multicolumn{3}{c}{\textbf{Sensor Grouping}} \\
\midrule
\endhead

\midrule
\multicolumn{3}{r}{{Continued on next page}} \\
\bottomrule
\endfoot

\bottomrule
\endlastfoot

\multicolumn{3}{c}{Sensors exhibiting insignificant phosphorescence} \\
\midrule
R44\_SW1 \\
\midrule
\multicolumn{3}{c}{Spot-like phosphorescence (vampire transients)} \\
\midrule
R00\_SG0($>$36) & R00\_SG1($>$36) & R00\_SW0($>$10) \\
R01\_S00($>$33) & R01\_S01($>$4) & R01\_S02($>$6) \\
R01\_S10($>$25) & R01\_S11(18) & R01\_S12(14) \\
R01\_S20($>$23) & R01\_S21($>$30) & R01\_S22($>$30) \\
R02\_S00($>$32++) & R02\_S01($>$36) & R02\_S02($>$28) \\
R02\_S10(6) & R02\_S11($>$30) & R02\_S12($>$25) \\
R02\_S20($>$14) & R02\_S21($>$9) & R02\_S22($>$6++) \\
R03\_S00(13) & R03\_S01(12) & R03\_S02($>$19) \\
R03\_S10(9) & R03\_S11(3) & R03\_S12(10) \\
R03\_S20(9) & R03\_S21(18++) & R03\_S22(16) \\
R04\_SG0($>$12) & R04\_SG1($>$30++) & R04\_SW0(25) \\
R04\_SW1($>$30) & R10\_S00($>$30) & R10\_S01(9) \\
R10\_S02(32) & R10\_S11(16) & R10\_S12($>$26) \\
R10\_S20(21) & R10\_S21($>$11++) & R10\_S22($>$10++) \\
R20\_S00(2) & R20\_S01(8) & R20\_S02(7) \\
R20\_S10($>$35) & R20\_S11(7) & R20\_S12(5) \\
R20\_S20(10) & R20\_S21(5) & R20\_S22(5) \\
R40\_SG0($>$50++) & R40\_SG1(6++) & R40\_SW0(6) \\
R40\_SW1(8) & R41\_S00(9++) & R41\_S01(16) \\
R41\_S02(10) & R41\_S10(12) & R41\_S11(3) \\
R41\_S12(10++) & R41\_S20(5++) & R41\_S21($\sim$30) \\
R41\_S22(3) & R42\_S00(24) & R42\_S01(6) \\
R42\_S02($>$10) & R42\_S10(4) & R42\_S11(11) \\
R42\_S12(33) & R42\_S20(7) & R42\_S21(5) \\
R42\_S22(4) & R43\_S00(22++) & R43\_S01(30) \\
R43\_S02(19) & R43\_S10(26) & R43\_S12(8++) \\
R43\_S21(14) & R43\_S22(4) & R44\_SG0($>$12) \\
R44\_SG1($>$10) & R44\_SW0(18) & \\
\midrule
\multicolumn{3}{c}{Segments exhibiting diffuse transient phosphorescence} \\
\midrule
R00\_SG1\_C10-12,C03-05 (++) & R00\_SW0\_C17 & R00\_SW1\_C** (++) \\
R01\_S00\_C13-14 (++) & R01\_S01\_C07,C16-17 & R01\_S10\_C00-01,C14-16 \\
R01\_S20\_C04-07 & R01\_S21\_C06-07,C17 & R01\_S22\_C00-01,C15-17 \\
R02\_S02\_C03-04 & R02\_S11\_C13-17,C07 (++) & R02\_S12\_C04-07,C10-12 \\
R02\_S20\_C06-07 & R04\_SG1\_C01,C11 (++) & R10\_S10\_C10,C16-17,C07 \\
R40\_SG0 (++) & R41\_S21\_C00,C10 & R42\_S00\_C01,C07,C17 \\
R43\_S11 (++) & R43\_S20\_C00-01 (++) & R44\_SG1\_C07 \\
\end{longtable}
\end{center}



The correspondence between {\it vampire pixels} and spot-like phosphorescence is laid out in Figure~\ref{fig:vamp-phos:R03_S10-R20_S20}, for two prominent cases.  These two {\it vampire pixels} may appear intrinsically different in that their flat-field responses {\bf do} (or do {\bf not}) exhibit a central bright pixel, which could aid in their identification. Details of the underlying distribution of trapped surface charges near the back-side electrode - or variations in the conductive properties of the same - apparently drive these details of the flat field response. However, it remains intriguing that these surface electrostatic properties are accompanied by an unmistakable transient phosphorescence signature.


\begin{figure}[!htbp]
\centering
\begin{subfigure}{0.45\textwidth}
    \includegraphics[width=\textwidth]{figures/phosphorescence-survey/vamp_comp_R03_S10_flatresp.png}
     \caption{flat field (blue) response, R03\_S10 ROI}
     \label{subfig:flatresp_R03_S10}
\end{subfigure}
\begin{subfigure}{0.45\textwidth}
    \includegraphics[width=\textwidth]{figures/phosphorescence-survey/vamp_comp_R03_S10_phosresp.png}
     \caption{transient phosphorescence, R03\_S10 ROI}
     \label{subfig:phosresp_R03_S10}
\end{subfigure}
\newline
\begin{subfigure}{0.45\textwidth}
    \includegraphics[width=\textwidth]{figures/phosphorescence-survey/vamp_comp_R20_S20_flatresp.png}
     \caption{flat field (blue) response, R20\_S20 ROI}
     \label{subfig:flatresp_R20_S20}
\end{subfigure}
\begin{subfigure}{0.45\textwidth}
    \includegraphics[width=\textwidth]{figures/phosphorescence-survey/vamp_comp_R20_S20_phosresp.png}
     \caption{transient phosphorescence, R20\_S20 ROI}
     \label{subfig:phosresp_R20_S20}
\end{subfigure}
\newline
\caption{Vampire pixel comparisons between their flat field response and their transient phosphorescence. Signal levels are given (relative for flat field response, absolute electrons per 15s following overexposure for transient phosphorescence). The relative flat field response amplitudes swing between 0.2 \& 16 (reaching full well) for R03\_S10, and between 0.4 \& 8 for R20\_S20. The transient phosphorescence response also reaches nominal full well (135ke$^-$/pix/15s for the central pixel) for R03\_S10, and a lower amplitude (3-4Ke$^-$/pix/15s for several hundred pixels) is reached for R20\_S20.}
\label{fig:vamp-phos:R03_S10-R20_S20}
\end{figure}

A curious aspect of the phosphorescence seen in ITL sensors lies in its voltage (HV Bias) dependence. The HV Bias, when turned on, reduces lateral diffusion of the photo-conversions and thereby maintains PSF image quality. In Figure~\ref{fig:phos:hvbias_comp} we compare side-by-side several phosphorescent regions with both HVBias states (off and on). There appears to be no trend that lends to predictability in these cases. In the cases of vampire pixels (R03\_S10 \& R20\_S20), the geometry of the phosphorescence is indeed very sensitive to the HV Bias states (cf. Figs \ref{subfig:hvb_off_R03_S10} vs. \ref{subfig:hvb_on_R03_S10}; \ref{subfig:hvb_off_R20_S20} vs. \ref{subfig:hvb_on_R20_S20}). These might be understood qualitatively However, for the diffuse phosphorescence examples, the expression appears to vanish entirely (R43\_S11, Fig.~\ref{subfig:hvb_off_R43_S11}) or become significantly stronger, together with morphological changes (R43\_S20, Fig.~\ref{subfig:hvb_off_R43_S20}) when the HV Bias is switched off.

\begin{figure}[!htbp]
\centering
 \begin{subfigure}{0.35\textwidth}
     \includegraphics[width=\textwidth]{figures/phosphorescence-survey/hvbias_comp_R03_S10_off.png}
     \caption{HVBias off, R03\_S10 ROI}
     \label{subfig:hvb_off_R03_S10}
 \end{subfigure}
 \begin{subfigure}{0.35\textwidth}
     \includegraphics[width=\textwidth]{figures/phosphorescence-survey/hvbias_comp_R03_S10_on.png} 
     \caption{HVBias on, R03\_S10 ROI}
     \label{subfig:hvb_on_R03_S10}
 \end{subfigure}  
 \newline
 \begin{subfigure}{0.35\textwidth}
     \includegraphics[width=\textwidth]{figures/phosphorescence-survey/hvbias_comp_R20_S20_off.png}
     \caption{HVBias off, R20\_S20 ROI}
     \label{subfig:hvb_off_R20_S20}
 \end{subfigure}
 \begin{subfigure}{0.35\textwidth}
     \includegraphics[width=\textwidth]{figures/phosphorescence-survey/hvbias_comp_R20_S20_on.png}
     \caption{HVBias on, R20\_S20 ROI}
     \label{subfig:hvb_on_R20_S20}
 \end{subfigure}
 \newline
 \begin{subfigure}{0.35\textwidth}
     \includegraphics[width=\textwidth]{figures/phosphorescence-survey/hvbias_comp_R43_S11_off.png}
     \caption{HVBias off, R43\_S11 ROI}
     \label{subfig:hvb_off_R43_S11}
 \end{subfigure}
 \begin{subfigure}{0.35\textwidth}
     \includegraphics[width=\textwidth]{figures/phosphorescence-survey/hvbias_comp_R43_S11_on.png}
     \caption{HVBias on, R43\_S11 ROI}
     \label{subfig:hvb_on_R43_S11}
 \end{subfigure} 
 \newline
 \begin{subfigure}{0.35\textwidth}
     \includegraphics[width=\textwidth]{figures/phosphorescence-survey/hvbias_comp_R43_S20_off.png}
     \caption{HVBias off, R43\_S20 ROI}
     \label{subfig:hvb_off_R43_S20}
 \end{subfigure}     
 \begin{subfigure}{0.35\textwidth}
     \includegraphics[width=\textwidth]{figures/phosphorescence-survey/hvbias_comp_R43_S20_on.png}
     \caption{HVBias on, R43\_S20 ROI}
     \label{subfig:hvb_on_R43_S20}
 \end{subfigure}
 \newline
\caption{Comparisons of transient phosphorescence between conditions where HV Bias is off (left) vs. on (right). Four different ROIs are shown, but with image scales set to match across HV Bias conditions.}
\label{fig:phos:hvbias_comp}
\end{figure}


\subsubsection{Other properties of phosphorescence}
\begin{itemize}
\tightlist
\item
  Dependence on HVBiasOn vs. HVBiasOff
\item
  Dependence on wavelength of the triggering exposure
\item
  Kinetics of the phosphorescence (based on {\it blue} CCOB LED)
  \item 
  Phosphorescence response: triggering exposure dependence of the expressed phosphorescence, the wavelength- and signal level-dependence. 
\end{itemize}


\begin{itemize}
\tightlist
\item
  phosphorescence background
\item
  phosphorescence on flat fields
\item
  phosphorescence on spot projections
\end{itemize}

\section{Conclusions}\label{conclusions}

\subsection{Run 7 final operating
parameters}\label{run-7-final-operating-parameters}

This section describes the conclusions of Run 7 optimization and the
operating conditions of the camera. Decisions regarding these parameters
were based upon the results of the
\href{https://sitcomtn-148.lsst.io/\#persistence-optimization}{voltage
optimization},
\href{https://sitcomtn-148.lsst.io/\#sequencer-optimization}{sequencer
optimization}, and
\href{https://sitcomtn-148.lsst.io/\#thermal-optimization}{thermal
optimization}.

\subsubsection{Voltage conditions}\label{voltage-conditions}

\begin{longtable}{@{}l|c|cc@{}}
\caption{Voltage Conditions with Updated Run 5 for ITL Values} \\
\toprule
Parameter & Run 5 (ITL) & Run 5 (dp93; e2v) & Run 7 (dp80; new voltage for e2v) \\
\midrule
\endfirsthead
\toprule
Parameter & Run 5 (ITL) & Run 5 (dp93; e2v) & Run 7 (dp80; new voltage for e2v) \\
\midrule
\endhead
\bottomrule
\endfoot
pclkHigh & 2.0 & 3.3 & 2.0 \\
pclkLow & $-$8.0 & $-$6.0 & $-$6.0 \\
dpclk & 10.0 & 9.3 & 8.0 \\
sclkHigh & 5.0 & 3.9 & 3.55 \\
sclkLow & $-$5.0 & $-$5.4 & $-$5.75 \\
rgHigh & 8.0 & 6.1 & 5.01 \\
rgLow & $-$2.0 & $-$4.0 & $-$4.99 \\
rd & 13.0 & 11.6 & 10.5 \\
od & 26.9 & 23.4 & 22.3 \\
og & $-$2.0 & $-$3.4 & $-$3.75 \\
gd & 20.0 & 26.0 & 26.0 \\
\end{longtable}



\subsubsection{Sequencer conditions}\label{sequencer-conditions}

\begin{table}[h!]
\centering
\caption{Sequencer conditions}
\begin{tabular}{ll}
\toprule
Detector type & File name \\
\midrule
e2v & FP\_E2V\_2s\_l3cp\_v30.seq \\
ITL & FP\_ITL\_2s\_l3cp\_v30.seq \\
\bottomrule
\end{tabular}
\end{table}


\begin{itemize}
\tightlist
\item
  v30 sequencers are identical to the
  FP\_ITL\_2s\_l3cp\_v29\_Noppp.seq
  and
  FP\_E2V\_2s\_l3cp\_v29\_NopSf.seq.
  All sequencer files can be found in the \href{https://github.com/lsst-camera-dh/sequencer-files/tree/master/run7}{GitHub
  repository}.
\end{itemize}

\subsubsection{Other camera conditions}\label{other-camera-conditions}

\begin{itemize}
\tightlist
\item
  Idle flush disabled
\end{itemize}

\subsection{Record runs}\label{record-runs} 

This section describes Run 7 record runs. All runs use our camera operating configuration, unless otherwise noted.

\begin{longtable}{|p{2.5cm}|p{1.5cm}|p{2.9cm}|p{6.5cm}|}
\hline
\textbf{Run Type} & \textbf{Run ID} & \textbf{Links} & \textbf{Notes} \\ \hline
\endfirsthead
\hline
\textbf{Run Type} & \textbf{Run ID} & \textbf{Links} & \textbf{Notes} \\ \hline
\endhead
\hline
\endfoot
\hline
\endlastfoot

\multirow{3}{*}{B-protocol} & E1880 & Web report \newline Test Execution & Initial B\_protocol taken with the new v30 definition. \\ \cline{2-4} 
                            & E2233 &             & dp80, first run after full CCS system reboot \\ \cline{2-4} 
                            & E3380 &             & First B protocol post-chiller recovery \\ \hline

\multirow{7}{*}{PTC}        & E3630 &             & Low flux red LED PTC, ND1 filter installed. Final operating conditions of camera. \\ \cline{2-4} 
                            & E3577 &             & Dense nm960 PTC. Final operating conditions of camera. \\ \cline{2-4} 
                            & E2237 &             & Final operating conditions of camera. Red LED dense. Acquired after CCS subsystem reboot. \\ \cline{2-4} 
                            & E748  &             & Final operating conditions of camera. nm960 dense \\ \cline{2-4} 
                            & E2016 &             & Final operating conditions of camera. Super dense red LED. HV Bias off for R13/Reb2. jGroups meltdown interrupted acquisitions, restarted \\ \cline{2-4} 
                            & E1886 &             & Final operating conditions of camera. Red LED dense. Dark interleaving between flat pairs \\ \cline{2-4} 
                            & E1881 &             & Final operating conditions of camera. Red LED dense. No dark interleaving between flat pairs \\ \hline

\multirow{3}{*}{Gain Stability} & E1955 &          & 6h Stability run 10k 750 nm V30, dp80, idle flush disabled \\ \cline{2-4} 
                                & E2008 &          & 6h Stability run 10k 750 nm V30, dp80, idle flush disabled, after zero-ing CCOB \\ \hline

\multirow{3}{*}{Long dark acquisitions} & E3540 &  & 900s dark. Shutter closed. \\ \cline{2-4} 
                                       & E3539 &  & 900s dark. Shutter closed. \\ \cline{2-4} 
                                       & E3538 &  & 900s dark. Shutter opened. \\ \hline

\multirow{2}{*}{Projector acquisitions} & E2184 &  & 10 30\,s dark images to capture background pattern. E2V:v29Nop, ITL:v29Nopp \\ \cline{2-4} 
                                        & E2181 &  & Flat pairs from 2--60\,s in 2\,s intervals. Two 15\,s darks interleaved after flat acquisition. Rectangle on C10 amplifier. E2V:v29Nop, ITL:v29Nopp \\ \hline

\multirow{2}{*}{OpSim runs}  & E2330 &           & Short dark sequence, filter changes in headers through OCS \\ \cline{2-4} 
                             & E2328 &           & Flats with shutter-controlled exposure \\ \hline

\multirow{5}{*}{Phosphorescence datasets} & E2015 & & 10 flats at 10 ke$^-$ followed by 10$\times$15\,s darks \\ \cline{2-4} 
                                          & E2014 & & 1 flat at 10 ke$^-$ followed by 10$\times$15\,s darks \\ \cline{2-4} 
                                          & E2013 & & 10 flats at 10 ke$^-$ followed by 10$\times$15\,s darks. Interleaved biases with the darks \\ \cline{2-4} 
                                          & E2012 & & 10 flats at 1 ke$^-$ followed by 10$\times$15\,s darks \\ \cline{2-4} 
                                          & E2011 & & 20 flats at 10 ke$^-$ followed by 10$\times$15\,s darks \\ \hline

\end{longtable}

% \begin{table}[H]
% \centering
% \begin{tabular}{|p{2.5cm}|p{1.5cm}|p{2.9cm}|p{6.5cm}|}
% \hline
% Run Type                        & Run ID & Links & Notes                                                                   \\ \hline
% \multirow{3}{*}{B-protocol}     & E1880  &  Web report \newline Test Execution     & Initial B\_protocol taken with the new v30 definition.                  \\ \cline{2-4} 
%                                 & E2233  &       & dp80, first run after full CCS system reboot                            \\ \cline{2-4} 
%                                 & E3380  &       & First B protocol post-chiller recovery                                  \\ \hline

% \multirow{7}{*}{PTC} & E3630 & & Low flux red LED PTC, ND1 filter installed. Final operating conditions of camera.\\ \cline{2-4} 
% & E3577 & & Dense nm960 PTC. Final operating conditions of camera.\\ \cline{2-4} 
% & E2237 & & Final operating conditions of camera. Red LED dense. Acquired after CCS subsystem reboot. \\ \cline{2-4} 
% & E748 & & Final operating conditions of camera. nm960 dense \\ \cline{2-4} 
% & E2016 & & Final operating conditions of camera. Super dense red LED. HV Bias off for R13/Reb2. jGroups meltdown interrupted acquisitions, restarted \\ \cline{2-4} 
% & E1886 & & Final operating conditions of camera. Red LED dense. Dark interleaving between flat pairs \\ \cline{2-4} 
% & E1881 & & Final operating conditions of camera. Red LED dense. No dark interleaving between flat pairs \\ \hline
% \multirow{3}{*}{Gain Stability} & E1955 & & 6h Stability run 10k 750 nm V30, dp80, idle flush disabled\\ \cline{2-4}
%  & E2008 & & 6h Stability run 10k 750 nm V30, dp80, idle flush disabled, after zero-ing CCOB\\ \cline{2-4}
% \multirow{3}{*}{Long dark acquisitions} & E3540 & & 900s dark. Shutter closed. \\ \cline{2-4}
%  & E3539 & & 900s dark. Shutter closed. \\ \cline{2-4}
%  & E3538 & & 900s dark. Shutter opened. \\ \hline
% \multirow{2}{*}{Projector acquisitions} & E2184 & & 10 30\,s dark images to capture background pattern. E2V:v29Nop, ITL:v29Nopp \\ \cline{2-4}
% & E2181 & & Flat pairs from 2--60\,s in 2\,s intervals. Two 15\,s darks interleaved after flat acquisition. Rectangle on C10 amplifier. E2V:v29Nop, ITL:v29Nopp \\ \hline
% \multirow{2}{*}{OpSim runs}  & E2330 & & Short dark sequence, filter changes in headers through OCS \\ \cline{2-4}
%  & E2328 & & Flats with shutter-controlled exposure \\ \hline
% \multirow{5}{*}{Phosphorescence datasets} & E2015 & & 10 flats at 10 ke$^-$ followed by 10$\times$15\,s darks \\ \cline{2-4}
%  & E2014 & & 1 flat at 10 ke$^-$ followed by 10$\times$15\,s darks \\ \cline{2-4}
%  & E2013 & & 10 flats at 10 ke$^-$ followed by 10$\times$15\,s darks. Interleaved biases with the darks \\ \cline{2-4}
%  & E2012 & & 10 flats at 1 ke$^-$ followed by 10$\times$15\,s darks \\ \cline{2-4}
%  & E2011 & & 20 flats at 10 ke$^-$ followed by 10$\times$15\,s darks \\ \hline
% \end{tabular}
% \end{table}

\subsection{Other runs of relevance}\label{relevant-runs}

Runs that use the Run 7 final camera operating configuration (Sec.~\ref{run-7-final-operating-parameters}) are denoted with \textbf{bold run ID}.

% B Protocol
\begin{table}[H]\label{table:runs_BProtocol}
\centering
\caption{B Protocol Runs}
\begin{tabular}{|p{1.5cm}|p{2.9cm}|p{9cm}|}
\hline
Run ID & Links & Notes \\ \hline
\textbf{E3380} & & First B protocol post-chiller recovery. v30, dp80, idle flush disabled. \\ \hline
\textbf{E2233} & & Identical to E1880. Acquired after CCS subsystem reboot. dp80, idle flush disabled. \\ \hline
\textbf{E1880}  &  Web report \newline Test Execution & Camera operating configuration \\ \hline
E1812 & & v29 NopSf (no pocket serial flush running for both e2v and ITL clear sequencers). dp80 voltages, idle flush ?? [likely disabled but verification needed] \\ \hline
E1497 & & v29 Nop sequencer, dp80, idle flush ?? [likely disabled but verification needed] \\ \hline
E1429 & & First dp84 run. v29, idle flush disabled \\ \hline
E1419 & & First dp88 run. v29, idle flush disabled \\ \hline
E1411 & & First dp865 run. v29, idle flush disabled \\ \hline
E1396 & & First dp80 run. v29 nonoverlapping sequencer, idle flush enabled \\ \hline
E1392 & & First dp80 run. v29 sequencer, idle flush enabled \\ \hline
E1290 & & using Guide sensors as guiders. v29, dp93, idle flush enabled \\ \hline
E1245 & & Refrigeration system software update mid-run. v29 halfoverlapping sequencer. dp93, idle flush enabled \\ \hline
E1195 & & v29 overlap113 sequencer (5\% overlap). dp93, idle flush enabled \\ \hline
E1146 & & First run with v29 nonoverlapping. dp93, idle flush enabled \\ \hline
E1144 & & First run with v29 Nop. dp93, idle flush enabled \\ \hline
E1110 & & v29 run. dp93, idle flush enabled \\ \hline
E1071 & & SOURCE = 63 in calib3.cfg. First run with HV on. dp93, v26 sequencer, idle flush enabled \\ \hline

\end{tabular}
\end{table}

% PTCs
\begin{table}[H]\label{table:runs_PTCs}
\centering
\caption{PTC Runs}
\begin{tabular}{|p{1.5cm}|p{2.9cm}|p{9cm}|}
\hline

Run ID & Links & Notes \\ \hline

\textbf{E3630} & & Low flux red LED PTC, ND1 filter installed. Final operating conditions of camera.\\ \hline
\textbf{E3577} & & Dense nm960 PTC. Final operating conditions of camera.\\ \hline
\textbf{E2237} & & Final operating conditions of camera. Red LED dense. Acquired after CCS subsystem reboot. \\ \hline
\textbf{E748} & & Final operating conditions of camera. nm960 dense \\ \hline
\textbf{E2016} & & Final operating conditions of camera. Super dense red LED. HV Bias off for R13/Reb2. jGroups meltdown interrupted acquisitions, restarted \\ \hline
\textbf{E1886} & & Final operating conditions of camera. Red LED dense. Dark interleaving between flat pairs \\ \hline
\textbf{E1881} & & Final operating conditions of camera. Red LED dense. No dark interleaving between flat pairs \\ \hline
E1765 & & Dense PTC, red, thresholded dark interleaves, overlaps in signal level for adjacent LED currents. v29 Nop sequencer, idle flush ??\\ \hline
E1495 & & dp80, nopp config. Idle flush ??\\ \hline
E1364 & & v29, dp80, idle flush ??. Possible incomplete data transfer\\ \hline
E1335 & & dp80 configuration, v29, idle flush ??.\\ \hline
E1275 & & Ordered flats. Failed dark interleaving, incomplete data transfer. v29 sequencer.\\ \hline
E1259 & & Randomized flats. v29 sequencer. \\ \hline
E1258 & & Randomized flux levels. Starting with 3 preimages, then 100 15s darks, then PTC set. No dark interleaving. v29 sequencer. \\ \hline
E1247 & & Re-do of E1188 (which lacked PD data). v29HalfOverlapping., Added pre-image acquisition to PTC-Red cfg file. \\ \hline
E1212 & & 5\% overlapping sequencer\\ \hline
E1145 & & No pocket sequencer\\ \hline
E1113 & & v29 sequencer\\ \hline
E749 & & v26, dp93, idle flush enabled. First PTC of run. \\ \hline
\end{tabular}
\end{table}

% Long Dark Acquisitions
\begin{table}[H]\label{table:runs_dark}
\centering
\caption{Long Dark Acquisitions}
\begin{tabular}{|p{1.5cm}|p{2.9cm}|p{9cm}|}
\hline
Run ID & Links & Notes \\ \hline
\textbf{E3540} & & 900s dark. Shutter closed. \\ \hline
\textbf{E3539} & & 900s dark. Shutter closed. \\ \hline
\textbf{E3538} & & 900s dark. Shutter opened. \\ \hline
E1140 & & Empty frame filter, shutter open, 24V clean and dirty FES changer powered off, one 900s dark image only.\\ \hline
E1117 & & 900s dark. r filter, shutter open.\\ \hline
E1116 & & 900s dark. y filter, shutter open.\\ \hline
E1115 & & 900s dark. g filter, shutter open.\\ \hline
E1114 & & 900s dark. EF filter, shutter open.\\ \hline
E1076 & & PH filter in place. Shutter open. v26 no RG\\ \hline
E1075 & & PH filter in place. v26 no RG\\ \hline
\end{tabular}
\end{table}

% Projector Acquisitions
\begin{table}[H]\label{table:runs_projector}
\centering
\caption{Projector Acquisitions}
\begin{tabular}{|p{1.5cm}|p{2.9cm}|p{9cm}|}
\hline
Run ID & Links & Notes \\ \hline
E2184 & & 10 30\,s dark images to capture background pattern \\ \hline
E2181 & & Flat pairs from 2--60\,s in 2\,s intervals. Two 15\,s darks interleaved after flat acquisition. Rectangle on C10 amplifier. e2v:v29Nop, ITL:v29Nopp \\ \hline
E1586 & & One 100\,s flat exposure, spots moved to selected phosphorescent regions. e2v:v29Nop, ITL:v29Nopp \\ \hline
E1558 & & Flat pairs, fine scan in flux from 1--100\,s in 1\,s intervals. e2v:v29Nop, ITL:v29Nopp \\ \hline
E1553 & & Flat pairs, coarse scan in flux from 5--120\,s in 5\,s intervals. e2v:v29Nop, ITL:v29Nopp \\ \hline
\end{tabular}
\end{table}

% OpSim Runs
\begin{table}[H]\label{table:runs_opSim}
\centering
\caption{OpSim Runs}
\begin{tabular}{|p{1.5cm}|p{2.9cm}|p{9cm}|}
\hline
Run ID & Links & Notes \\ \hline


\textbf{E3629} & & Mock OCS calibrations, failed\\ \hline 
\textbf{E3576} & & Mock OCS calibrations, failed\\ \hline 
\textbf{E3570} & & Mock OCS calibrations, failed\\ \hline 
\textbf{E2330} & & Short dark sequence, filter changes in headers through OCS \\ \hline
\textbf{E2329} & & Mock OCS calibrations, failed\\ \hline 
\textbf{E2328} & & Flats with shutter-controlled exposure \\ \hline
\textbf{E2283} & & Full night of OpSim flats, failed \\ \hline
\textbf{E2280} & & Mock OCS calibrations, failed\\ \hline 
\textbf{E2279} & & Mock OCS calibrations, failed\\ \hline 
E1717 & & Long dark sequence, no filter changes \\ \hline
E1657 & & 10 hour OpSim dark run, \textasciitilde50\% of darks were acquired properly \\ \hline
E1414 & & 30 minutes OpSim run with shutter control, filter change, and realistic survey cadence \\ \hline
E1403 & & 30 minutes OpSim run with shutter control, filter change, and realistic survey cadence \\ \hline
E1255 & & 30 minutes OpSim run with shutter control, filter change, and realistic survey cadence \\ \hline
E1254 & & 30 minutes OpSim run with shutter control, filter change, and realistic survey cadence \\ \hline
E1092 & & 30 minutes OpSim run with shutter control, filter change, and realistic survey cadence \\ \hline

\end{tabular}
\end{table}

% Phosphorescence Datasets
\begin{table}[H]\label{table:runs_phosphorescence}
\centering
\caption{Phosphorescence Datasets}
\begin{tabular}{|p{1.5cm}|p{2.9cm}|p{9cm}|}
\hline
Run ID & Links & Notes \\ \hline
\textbf{E2015} & & 10 flats at 10 ke$^-$ followed by 10$\times$15\,s darks \\ \hline
\textbf{E2014} & & 1 flat at 10 ke$^-$ followed by 10$\times$15\,s darks \\ \hline
\textbf{E2013} & & 10 flats at 10 ke$^-$ followed by 10$\times$15\,s darks. Interleaved biases with the darks \\ \hline
\textbf{E2012} & & 10 flats at 1 ke$^-$ followed by 10$\times$15\,s darks \\ \hline
\textbf{E2011} & & 20 flats at 10 ke$^-$ followed by 10$\times$15\,s darks \\ \hline
\end{tabular}
\end{table}

% Tree Ring Flats
\begin{table}[H]\label{table:runs_treeRing}
\centering
\caption{Tree Ring Flats}
\begin{tabular}{|p{1.5cm}|p{2.9cm}|p{9cm}|}
\hline
Run ID & Links & Notes \\ \hline
E1050 & & Red LED. HV off. Diffuser installed. \\ \hline
E1052 & & Blue LED. HV off. Diffuser installed. \\ \hline
E1053 & & Nm750 LED. HV off. Diffuser installed. \\ \hline
E1055 & & Nm850 LED. HV off. Diffuser installed. \\ \hline
E1056 & & Nm960 LED. HV off. Diffuser installed. \\ \hline
\hline
E1021 & & Red LED. HV off. Diffuser removed. \\ \hline
E1023 & & Blue LED. HV off. Diffuser removed. \\ \hline
E1024 & & Nm750 LED. HV off. Diffuser removed. \\ \hline
E1025 & & Nm850 LED. HV off. Diffuser removed. \\ \hline
E1026 & & Nm960 LED. HV off. Diffuser removed. \\ \hline
\end{tabular}
\end{table}

% Gain Stability Runs
\begin{table}[H]\label{table:runs_gainStability}
\centering
\caption{Gain Stability Runs}
\begin{tabular}{|p{1.5cm}|p{2.9cm}|p{9cm}|}
\hline
Run ID & Links & Notes \\ \hline
\textbf{E1955} & & 6h Stability run 10k 750 nm V30, dp80, idle flush disabled\\ \hline
\textbf{E2008} & & 6h Stability run 10k 750 nm V30, dp80, idle flush disabled, after zero-ing CCOB\\ \hline
\textbf{E1968} & & 6h Stability run 2k 750 nm V30, dp80, idle flush disabled\\ \hline
E1367 & & Changing PCS setpoint mid run., PCS changed from -45 deg C to -47 deg C at 10:40:06 AM UTC. 6h, 50k at 750nm, v29 seq, dp80 config. \\ \hline
E1362 & & dp80, partial data ingestion. v29 sequencer. 6h 10k at 750nm. \\ \hline
E756 & & dp 80, v29 sequencer. 6h 10k at 750nm. Partial data ingestion.\\ \hline
E1496 & & dp80, nopp config, 6h 750nm at 10k\\ \hline
\end{tabular}
\end{table}

% Persistence Datasets
\begin{table}[H]
\centering
\caption{Persistence Datasets\label{table:runs_persistence}}
\begin{tabular}{|p{1.5cm}|p{2.9cm}|p{9cm}|}
\hline
Run ID & Links & Notes \\ \hline
\textbf{E2286} & & 30k uv flash with increased hilim\\ \hline
E1507 & & dp80, uv led @ 30k \\ \hline
E1506 & & dp80, uv led @ 10k\\ \hline
E1505 & & dp80, uv led @ 5k\\ \hline
E1504 & & dp80, uv led @ 3k\\ \hline
E1503 & & dp80, uv led @ 1k\\ \hline
E1502 & & dp80, blue led @ 1k\\ \hline
E1501 & & dp80, blue led @ 3k\\ \hline
E1500 & & dp80, blue led @ 5k\\ \hline
E1499 & & dp80, blue led @ 10k\\ \hline
E1498 & & dp80, blue led @ 30k\\ \hline
E1494 & & dp80, nm960 led @ 50k\\ \hline
E1493 & & dp80, nm850 led @ 50k\\ \hline
E1492 & & dp80, nm750 led @ 50k\\ \hline
E1491 & & dp80, blue led @ 50k\\ \hline
E1490 & & dp80, red led @ 50k\\ \hline
E1489 & & dp80, nm960 led @ 150k\\ \hline
E1488 & & dp80, nm850 led @ 150k\\ \hline
E1487 & & dp80, nm750 led @ 150k\\ \hline
E1486 & & dp80, red led @ 150k\\ \hline
E1485 & & dp80, blue led @ 150k\\ \hline
E1484 & & dp80, blue led @ 400k\\ \hline
E1483 & & dp80, red led @ 400k\\ \hline
E1479 & & dp80, nm750 led @ 400k\\ \hline
E1478 & & dp80, nm960 led @ 400k\\ \hline
E1477 & & dp80, nm850 led @ 400k\\ \hline
\end{tabular}
\end{table}

% Guider ROI Acquisitions
\begin{table}[H]\label{table:runs_guider}
\centering
\caption{Guider ROI Acquisitions}
\begin{tabular}{|p{1.5cm}|p{2.9cm}|p{9cm}|}
\hline
Run ID & Links & Notes \\ \hline
E1509 & & ROI reference dataset\\ \hline
E1510 & & ROI crossing amplifier segments\\ \hline
E1518 & & 200ms integration time\\ \hline
E1519 & & 100ms integration time\\ \hline
E1508 & & 50ms integration time\\ \hline
E1520 & & 400x400 pixel ROIs\\ \hline
E1511 & & 200x200 pixel ROIs\\ \hline
E1521 & & 100x100 pixel ROIs\\ \hline
E1512 & & New row from reference dataset\\ \hline
E1513 & & New column from reference dataset\\ \hline
E1514 & & New column and row from reference dataset\\ \hline
E1517 & & Different row for sensors on the same REB\\ \hline
\end{tabular}
\end{table}



\section{References}


\appendix

\section{FCS work}

\section{Reference figures}

% All the figures from Jim's web reports that have been used here
% 5x5 plots
% differential histograms

\section{CCS work}
\subsection{JGroups issue}

\section{OCS integration}

\section{Phosphorescence identification on ITL set of sensors}
\label{appendix:phos:ident}
\begin{figure}[!htbp]
\centering
\includegraphics[width=0.9\textwidth]{figures/phosphorescence-survey/itl_fluor_R00_0-19_rb1_log.png}
\caption{Phosphorescence transients for the R00 CRTM captured in the first 15\,s following {\it red} CCOB LED at 400\,ke$^-$/pix. With 8$\times$8 blocking, the upper end of the color scale (640) corresponds to 10\,e$^-$/pixel when averaged over 64 pixels contributing.}
\label{fig:phos:R00}
\end{figure}

\begin{figure}[!htbp]
\centering
\includegraphics[width=0.9\textwidth]{figures/phosphorescence-survey/itl_fluor_R01_0-19_rb1_log.png}
\caption{Phosphorescence transients for the R01 RTM captured in the first 15\,s following {\it red} CCOB LED at 400\,ke$^-$/pix. With 8$\times$8 blocking, the upper end of the color scale (640) corresponds to 10 e$^-$/pixel when averaged over 64 pixels contributing.}
\label{fig:phos:R01}
\end{figure}

\begin{figure}[!htbp]
\centering
\includegraphics[width=0.9\textwidth]{figures/phosphorescence-survey/itl_fluor_R02_0-19_rb1_log.png}
\caption{Phosphorescence transients for the R02 RTM captured in the first 15\,s following {\it red} CCOB LED at 400\,ke$^-$/pix. With 8$\times$8 blocking, the upper end of the color scale (640) corresponds to 10 e$^-$/pixel when averaged over 64 pixels contributing.}
\label{fig:phos:R02}
\end{figure}

\begin{figure}[!htbp]
\centering
\includegraphics[width=0.9\textwidth]{figures/phosphorescence-survey/itl_fluor_R03_0-19_rb1_log.png}
\caption{Phosphorescence transients for the R03 RTM captured in the first 15\,s following {\it red} CCOB LED at 400\,ke$^-$/pix. With 8$\times$8 blocking, the upper end of the color scale (640) corresponds to 10 e$^-$/pixel when averaged over 64 pixels contributing.}
\label{fig:phos:R03}
\end{figure}

\begin{figure}[!htbp]
\centering
\includegraphics[width=0.9\textwidth]{figures/phosphorescence-survey/itl_fluor_R04_0-19_rb1_log.png}
\caption{Phosphorescence transients for the R04 CRTM captured in the first 15\,s following {\it red} CCOB LED at 400\,ke$^-$/pix. With 8$\times$8 blocking, the upper end of the color scale (640) corresponds to 10 e$^-$/pixel when averaged over 64 pixels contributing.}
\label{fig:phos:R04}
\end{figure}

\begin{figure}[!htbp]
\centering
\includegraphics[width=0.9\textwidth]{figures/phosphorescence-survey/itl_fluor_R10_0-19_rb1_log.png}
\caption{Phosphorescence transients for the R10 RTM captured in the first 15\,s following {\it red} CCOB LED at 400\,ke$^-$. With 8$\times$8 blocking, the upper end of the color scale (640) corresponds to 10 e$^-$/pixel when averaged over 64 pixels contributing.}
\label{fig:phos:R10}
\end{figure}

\begin{figure}[!htbp]
\centering
\includegraphics[width=0.9\textwidth]{figures/phosphorescence-survey/itl_fluor_R20_0-19_rb1_log.png}
\caption{Phosphorescence transients for the R20 RTM captured in the first 15\,s following {\it red} CCOB LED at 400\,ke$^-$/pix. With 8$\times$8 blocking, the upper end of the color scale (640) corresponds to 10 e$^-$/pixel when averaged over 64 pixels contributing.}
\label{fig:phos:R20}
\end{figure}

\begin{figure}[!htbp]
\centering
\includegraphics[width=0.9\textwidth]{figures/phosphorescence-survey/itl_fluor_R40_0-19_rb1_log.png}
\caption{Phosphorescence transients for the R40 CRTM captured in the first 15\,s following {\it red} CCOB LED at 400\,ke$^-$/pix. With 8$\times$8 blocking, the upper end of the color scale (640) corresponds to 10 e$^-$/pixel when averaged over 64 pixels contributing.}
\label{fig:phos:R40}
\end{figure}

\begin{figure}[!htbp]
\centering
\includegraphics[width=0.9\textwidth]{figures/phosphorescence-survey/itl_fluor_R41_0-19_rb1_log.png}
\caption{Phosphorescence transients for the R41 RTM captured in the first 15\,s following {\it red} CCOB LED at 400\,ke$^-$/pix. With 8$\times$8 blocking, the upper end of the color scale (640) corresponds to 10 e$^-$/pixel when averaged over 64 pixels contributing.}
\label{fig:phos:R41}
\end{figure}

\begin{figure}[!htbp]
\centering
\includegraphics[width=0.9\textwidth]{figures/phosphorescence-survey/itl_fluor_R42_0-19_rb1_log.png}
\caption{Phosphorescence transients for the R42 RTM captured in the first 15\,s following {\it red} CCOB LED at 400\,ke$^-$/pix. With 8$\times$8 blocking, the upper end of the color scale (640) corresponds to 10 e$^-$/pixel when averaged over 64 pixels contributing.}
\label{fig:phos:R42}
\end{figure}

\begin{figure}[!htbp]
\centering
\includegraphics[width=0.9\textwidth]{figures/phosphorescence-survey/itl_fluor_R43_0-19_rb1_log.png}
\caption{Phosphorescence transients for the R43 RTM captured in the first 15\,s following {\it red} CCOB LED at 400\,ke$^-$/pix. With 8$\times$8 blocking, the upper end of the color scale (640) corresponds to 10 e$^-$/pixel when averaged over 64 pixels contributing.}
\label{fig:phos:R43}
\end{figure}

\begin{figure}[!htbp]
\centering
\includegraphics[width=0.9\textwidth]{figures/phosphorescence-survey/itl_fluor_R44_0-19_rb1_log.png}
\caption{Phosphorescence transients for the R44 CRTM captured in the first 15\,s following {\it red} CCOB LED at 400\,ke$^-$/pix. With 8$\times$8 blocking, the upper end of the color scale (640) corresponds to 10 e$^-$/pixel when averaged over 64 pixels contributing.}
\label{fig:phos:R44}
\end{figure}

\section{Phosphorescence morphological comparisons with features seen in {\it blue} flat field response}

Figures~\ref{fig:phos:stains:R01S00} through \ref{fig:phos:stains:R43S20} are an incomplete selection of ITL sensors with phosphorescence. They compare expressed phosphorescence (transient term) with the {\it blue} CCOB LED flat response. Inspection of these images would lead one to conclude that in certain cases, the phosphorescence patterns resemble the coffee stain patterns' regions of lower QE at short wavelength ({\it cf.} Fig.~\ref{fig:phos:stains}, Fig.~\ref{fig:phos:stains:R43S11}). In other cases, the opposite appears to be true ({\it cf.} Fig.~\ref{fig:phos:stains:R02S02}, Fig.~\ref{fig:phos:stains:R02S12}). In several cases, there appear to be no particular correlations. 

In cases where variations in the blue flat-field response are due to {\it vampire pixels} ({\it cf.} Fig.~\ref{fig:vamp-phos:R03_S10-R20_S20}) with a completely different wavelength dependence, presumably due to depth-dependence in the direction of the drift field lines), we see {\it high amplitude} and {\it long timescale} transient phosphorescence associated with these {\it vampire pixel} complexes. These tend to be the brightest phosphorescent features we see, and this fact may provide a strong hint regarding the origin of this phosphorescence phenomena. These quantitative differences are most easily seen in the kinetics discussion, Section~\ref{sect:kinetics}.

\begin{figure}[!htbp]
\centering
\begin{minipage}{1.0\textwidth}    
  \centering
  \includegraphics[width=.6\linewidth]{figures/phosphorescence-survey/stains_phos_R01_S00.png}    
\end{minipage}
\begin{minipage}{1.0\textwidth}
  \centering
  \includegraphics[width=.6\linewidth]{figures/phosphorescence-survey/stains_abs_R01_S00.png}
\end{minipage}
\caption{The ITL sensor R01\_S00. Top: the transient phosphorescence term. Bottom: the {\it blue} flat response. The large, extended spot appears to be centered on a {\it vampire pixel}, which also expresses a large amplitude of phosphorescence, which emits enough current to contaminate the parallel overscan in at least the first 15\,s exposure following trigger. The flat response feature has opposite polarity from the phosphorescence.}
\label{fig:phos:stains:R01S00}
\end{figure}


\begin{figure}[!htbp]
\centering
\begin{minipage}{1.0\textwidth}    
  \centering
  \includegraphics[width=.6\linewidth]{figures/phosphorescence-survey/stains_phos_R02_S02.png}    
\end{minipage}
\begin{minipage}{1.0\textwidth}
  \centering
  \includegraphics[width=.6\linewidth]{figures/phosphorescence-survey/stains_abs_R02_S02.png}
\end{minipage}
\caption{The ITL sensor R02\_S02. Top: the transient phosphorescence term. Bottom: the {\it blue} flat response. The {\it coffee stain} feature in the flat response has the same polarity as the phosphorescence. A phosphorescent {\it vampire pixel} is seen in segment R02\_S02\_C07.}
\label{fig:phos:stains:R02S02}
\end{figure}

\begin{figure}[!htbp]
\centering
\begin{minipage}{1.0\textwidth}    
  \centering
  \includegraphics[width=.6\linewidth]{figures/phosphorescence-survey/stains_phos_R02_S12.png}    
\end{minipage}
\begin{minipage}{1.0\textwidth}
  \centering
  \includegraphics[width=.6\linewidth]{figures/phosphorescence-survey/stains_abs_R02_S12.png}
\end{minipage}
\caption{The ITL sensor R02\_S12. Top: the transient phosphorescence term. Bottom: the {\it blue} flat response. Generally the polarity of the phosphorescence matches that of the {\it coffee stain} in the flat field response, but exceptions include the {\it vampire pixel} seen in segment R02\_S12\_C05.}
\label{fig:phos:stains:R02S12}
\end{figure}


\begin{figure}[!htbp]
\centering
\begin{minipage}{1.0\textwidth}    
  \centering
  \includegraphics[width=.6\linewidth]{figures/phosphorescence-survey/stains_phos_R03_S10.png}    
\end{minipage}
\begin{minipage}{1.0\textwidth}
  \centering
  \includegraphics[width=.6\linewidth]{figures/phosphorescence-survey/stains_abs_R03_S10.png}
\end{minipage}
\caption{The ITL sensor R03\_S10, detail of the {\it vampire pixel} of R03\_S10\_C15. Top: the transient phosphorescence term. Bottom: the {\it blue} flat response. As in previous examples, this {\it vampire pixel}'s transient term is large enough to contaminate the parallel overscan even after the first 15\,s following trigger.}
\label{fig:phos:stains:R03S10}
\end{figure}


\begin{figure}[!htbp]
\centering
\begin{minipage}{1.0\textwidth}    
  \centering
  \includegraphics[width=.6\linewidth]{figures/phosphorescence-survey/stains_phos_R43_S11.png}    
\end{minipage}
\begin{minipage}{1.0\textwidth}
  \centering
  \includegraphics[width=.6\linewidth]{figures/phosphorescence-survey/stains_abs_R43_S11.png}
\end{minipage}
\caption{The ITL sensor R43\_S11. Top: the transient phosphorescence term. Bottom: the {\it blue} flat response. This sensor appears to have the largest integrated phosphorescence among ITL sensors studied. The flat response feature has opposite polarity from the phosphorescence.}
\label{fig:phos:stains:R43S11}
\end{figure}


\begin{figure}[!htbp]
\centering
\begin{minipage}{1.0\textwidth}    
  \centering
  \includegraphics[width=.6\linewidth]{figures/phosphorescence-survey/stains_phos_R43_S20_detail.png}    
\end{minipage}
\begin{minipage}{1.0\textwidth}
  \centering
  \includegraphics[width=.6\linewidth]{figures/phosphorescence-survey/stains_abs_R43_S20_detail.png}
\end{minipage}
\caption{The ITL sensor R43\_S20, segments C00 through C03. Top: the transient phosphorescence term. Bottom: the {\it blue} flat response. This sensor apparently exhibits peculiar radial crazing patterns seen in both phosphorescence as well as in flat field response, with polarities aligned.}
\label{fig:phos:stains:R43S20}
\end{figure}

\section{Phosphorescence kinetics characterization}
\label{sect:kinetics}
Figures~\ref{fig:phos:kinetics:R01S00} through \ref{fig:phos:kinetics:R43S20} quantify the expressed phosphorescence distributions in ROIs on seven of the problematic ITL sensors. Previously, we had captured the phosphorescence {\it transient term} across the ITL sensors ({\it cf.} Figs.~\ref{fig:phos:R00} thru \ref{fig:phos:R44}); here we track ROI pixel distribution parameters of individual median images constructed from the selection of specific images acquired across the 20 B-protocol datasets available (listed in Table~\ref{tab:phosphorescence:datasets}). 

By fitting decay models to these persistence curves, it is immediately clear that there are multiple (>2) timescales at play for the pixels in each ROI. An example of such a fit is given in Figure~\ref{fig:phos:kinetics:fit:R20S20C13} where a 3-population relaxation model is used to characterize evolution of the 99\% quantile level of the distribution. In this case, there are three different exponential timescales determined: $(\tau_1,\tau_2,\tau_3) = (0.62,2.5,18.3)$ in image units (10.9, 43.8 \& 320 seconds, respectively). The corresponding ratio of these populations works out to 4.5\% (fast), 21.5\% (medium) and 74\% (slow), respectively. Inspection of the more detailed parameters plotted generally indicate skewed distributions from mismatches between medians and means; the choice of the 99\% quantile level to characterize was mainly to estimate the degree to which images would need to be phosphorescence-corrected (and/or the variance plane modified, given the asymmetric impact of the position specific, phosphorescence contribution in recorded images). 

\begin{figure}[!htbp]
\centering
\begin{subfigure}{0.8\textwidth}    
  \centering
  \includegraphics[width=\textwidth]{figures/phosphorescence-survey/phos_kinetics/R20_S20_sel_1820-1920_535-635_phos_decay_fit.png}    
\end{subfigure}
\caption{A three-population fit of the phosphorescence expressed by the vapire pixel region of R20\_S20\_C13. The fit was performed on the 99\% quantile level where signal levels are well above the $3\sigma$ level of the noise distribution. Here, image numbers are parasitically used as time units, with roughly 17.5 seconds per image.}
\label{fig:phos:kinetics:fit:R20S20C13}
\end{figure}

\begin{figure}[!htbp]
\begin{subfigure}{0.45\textwidth}    
  \centering
  \includegraphics[width=\textwidth]{figures/phosphorescence-survey/phos_kinetics/R01_S00_sel_1961-2020_1268-1325_cumdist.png}    
\end{subfigure}
\hfil
\begin{subfigure}{0.45\textwidth}
  \centering
  \includegraphics[width=\textwidth]{figures/phosphorescence-survey/phos_kinetics/R01_S00_sel_1961-2020_1268-1325_phos_decay.png}
\end{subfigure}
\newline
\begin{subfigure}{0.45\textwidth}    
  \centering
  \includegraphics[width=\textwidth]{figures/phosphorescence-survey/phos_kinetics/R01_S00_sel_1983-2042_1065-1121_cumdist.png}    
\end{subfigure}
\hfil
\begin{subfigure}{0.45\textwidth}
  \centering
  \includegraphics[width=\textwidth]{figures/phosphorescence-survey/phos_kinetics/R01_S00_sel_1983-2042_1065-1121_phos_decay.png}
\end{subfigure}
\newline
\begin{subfigure}{0.45\textwidth}    
  \centering
  \includegraphics[width=\textwidth]{figures/phosphorescence-survey/phos_kinetics/R01_S00_sel_2082-2141_1185-1240_cumdist.png}    
\end{subfigure}
\hfil
\begin{subfigure}{0.45\textwidth}
  \centering
  \includegraphics[width=\textwidth]{figures/phosphorescence-survey/phos_kinetics/R01_S00_sel_2082-2141_1185-1240_phos_decay.png}
\end{subfigure}
\newline
\caption{Kinetics for phosphorescence expression in ROIs of images for R01\_S00. This is the prominent cosmetic seen in Fig.~\ref{fig:phos:stains:R01S00}, which is apparently a {\it vampire} pixel.}
\label{fig:phos:kinetics:R01S00}
\end{figure}

\begin{figure}[!htbp]
\begin{subfigure}{0.45\textwidth}    
  \centering
  \includegraphics[width=\textwidth]{figures/phosphorescence-survey/phos_kinetics/R02_S02_sel_1724-1829_3558-3666_cumdist.png}    
\end{subfigure}
\hfil
\begin{subfigure}{0.45\textwidth}
  \centering
  \includegraphics[width=\textwidth]{figures/phosphorescence-survey/phos_kinetics/R02_S02_sel_1724-1829_3558-3666_phos_decay.png}
\end{subfigure}
\newline
\begin{subfigure}{0.45\textwidth}    
  \centering
  \includegraphics[width=\textwidth]{figures/phosphorescence-survey/phos_kinetics/R02_S02_sel_1774-1880_3791-3900_cumdist.png}    
\end{subfigure}
\hfil
\begin{subfigure}{0.45\textwidth}
  \centering
  \includegraphics[width=\textwidth]{figures/phosphorescence-survey/phos_kinetics/R02_S02_sel_1774-1880_3791-3900_phos_decay.png}
\end{subfigure}
\newline
\begin{subfigure}{0.45\textwidth}    
  \centering
  \includegraphics[width=\textwidth]{figures/phosphorescence-survey/phos_kinetics/R02_S02_sel_2100-2206_3851-3959_cumdist.png}    
\end{subfigure}
\hfil
\begin{subfigure}{0.45\textwidth}
  \centering
  \includegraphics[width=\textwidth]{figures/phosphorescence-survey/phos_kinetics/R02_S02_sel_2100-2206_3851-3959_phos_decay.png}
\end{subfigure}
\newline
\caption{Kinetics for phosphorescence expression in ROIs of images for R02\_S02. This is the diffuse phosphorescence that correlates with the coffee stains seen in Fig.~\ref{fig:phos:stains:R02S02}. No extraction was performed on the {\it vampire pixel} found on the same sensor (R02\_S02\_C07).}
\label{fig:phos:kinetics:R02S02}
\end{figure}

\begin{figure}[!htbp]
\begin{subfigure}{0.45\textwidth}    
  \centering
  \includegraphics[width=\textwidth]{figures/phosphorescence-survey/phos_kinetics/R02_S12_sel_2556-2578_3445-3469_cumdist.png}    
\end{subfigure}
\hfil
\begin{subfigure}{0.45\textwidth}
  \centering
  \includegraphics[width=\textwidth]{figures/phosphorescence-survey/phos_kinetics/R02_S12_sel_2556-2578_3445-3469_phos_decay.png}
\end{subfigure}
\newline
\begin{subfigure}{0.45\textwidth}    
  \centering
  \includegraphics[width=\textwidth]{figures/phosphorescence-survey/phos_kinetics/R02_S12_sel_2695-2859_3353-3524_cumdist.png}    
\end{subfigure}
\hfil
\begin{subfigure}{0.45\textwidth}
  \centering
  \includegraphics[width=\textwidth]{figures/phosphorescence-survey/phos_kinetics/R02_S12_sel_2695-2859_3353-3524_phos_decay.png}
\end{subfigure}
\newline
\begin{subfigure}{0.45\textwidth}    
  \centering
  \includegraphics[width=\textwidth]{figures/phosphorescence-survey/phos_kinetics/R02_S12_sel_523-676_9-129_cumdist.png}    
\end{subfigure}
\hfil
\begin{subfigure}{0.45\textwidth}
  \centering
  \includegraphics[width=\textwidth]{figures/phosphorescence-survey/phos_kinetics/R02_S12_sel_523-676_9-129_phos_decay.png}
\end{subfigure}
\newline
\caption{Kinetics for phosphorescence expression in ROIs of images for R02\_S12. This is the structured phosphorescence that correlates with the coffee stains seen in Fig.~\ref{fig:phos:stains:R02S12}.}
\label{fig:phos:kinetics:R02S12}
\end{figure}

\begin{figure}[!htbp]
\begin{subfigure}{0.45\textwidth}    
  \centering
  \includegraphics[width=\textwidth]{figures/phosphorescence-survey/phos_kinetics/R03_S10_sel_2969-3010_1763-1800_cumdist.png}    
\end{subfigure}
\hfil
\begin{subfigure}{0.45\textwidth}
  \centering
  \includegraphics[width=\textwidth]{figures/phosphorescence-survey/phos_kinetics/R03_S10_sel_2969-3010_1763-1800_phos_decay.png}
\end{subfigure}
\newline
\begin{subfigure}{0.45\textwidth}    
  \centering
  \includegraphics[width=\textwidth]{figures/phosphorescence-survey/phos_kinetics/R03_S10_sel_2970-2985_1767-1797_cumdist.png}    
\end{subfigure}
\hfil
\begin{subfigure}{0.45\textwidth}
  \centering
  \includegraphics[width=\textwidth]{figures/phosphorescence-survey/phos_kinetics/R03_S10_sel_2970-2985_1767-1797_phos_decay.png}
\end{subfigure}
\newline
\begin{subfigure}{0.45\textwidth}    
  \centering
  \includegraphics[width=\textwidth]{figures/phosphorescence-survey/phos_kinetics/R03_S10_sel_2995-3009_1767-1797_cumdist.png}    
\end{subfigure}
\hfil
\begin{subfigure}{0.45\textwidth}
  \centering
  \includegraphics[width=\textwidth]{figures/phosphorescence-survey/phos_kinetics/R03_S10_sel_2995-3009_1767-1797_phos_decay.png}
\end{subfigure}
\newline
\caption{Kinetics for phosphorescence expression in ROIs of images for R03\_S10. These describe regions including or near the bright/focusing {\it vampire pixel} seen in Figs.~\ref{fig:phos:stains:R03S10}, \ref{subfig:phosresp_R03_S10} and \ref{subfig:hvb_on_R03_S10}.}
\label{fig:phos:kinetics:R03S10}
\end{figure}

\begin{figure}[!htbp]
\begin{subfigure}{0.45\textwidth}    
  \centering
  \includegraphics[width=\textwidth]{figures/phosphorescence-survey/phos_kinetics/R20_S20_sel_1820-1920_535-635_cumdist.png}    
\end{subfigure}
\hfil
\begin{subfigure}{0.45\textwidth}
  \centering
  \includegraphics[width=\textwidth]{figures/phosphorescence-survey/phos_kinetics/R20_S20_sel_1820-1920_535-635_phos_decay.png}
\end{subfigure}
\newline
\caption{Kinetics for phosphorescence expression in ROIs of images for R20\_S20. These describe the prominent non-focusing {\it vampire pixel} seen in Figs.~\ref{subfig:phosresp_R20_S20} and \ref{subfig:hvb_on_R20_S20}.}
\label{fig:phos:kinetics:R20S20}
\end{figure}

\begin{figure}[!htbp]
\begin{subfigure}{0.45\textwidth}    
  \centering
  \includegraphics[width=\textwidth]{figures/phosphorescence-survey/phos_kinetics/R43_S11_sel_1741-1955_3754-3806_cumdist.png}    
\end{subfigure}
\hfil
\begin{subfigure}{0.45\textwidth}
  \centering
  \includegraphics[width=\textwidth]{figures/phosphorescence-survey/phos_kinetics/R43_S11_sel_1741-1955_3754-3806_phos_decay.png}
\end{subfigure}
\newline
\begin{subfigure}{0.45\textwidth}    
  \centering
  \includegraphics[width=\textwidth]{figures/phosphorescence-survey/phos_kinetics/R43_S11_sel_1763-1976_3826-3878_cumdist.png}    
\end{subfigure}
\hfil
\begin{subfigure}{0.45\textwidth}
  \centering
  \includegraphics[width=\textwidth]{figures/phosphorescence-survey/phos_kinetics/R43_S11_sel_1763-1976_3826-3878_phos_decay.png}
\end{subfigure}
\newline
\begin{subfigure}{0.45\textwidth}    
  \centering
  \includegraphics[width=\textwidth]{figures/phosphorescence-survey/phos_kinetics/R43_S11_sel_40-90_2751-2957_cumdist.png}    
\end{subfigure}
\hfil
\begin{subfigure}{0.45\textwidth}
  \centering
  \includegraphics[width=\textwidth]{figures/phosphorescence-survey/phos_kinetics/R43_S11_sel_40-90_2751-2957_phos_decay.png}
\end{subfigure}
\newline
\caption{Kinetics for phosphorescence expression in ROIs of images for R43\_S11. These describe bright, diffuse transient regions seen in Figs.~\ref{fig:phos:stains:R43S11} and \ref{subfig:hvb_on_R43_S11}, which apparently turn off completely when the HV Bias is {\it off}.}
\label{fig:phos:kinetics:R43S11}
\end{figure}

\begin{figure}[!htbp]
\begin{subfigure}{0.45\textwidth}    
  \centering
  \includegraphics[width=\textwidth]{figures/phosphorescence-survey/phos_kinetics/R43_S20_sel_207-292_3352-3435_cumdist.png}    
\end{subfigure}
\hfil
\begin{subfigure}{0.45\textwidth}
  \centering
  \includegraphics[width=\textwidth]{figures/phosphorescence-survey/phos_kinetics/R43_S20_sel_207-292_3352-3435_phos_decay.png}
\end{subfigure}
\newline
\begin{subfigure}{0.45\textwidth}    
  \centering
  \includegraphics[width=\textwidth]{figures/phosphorescence-survey/phos_kinetics/R43_S20_sel_547-633_3346-3430_cumdist.png}    
\end{subfigure}
\hfil
\begin{subfigure}{0.45\textwidth}
  \centering
  \includegraphics[width=\textwidth]{figures/phosphorescence-survey/phos_kinetics/R43_S20_sel_547-633_3346-3430_phos_decay.png}
\end{subfigure}
\newline
\begin{subfigure}{0.45\textwidth}    
  \centering
  \includegraphics[width=\textwidth]{figures/phosphorescence-survey/phos_kinetics/R43_S20_sel_707-816_3373-3589_cumdist.png}    
\end{subfigure}
\hfil
\begin{subfigure}{0.45\textwidth}
  \centering
  \includegraphics[width=\textwidth]{figures/phosphorescence-survey/phos_kinetics/R43_S20_sel_707-816_3373-3589_phos_decay.png}
\end{subfigure}
\newline
\caption{Kinetics for phosphorescence expression in ROIs of images for R43\_S20. These include some of the the highly structured {\it snowflake-like} transient regions seen in Figs.~\ref{subfig:hvb_on_R43_S20} and \ref{fig:phos:stains:R43S20}.}
\label{fig:phos:kinetics:R43S20}
\end{figure}

\section{Phosphorescence response characterization}
\label{sect:response}
Figures~\ref{fig:phos:resp:R01S00} through \ref{fig:phos:resp:R43S20} attempt to quantify the expressed phosphorescence response in ROIs on seven of the problematic ITL sensors. Previously, we had captured the phosphorescence {\it transient term} across the ITL sensors ({\it cf.} Figs.~\ref{fig:phos:R00} thru \ref{fig:phos:R44}); we also tracked ROI pixel distribution parameters of individual median images constructed from the selection of specific images acquired across the 20 B-protocol datasets available (listed in Table~\ref{tab:phosphorescence:datasets}). Here we analyze the signal level- and wavelength-dependences of the expressed phosphorescence captured in the first dark image following flat exposure. Table~XX provides the image numbers.. 

Because these runs were performed to sample a two dimensional parameter space,  that would lead to 

By fitting decay models to these persistence curves, it is immediately clear that there are multiple (>2) timescales at play for the pixels in each ROI. An example of such a fit is given in Figure~\ref{fig:phos:kinetics:fit:R20S20C13} where a 3-population relaxation model is used to characterize evolution of the 99\% quantile level of the distribution. In this case, there are three different exponential timescales determined: $(\tau_1,\tau_2,\tau_3) = (0.62,2.5,18.3)$ in image units (10.9, 43.8 \& 320 seconds, respectively). The corresponding ratio of these populations works out to 4.5\% (fast), 21.5\% (medium) and 74\% (slow), respectively. Inspection of the more detailed parameters plotted generally indicate skewed distributions from mismatches between medians and means; the choice of the 99\% quantile level to characterize was mainly to estimate the degree to which images would need to be phosphorescence-corrected (and/or the variance plane modified, given the asymmetric impact of the position specific, phosphorescence contribution in recorded images). 

%\begin{figure}[!htbp]
%\centering
%\begin{subfigure}{0.8\textwidth}    
%  \centering
%  \includegraphics[width=\textwidth]{figures/phosphorescence-%survey/phos_kinetics/R20_S20_sel_1820-1920_535-635_phos_decay_fit.png}    
%\end{subfigure}
%\caption{A three-population fit of the phosphorescence expressed by the %vapire pixel region of R20\_S20\_C13. The fit was performed on the 99\% %quantile level where signal levels are well above the $3\sigma$ level of %the noise distribution. Here, image numbers are parasitically used as %time units, with roughly 17.5 seconds per image.}
%\label{fig:phos:resp:fit:R20S20C13}
%\end{figure}

\begin{figure}[!htbp]
\centering
\begin{subfigure}{0.45\textwidth}    
  \centering
  \includegraphics[width=\textwidth]{figures/phosphorescence-survey/phos_resp/resp_99_R01_S00_1961-2020_1268-1325.png}    
\end{subfigure}
\newline
\centering
\begin{subfigure}{0.45\textwidth}    
  \centering
  \includegraphics[width=\textwidth]{figures/phosphorescence-survey/phos_resp/resp_99_R01_S00_1983-2042_1065-1121.png}    
\end{subfigure}
\newline
\centering
\begin{subfigure}{0.45\textwidth}    
  \centering
  \includegraphics[width=\textwidth]{figures/phosphorescence-survey/phos_resp/resp_99_R01_S00_2082-2141_1185-1240.png}    
\end{subfigure}
\newline
\caption{Signal and wavelength response for phosphorescence expression (99\% level) in ROIs of images for R01\_S00. This is the prominent cosmetic seen in Fig.~\ref{fig:phos:stains:R01S00}, which is apparently a {\it vampire} pixel.}
\label{fig:phos:resp:R01S00}
\end{figure}

\begin{figure}[!htbp]
\centering
\begin{subfigure}{0.45\textwidth}    
  \centering
  \includegraphics[width=\textwidth]{figures/phosphorescence-survey/phos_resp/resp_99_R02_S02_1724-1829_3558-3666.png}    
\end{subfigure}
\newline
\centering
\begin{subfigure}{0.45\textwidth}    
  \centering
  \includegraphics[width=\textwidth]{figures/phosphorescence-survey/phos_resp/resp_99_R02_S02_1774-1880_3791-3900.png}    
\end{subfigure}
\newline
\centering
\begin{subfigure}{0.45\textwidth}    
  \centering
  \includegraphics[width=\textwidth]{figures/phosphorescence-survey/phos_resp/resp_99_R02_S02_2100-2206_3851-3959.png}    
\end{subfigure}
\newline
\caption{Signal and wavelength response for phosphorescence expression (99\% level) in ROIs of images for R02\_S02. This is the diffuse phosphorescence that correlates with the coffee stains seen in Fig.~\ref{fig:phos:stains:R02S02}. No extractions were performed on the {\it vampire pixels} found on the same sensor (R02\_S02\_C15 and R02\_S02\_C07).}
\label{fig:phos:resp:R02S02}
\end{figure}

\begin{figure}[!htbp]
\centering
\begin{subfigure}{0.45\textwidth}    
  \centering
  \includegraphics[width=\textwidth]{figures/phosphorescence-survey/phos_resp/resp_99_R02_S12_2556-2578_3445-3469.png}    
\end{subfigure}
\newline
\centering
\begin{subfigure}{0.45\textwidth}    
  \centering
  \includegraphics[width=\textwidth]{figures/phosphorescence-survey/phos_resp/resp_99_R02_S12_2695-2859_3353-3524.png}    
\end{subfigure}
\newline
\centering
\begin{subfigure}{0.45\textwidth}    
  \centering
  \includegraphics[width=\textwidth]{figures/phosphorescence-survey/phos_resp/resp_99_R02_S12_523-676_9-129.png}    
\end{subfigure}
\newline
\caption{Signal and wavelength response for phosphorescence expression (99\% level) in ROIs of images for R02\_S12. This is the structured phosphorescence that correlates with the coffee stains seen in Fig.~\ref{fig:phos:stains:R02S12}.}
\label{fig:phos:resp:R02S12}
\end{figure}

\begin{figure}[!htbp]
\centering
\begin{subfigure}{0.45\textwidth}    
  \centering
  \includegraphics[width=\textwidth]{figures/phosphorescence-survey/phos_resp/resp_99_R03_S10_2969-3010_1763-1800.png}    
\end{subfigure}
\newline
\centering
\begin{subfigure}{0.45\textwidth}    
  \centering
  \includegraphics[width=\textwidth]{figures/phosphorescence-survey/phos_resp/resp_99_R03_S10_2970-2985_1767-1797.png}    
\end{subfigure}
\newline
\centering
\begin{subfigure}{0.45\textwidth}    
  \centering
  \includegraphics[width=\textwidth]{figures/phosphorescence-survey/phos_resp/resp_99_R03_S10_2995-3009_1767-1797.png}    
\end{subfigure}
\newline
\caption{Signal and wavelength response for phosphorescence expression (99\% level) in ROIs of images for R03\_S10. These describe regions including or near the bright/focusing {\it vampire pixel} seen in Figs.~\ref{fig:phos:stains:R03S10}, \ref{subfig:phosresp_R03_S10} and \ref{subfig:hvb_on_R03_S10}.}
\label{fig:phos:resp:R03S10}
\end{figure}

\begin{figure}[!htbp]
\centering
\begin{subfigure}{0.45\textwidth}    
  \centering
  \includegraphics[width=\textwidth]{figures/phosphorescence-survey/phos_resp/resp_99_R20_S20_1820-1920_535-635.png}    
\end{subfigure}
\newline
\caption{Signal and wavelength response for phosphorescence expression (99\% level) in an ROI of images for R20\_S20. These describe the prominent non-focusing {\it vampire pixel} seen in Figs.~\ref{subfig:phosresp_R20_S20} and \ref{subfig:hvb_on_R20_S20}.}
\label{fig:phos:kinetics:R20S20}
\end{figure}

\begin{figure}[!htbp]
\centering
\begin{subfigure}{0.45\textwidth}    
  \centering
  \includegraphics[width=\textwidth]{figures/phosphorescence-survey/phos_resp/resp_99_R43_S11_1741-1955_3754-3806.png}    
\end{subfigure}
\newline
\centering
\begin{subfigure}{0.45\textwidth}    
  \centering
  \includegraphics[width=\textwidth]{figures/phosphorescence-survey/phos_resp/resp_99_R43_S11_1763-1976_3826-3878.png}    
\end{subfigure}
\newline
\centering
\begin{subfigure}{0.45\textwidth}    
  \centering
  \includegraphics[width=\textwidth]{figures/phosphorescence-survey/phos_resp/resp_99_R43_S11_40-90_2751-2957.png}    
\end{subfigure}
\newline
\caption{Signal and wavelength response for phosphorescence expression (99\% level) in ROIs of images for R43\_S11. These describe bright, diffuse transient regions seen in Figs.~\ref{fig:phos:stains:R43S11} and \ref{subfig:hvb_on_R43_S11}, which apparently turn off completely when the HV Bias is {\it off}.}
\label{fig:phos:resp:R43S11}
\end{figure}

\begin{figure}[!htbp]
\centering
\begin{subfigure}{0.45\textwidth}    
  \centering
  \includegraphics[width=\textwidth]{figures/phosphorescence-survey/phos_resp/resp_99_R43_S20_207-292_3352-3435.png}    
\end{subfigure}
\newline
\centering
\begin{subfigure}{0.45\textwidth}    
  \centering
  \includegraphics[width=\textwidth]{figures/phosphorescence-survey/phos_resp/resp_99_R43_S20_547-633_3346-3430.png}    
\end{subfigure}
\newline
\centering
\begin{subfigure}{0.45\textwidth}    
  \centering
  \includegraphics[width=\textwidth]{figures/phosphorescence-survey/phos_resp/resp_99_R43_S20_707-816_3373-3589.png}    
\end{subfigure}
\newline
\caption{Signal and wavelength response for phosphorescence expression (99\% level) in ROIs of images for R43\_S20. These include some of the the highly structured {\it snowflake-like} transient regions seen in Figs.~\ref{subfig:hvb_on_R43_S20} and \ref{fig:phos:stains:R43S20}.}
\label{fig:phos:resp:R43S20}
\end{figure}

